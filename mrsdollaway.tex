\documentclass[lang=cn,10pt]{elegantbook}

\title{Mrs. Dalloway}
\subtitle{Virginia Woolf}

\setcounter{tocdepth}{3}

\logo{logo-blue.png}
\cover{cover.jpg}

% 本文档命令
\usepackage{array}
\newcommand{\ccr}[1]{\makecell{{\color{#1}\rule{1cm}{1cm}}}}

% 修改标题页的橙色带
% \definecolor{customcolor}{RGB}{32,178,170}
% \colorlet{coverlinecolor}{customcolor}

\begin{document}

\maketitle
\frontmatter

\tableofcontents

\mainmatter

Mrs. Dalloway said she would buy the flowers herself.

For Lucy had her work cut out for her.  The doors would be taken
off their hinges; Rumpelmayer's men were coming.  And then, thought
Clarissa Dalloway, what a morning--fresh as if issued to children
on a beach.

What a lark!  What a plunge!  For so it had always seemed to her,
when, with a little squeak of the hinges, which she could hear now,
she had burst open the French windows and plunged at Bourton into
the open air.  How fresh, how calm, stiller than this of course,
the air was in the early morning; like the flap of a wave; the kiss
of a wave; chill and sharp and yet (for a girl of eighteen as she
then was) solemn, feeling as she did, standing there at the open
window, that something awful was about to happen; looking at the
flowers, at the trees with the smoke winding off them and the rooks
rising, falling; standing and looking until Peter Walsh said,
"Musing among the vegetables?"--was that it?--"I prefer men to
cauliflowers"--was that it?  He must have said it at breakfast one
morning when she had gone out on to the terrace--Peter Walsh.  He
would be back from India one of these days, June or July, she
forgot which, for his letters were awfully dull; it was his sayings
one remembered; his eyes, his pocket-knife, his smile, his
grumpiness and, when millions of things had utterly vanished--how
strange it was!--a few sayings like this about cabbages.

She stiffened a little on the kerb, waiting for Durtnall's van to
pass.  A charming woman, Scrope Purvis thought her (knowing her as
one does know people who live next door to one in Westminster); a
touch of the bird about her, of the jay, blue-green, light,
vivacious, though she was over fifty, and grown very white since
her illness.  There she perched, never seeing him, waiting to
cross, very upright.

For having lived in Westminster--how many years now? over twenty,--
one feels even in the midst of the traffic, or waking at night,
Clarissa was positive, a particular hush, or solemnity; an
indescribable pause; a suspense (but that might be her heart,
affected, they said, by influenza) before Big Ben strikes.  There!
Out it boomed.  First a warning, musical; then the hour,
irrevocable.  The leaden circles dissolved in the air.  Such fools
we are, she thought, crossing Victoria Street.  For Heaven only
knows why one loves it so, how one sees it so, making it up,
building it round one, tumbling it, creating it every moment
afresh; but the veriest frumps, the most dejected of miseries
sitting on doorsteps (drink their downfall) do the same; can't be
dealt with, she felt positive, by Acts of Parliament for that very
reason: they love life.  In people's eyes, in the swing, tramp, and
trudge; in the bellow and the uproar; the carriages, motor cars,
omnibuses, vans, sandwich men shuffling and swinging; brass bands;
barrel organs; in the triumph and the jingle and the strange high
singing of some aeroplane overhead was what she loved; life;
London; this moment of June.

For it was the middle of June.  The War was over, except for some
one like Mrs. Foxcroft at the Embassy last night eating her heart
out because that nice boy was killed and now the old Manor House
must go to a cousin; or Lady Bexborough who opened a bazaar, they
said, with the telegram in her hand, John, her favourite, killed;
but it was over; thank Heaven--over.  It was June.  The King and
Queen were at the Palace.  And everywhere, though it was still so
early, there was a beating, a stirring of galloping ponies, tapping
of cricket bats; Lords, Ascot, Ranelagh and all the rest of it;
wrapped in the soft mesh of the grey-blue morning air, which, as
the day wore on, would unwind them, and set down on their lawns and
pitches the bouncing ponies, whose forefeet just struck the ground
and up they sprung, the whirling young men, and laughing girls in
their transparent muslins who, even now, after dancing all night,
were taking their absurd woolly dogs for a run; and even now, at
this hour, discreet old dowagers were shooting out in their motor
cars on errands of mystery; and the shopkeepers were fidgeting in
their windows with their paste and diamonds, their lovely old sea-
green brooches in eighteenth-century settings to tempt Americans
(but one must economise, not buy things rashly for Elizabeth), and
she, too, loving it as she did with an absurd and faithful passion,
being part of it, since her people were courtiers once in the time
of the Georges, she, too, was going that very night to kindle and
illuminate; to give her party.  But how strange, on entering the
Park, the silence; the mist; the hum; the slow-swimming happy
ducks; the pouched birds waddling; and who should be coming along
with his back against the Government buildings, most appropriately,
carrying a despatch box stamped with the Royal Arms, who but Hugh
Whitbread; her old friend Hugh--the admirable Hugh!

"Good-morning to you, Clarissa!" said Hugh, rather extravagantly,
for they had known each other as children.  "Where are you off to?"

"I love walking in London," said Mrs. Dalloway.  "Really it's
better than walking in the country."

They had just come up--unfortunately--to see doctors.  Other people
came to see pictures; go to the opera; take their daughters out;
the Whitbreads came "to see doctors."  Times without number
Clarissa had visited Evelyn Whitbread in a nursing home.  Was
Evelyn ill again?  Evelyn was a good deal out of sorts, said Hugh,
intimating by a kind of pout or swell of his very well-covered,
manly, extremely handsome, perfectly upholstered body (he was
almost too well dressed always, but presumably had to be, with his
little job at Court) that his wife had some internal ailment,
nothing serious, which, as an old friend, Clarissa Dalloway would
quite understand without requiring him to specify.  Ah yes, she did
of course; what a nuisance; and felt very sisterly and oddly
conscious at the same time of her hat.  Not the right hat for the
early morning, was that it?  For Hugh always made her feel, as he
bustled on, raising his hat rather extravagantly and assuring her
that she might be a girl of eighteen, and of course he was coming
to her party to-night, Evelyn absolutely insisted, only a little
late he might be after the party at the Palace to which he had to
take one of Jim's boys,--she always felt a little skimpy beside
Hugh; schoolgirlish; but attached to him, partly from having known
him always, but she did think him a good sort in his own way,
though Richard was nearly driven mad by him, and as for Peter
Walsh, he had never to this day forgiven her for liking him.

She could remember scene after scene at Bourton--Peter furious;
Hugh not, of course, his match in any way, but still not a positive
imbecile as Peter made out; not a mere barber's block.  When his
old mother wanted him to give up shooting or to take her to Bath he
did it, without a word; he was really unselfish, and as for saying,
as Peter did, that he had no heart, no brain, nothing but the
manners and breeding of an English gentleman, that was only her
dear Peter at his worst; and he could be intolerable; he could be
impossible; but adorable to walk with on a morning like this.

(June had drawn out every leaf on the trees.  The mothers of
Pimlico gave suck to their young.  Messages were passing from the
Fleet to the Admiralty.  Arlington Street and Piccadilly seemed to
chafe the very air in the Park and lift its leaves hotly,
brilliantly, on waves of that divine vitality which Clarissa loved.
To dance, to ride, she had adored all that.)

For they might be parted for hundreds of years, she and Peter; she
never wrote a letter and his were dry sticks; but suddenly it would
come over her, If he were with me now what would he say?--some
days, some sights bringing him back to her calmly, without the old
bitterness; which perhaps was the reward of having cared for
people; they came back in the middle of St. James's Park on a fine
morning--indeed they did.  But Peter--however beautiful the day
might be, and the trees and the grass, and the little girl in pink--
Peter never saw a thing of all that.  He would put on his
spectacles, if she told him to; he would look.  It was the state of
the world that interested him; Wagner, Pope's poetry, people's
characters eternally, and the defects of her own soul.  How he
scolded her!  How they argued!  She would marry a Prime Minister
and stand at the top of a staircase; the perfect hostess he called
her (she had cried over it in her bedroom), she had the makings of
the perfect hostess, he said.

So she would still find herself arguing in St. James's Park, still
making out that she had been right--and she had too--not to marry
him.  For in marriage a little licence, a little independence there
must be between people living together day in day out in the same
house; which Richard gave her, and she him.  (Where was he this
morning for instance?  Some committee, she never asked what.)  But
with Peter everything had to be shared; everything gone into.  And
it was intolerable, and when it came to that scene in the little
garden by the fountain, she had to break with him or they would
have been destroyed, both of them ruined, she was convinced; though
she had borne about with her for years like an arrow sticking in
her heart the grief, the anguish; and then the horror of the moment
when some one told her at a concert that he had married a woman met
on the boat going to India!  Never should she forget all that!
Cold, heartless, a prude, he called her.  Never could she
understand how he cared.  But those Indian women did presumably--
silly, pretty, flimsy nincompoops.  And she wasted her pity.  For
he was quite happy, he assured her--perfectly happy, though he had
never done a thing that they talked of; his whole life had been a
failure.  It made her angry still.

She had reached the Park gates.  She stood for a moment, looking at
the omnibuses in Piccadilly.

She would not say of any one in the world now that they were this
or were that.  She felt very young; at the same time unspeakably
aged.  She sliced like a knife through everything; at the same time
was outside, looking on.  She had a perpetual sense, as she watched
the taxi cabs, of being out, out, far out to sea and alone; she
always had the feeling that it was very, very dangerous to live
even one day.  Not that she thought herself clever, or much out of
the ordinary.  How she had got through life on the few twigs of
knowledge Fräulein Daniels gave them she could not think.  She knew
nothing; no language, no history; she scarcely read a book now,
except memoirs in bed; and yet to her it was absolutely absorbing;
all this; the cabs passing; and she would not say of Peter, she
would not say of herself, I am this, I am that.

Her only gift was knowing people almost by instinct, she thought,
walking on.  If you put her in a room with some one, up went her
back like a cat's; or she purred.  Devonshire House, Bath House,
the house with the china cockatoo, she had seen them all lit up
once; and remembered Sylvia, Fred, Sally Seton--such hosts of
people; and dancing all night; and the waggons plodding past to
market; and driving home across the Park.  She remembered once
throwing a shilling into the Serpentine.  But every one remembered;
what she loved was this, here, now, in front of her; the fat lady
in the cab.  Did it matter then, she asked herself, walking towards
Bond Street, did it matter that she must inevitably cease
completely; all this must go on without her; did she resent it; or
did it not become consoling to believe that death ended absolutely?
but that somehow in the streets of London, on the ebb and flow of
things, here, there, she survived, Peter survived, lived in each
other, she being part, she was positive, of the trees at home; of
the house there, ugly, rambling all to bits and pieces as it was;
part of people she had never met; being laid out like a mist
between the people she knew best, who lifted her on their branches
as she had seen the trees lift the mist, but it spread ever so far,
her life, herself.  But what was she dreaming as she looked into
Hatchards' shop window?  What was she trying to recover?  What
image of white dawn in the country, as she read in the book spread
open:


     Fear no more the heat o' the sun
     Nor the furious winter's rages.


This late age of the world's experience had bred in them all, all
men and women, a well of tears.  Tears and sorrows; courage and
endurance; a perfectly upright and stoical bearing.  Think, for
example, of the woman she admired most, Lady Bexborough, opening
the bazaar.

There were Jorrocks' Jaunts and Jollities; there were Soapy Sponge
and Mrs. Asquith's Memoirs and Big Game Shooting in Nigeria, all
spread open.  Ever so many books there were; but none that seemed
exactly right to take to Evelyn Whitbread in her nursing home.
Nothing that would serve to amuse her and make that indescribably
dried-up little woman look, as Clarissa came in, just for a moment
cordial; before they settled down for the usual interminable talk
of women's ailments.  How much she wanted it--that people should
look pleased as she came in, Clarissa thought and turned and walked
back towards Bond Street, annoyed, because it was silly to have
other reasons for doing things.  Much rather would she have been
one of those people like Richard who did things for themselves,
whereas, she thought, waiting to cross, half the time she did
things not simply, not for themselves; but to make people think
this or that; perfect idiocy she knew (and now the policeman held
up his hand) for no one was ever for a second taken in.  Oh if she
could have had her life over again! she thought, stepping on to the
pavement, could have looked even differently!

She would have been, in the first place, dark like Lady Bexborough,
with a skin of crumpled leather and beautiful eyes.  She would have
been, like Lady Bexborough, slow and stately; rather large;
interested in politics like a man; with a country house; very
dignified, very sincere.  Instead of which she had a narrow pea-
stick figure; a ridiculous little face, beaked like a bird's.  That
she held herself well was true; and had nice hands and feet; and
dressed well, considering that she spent little.  But often now
this body she wore (she stopped to look at a Dutch picture), this
body, with all its capacities, seemed nothing--nothing at all.  She
had the oddest sense of being herself invisible; unseen; unknown;
there being no more marrying, no more having of children now, but
only this astonishing and rather solemn progress with the rest of
them, up Bond Street, this being Mrs. Dalloway; not even Clarissa
any more; this being Mrs. Richard Dalloway.

Bond Street fascinated her; Bond Street early in the morning in the
season; its flags flying; its shops; no splash; no glitter; one
roll of tweed in the shop where her father had bought his suits for
fifty years; a few pearls; salmon on an iceblock.

"That is all," she said, looking at the fishmonger's.  "That is
all," she repeated, pausing for a moment at the window of a glove
shop where, before the War, you could buy almost perfect gloves.
And her old Uncle William used to say a lady is known by her shoes
and her gloves.  He had turned on his bed one morning in the middle
of the War.  He had said, "I have had enough."  Gloves and shoes;
she had a passion for gloves; but her own daughter, her Elizabeth,
cared not a straw for either of them.

Not a straw, she thought, going on up Bond Street to a shop where
they kept flowers for her when she gave a party.  Elizabeth really
cared for her dog most of all.  The whole house this morning smelt
of tar.  Still, better poor Grizzle than Miss Kilman; better
distemper and tar and all the rest of it than sitting mewed in a
stuffy bedroom with a prayer book!  Better anything, she was
inclined to say.  But it might be only a phase, as Richard said,
such as all girls go through.  It might be falling in love.  But
why with Miss Kilman? who had been badly treated of course; one
must make allowances for that, and Richard said she was very able,
had a really historical mind.  Anyhow they were inseparable, and
Elizabeth, her own daughter, went to Communion; and how she
dressed, how she treated people who came to lunch she did not care
a bit, it being her experience that the religious ecstasy made
people callous (so did causes); dulled their feelings, for Miss
Kilman would do anything for the Russians, starved herself for the
Austrians, but in private inflicted positive torture, so
insensitive was she, dressed in a green mackintosh coat.  Year in
year out she wore that coat; she perspired; she was never in the
room five minutes without making you feel her superiority, your
inferiority; how poor she was; how rich you were; how she lived in
a slum without a cushion or a bed or a rug or whatever it might be,
all her soul rusted with that grievance sticking in it, her
dismissal from school during the War--poor embittered unfortunate
creature!  For it was not her one hated but the idea of her, which
undoubtedly had gathered in to itself a great deal that was not
Miss Kilman; had become one of those spectres with which one
battles in the night; one of those spectres who stand astride us
and suck up half our life-blood, dominators and tyrants; for no
doubt with another throw of the dice, had the black been uppermost
and not the white, she would have loved Miss Kilman!  But not in
this world.  No.

It rasped her, though, to have stirring about in her this brutal
monster! to hear twigs cracking and feel hooves planted down in the
depths of that leaf-encumbered forest, the soul; never to be
content quite, or quite secure, for at any moment the brute would
be stirring, this hatred, which, especially since her illness, had
power to make her feel scraped, hurt in her spine; gave her
physical pain, and made all pleasure in beauty, in friendship, in
being well, in being loved and making her home delightful rock,
quiver, and bend as if indeed there were a monster grubbing at the
roots, as if the whole panoply of content were nothing but self
love! this hatred!

Nonsense, nonsense! she cried to herself, pushing through the swing
doors of Mulberry's the florists.

She advanced, light, tall, very upright, to be greeted at once by
button-faced Miss Pym, whose hands were always bright red, as if
they had been stood in cold water with the flowers.

There were flowers: delphiniums, sweet peas, bunches of lilac; and
carnations, masses of carnations.  There were roses; there were
irises.  Ah yes--so she breathed in the earthy garden sweet smell
as she stood talking to Miss Pym who owed her help, and thought her
kind, for kind she had been years ago; very kind, but she looked
older, this year, turning her head from side to side among the
irises and roses and nodding tufts of lilac with her eyes half
closed, snuffing in, after the street uproar, the delicious scent,
the exquisite coolness.  And then, opening her eyes, how fresh like
frilled linen clean from a laundry laid in wicker trays the roses
looked; and dark and prim the red carnations, holding their heads
up; and all the sweet peas spreading in their bowls, tinged violet,
snow white, pale--as if it were the evening and girls in muslin
frocks came out to pick sweet peas and roses after the superb
summer's day, with its almost blue-black sky, its delphiniums, its
carnations, its arum lilies was over; and it was the moment between
six and seven when every flower--roses, carnations, irises, lilac--
glows; white, violet, red, deep orange; every flower seems to burn
by itself, softly, purely in the misty beds; and how she loved the
grey-white moths spinning in and out, over the cherry pie, over the
evening primroses!

And as she began to go with Miss Pym from jar to jar, choosing,
nonsense, nonsense, she said to herself, more and more gently, as
if this beauty, this scent, this colour, and Miss Pym liking her,
trusting her, were a wave which she let flow over her and surmount
that hatred, that monster, surmount it all; and it lifted her up
and up when--oh! a pistol shot in the street outside!

"Dear, those motor cars," said Miss Pym, going to the window to
look, and coming back and smiling apologetically with her hands
full of sweet peas, as if those motor cars, those tyres of motor
cars, were all HER fault.



The violent explosion which made Mrs. Dalloway jump and Miss Pym go
to the window and apologise came from a motor car which had drawn
to the side of the pavement precisely opposite Mulberry's shop
window.  Passers-by who, of course, stopped and stared, had just
time to see a face of the very greatest importance against the
dove-grey upholstery, before a male hand drew the blind and there
was nothing to be seen except a square of dove grey.

Yet rumours were at once in circulation from the middle of Bond
Street to Oxford Street on one side, to Atkinson's scent shop on
the other, passing invisibly, inaudibly, like a cloud, swift, veil-
like upon hills, falling indeed with something of a cloud's sudden
sobriety and stillness upon faces which a second before had been
utterly disorderly.  But now mystery had brushed them with her
wing; they had heard the voice of authority; the spirit of religion
was abroad with her eyes bandaged tight and her lips gaping wide.
But nobody knew whose face had been seen.  Was it the Prince of
Wales's, the Queen's, the Prime Minister's?  Whose face was it?
Nobody knew.

Edgar J. Watkiss, with his roll of lead piping round his arm, said
audibly, humorously of course:  "The Proime Minister's kyar."

Septimus Warren Smith, who found himself unable to pass, heard him.

Septimus Warren Smith, aged about thirty, pale-faced, beak-nosed,
wearing brown shoes and a shabby overcoat, with hazel eyes which
had that look of apprehension in them which makes complete
strangers apprehensive too.  The world has raised its whip; where
will it descend?

Everything had come to a standstill.  The throb of the motor
engines sounded like a pulse irregularly drumming through an entire
body.  The sun became extraordinarily hot because the motor car had
stopped outside Mulberry's shop window; old ladies on the tops of
omnibuses spread their black parasols; here a green, here a red
parasol opened with a little pop.  Mrs. Dalloway, coming to the
window with her arms full of sweet peas, looked out with her little
pink face pursed in enquiry.  Every one looked at the motor car.
Septimus looked.  Boys on bicycles sprang off.  Traffic
accumulated.  And there the motor car stood, with drawn blinds, and
upon them a curious pattern like a tree, Septimus thought, and this
gradual drawing together of everything to one centre before his
eyes, as if some horror had come almost to the surface and was
about to burst into flames, terrified him.  The world wavered and
quivered and threatened to burst into flames.  It is I who am
blocking the way, he thought.  Was he not being looked at and
pointed at; was he not weighted there, rooted to the pavement, for
a purpose?  But for what purpose?

"Let us go on, Septimus," said his wife, a little woman, with large
eyes in a sallow pointed face; an Italian girl.

But Lucrezia herself could not help looking at the motor car and
the tree pattern on the blinds.  Was it the Queen in there--the
Queen going shopping?

The chauffeur, who had been opening something, turning something,
shutting something, got on to the box.

"Come on," said Lucrezia.

But her husband, for they had been married four, five years now,
jumped, started, and said, "All right!" angrily, as if she had
interrupted him.

People must notice; people must see.  People, she thought, looking
at the crowd staring at the motor car; the English people, with
their children and their horses and their clothes, which she
admired in a way; but they were "people" now, because Septimus had
said, "I will kill myself"; an awful thing to say.  Suppose they
had heard him?  She looked at the crowd.  Help, help! she wanted to
cry out to butchers' boys and women.  Help!  Only last autumn she
and Septimus had stood on the Embankment wrapped in the same cloak
and, Septimus reading a paper instead of talking, she had snatched
it from him and laughed in the old man's face who saw them!  But
failure one conceals.  She must take him away into some park.

"Now we will cross," she said.

She had a right to his arm, though it was without feeling.  He
would give her, who was so simple, so impulsive, only twenty-four,
without friends in England, who had left Italy for his sake, a
piece of bone.

The motor car with its blinds drawn and an air of inscrutable
reserve proceeded towards Piccadilly, still gazed at, still
ruffling the faces on both sides of the street with the same dark
breath of veneration whether for Queen, Prince, or Prime Minister
nobody knew.  The face itself had been seen only once by three
people for a few seconds.  Even the sex was now in dispute.  But
there could be no doubt that greatness was seated within; greatness
was passing, hidden, down Bond Street, removed only by a hand's-
breadth from ordinary people who might now, for the first and last
time, be within speaking distance of the majesty of England, of the
enduring symbol of the state which will be known to curious
antiquaries, sifting the ruins of time, when London is a grass-
grown path and all those hurrying along the pavement this Wednesday
morning are but bones with a few wedding rings mixed up in their
dust and the gold stoppings of innumerable decayed teeth.  The face
in the motor car will then be known.

It is probably the Queen, thought Mrs. Dalloway, coming out of
Mulberry's with her flowers; the Queen.  And for a second she wore
a look of extreme dignity standing by the flower shop in the
sunlight while the car passed at a foot's pace, with its blinds
drawn.  The Queen going to some hospital; the Queen opening some
bazaar, thought Clarissa.

The crush was terrific for the time of day.  Lords, Ascot,
Hurlingham, what was it? she wondered, for the street was blocked.
The British middle classes sitting sideways on the tops of
omnibuses with parcels and umbrellas, yes, even furs on a day like
this, were, she thought, more ridiculous, more unlike anything
there has ever been than one could conceive; and the Queen herself
held up; the Queen herself unable to pass.  Clarissa was suspended
on one side of Brook Street; Sir John Buckhurst, the old Judge on
the other, with the car between them (Sir John had laid down the
law for years and liked a well-dressed woman) when the chauffeur,
leaning ever so slightly, said or showed something to the
policeman, who saluted and raised his arm and jerked his head and
moved the omnibus to the side and the car passed through.  Slowly
and very silently it took its way.

Clarissa guessed; Clarissa knew of course; she had seen something
white, magical, circular, in the footman's hand, a disc inscribed
with a name,--the Queen's, the Prince of Wales's, the Prime
Minister's?--which, by force of its own lustre, burnt its way
through (Clarissa saw the car diminishing, disappearing), to blaze
among candelabras, glittering stars, breasts stiff with oak leaves,
Hugh Whitbread and all his colleagues, the gentlemen of England,
that night in Buckingham Palace.  And Clarissa, too, gave a party.
She stiffened a little; so she would stand at the top of her
stairs.

The car had gone, but it had left a slight ripple which flowed
through glove shops and hat shops and tailors' shops on both sides
of Bond Street.  For thirty seconds all heads were inclined the
same way--to the window.  Choosing a pair of gloves--should they be
to the elbow or above it, lemon or pale grey?--ladies stopped; when
the sentence was finished something had happened.  Something so
trifling in single instances that no mathematical instrument,
though capable of transmitting shocks in China, could register the
vibration; yet in its fulness rather formidable and in its common
appeal emotional; for in all the hat shops and tailors' shops
strangers looked at each other and thought of the dead; of the
flag; of Empire.  In a public house in a back street a Colonial
insulted the House of Windsor which led to words, broken beer
glasses, and a general shindy, which echoed strangely across the
way in the ears of girls buying white underlinen threaded with pure
white ribbon for their weddings.  For the surface agitation of the
passing car as it sunk grazed something very profound.

Gliding across Piccadilly, the car turned down St. James's Street.
Tall men, men of robust physique, well-dressed men with their tail-
coats and their white slips and their hair raked back who, for
reasons difficult to discriminate, were standing in the bow window
of Brooks's with their hands behind the tails of their coats,
looking out, perceived instinctively that greatness was passing,
and the pale light of the immortal presence fell upon them as it
had fallen upon Clarissa Dalloway.  At once they stood even
straighter, and removed their hands, and seemed ready to attend
their Sovereign, if need be, to the cannon's mouth, as their
ancestors had done before them.  The white busts and the little
tables in the background covered with copies of the Tatler and
syphons of soda water seemed to approve; seemed to indicate the
flowing corn and the manor houses of England; and to return the
frail hum of the motor wheels as the walls of a whispering gallery
return a single voice expanded and made sonorous by the might of a
whole cathedral.  Shawled Moll Pratt with her flowers on the
pavement wished the dear boy well (it was the Prince of Wales for
certain) and would have tossed the price of a pot of beer--a bunch
of roses--into St. James's Street out of sheer light-heartedness
and contempt of poverty had she not seen the constable's eye upon
her, discouraging an old Irishwoman's loyalty.  The sentries at St.
James's saluted; Queen Alexandra's policeman approved.

A small crowd meanwhile had gathered at the gates of Buckingham
Palace.  Listlessly, yet confidently, poor people all of them, they
waited; looked at the Palace itself with the flag flying; at
Victoria, billowing on her mound, admired her shelves of running
water, her geraniums; singled out from the motor cars in the Mall
first this one, then that; bestowed emotion, vainly, upon commoners
out for a drive; recalled their tribute to keep it unspent while
this car passed and that; and all the time let rumour accumulate in
their veins and thrill the nerves in their thighs at the thought of
Royalty looking at them; the Queen bowing; the Prince saluting; at
the thought of the heavenly life divinely bestowed upon Kings; of
the equerries and deep curtsies; of the Queen's old doll's house;
of Princess Mary married to an Englishman, and the Prince--ah! the
Prince! who took wonderfully, they said, after old King Edward, but
was ever so much slimmer.  The Prince lived at St. James's; but he
might come along in the morning to visit his mother.

So Sarah Bletchley said with her baby in her arms, tipping her foot
up and down as though she were by her own fender in Pimlico, but
keeping her eyes on the Mall, while Emily Coates ranged over the
Palace windows and thought of the housemaids, the innumerable
housemaids, the bedrooms, the innumerable bedrooms.  Joined by an
elderly gentleman with an Aberdeen terrier, by men without
occupation, the crowd increased.  Little Mr. Bowley, who had rooms
in the Albany and was sealed with wax over the deeper sources of
life but could be unsealed suddenly, inappropriately, sentimentally,
by this sort of thing--poor women waiting to see the Queen go past--
poor women, nice little children, orphans, widows, the War--tut-
tut--actually had tears in his eyes.  A breeze flaunting ever so
warmly down the Mall through the thin trees, past the bronze heroes,
lifted some flag flying in the British breast of Mr. Bowley and he
raised his hat as the car turned into the Mall and held it high as
the car approached; and let the poor mothers of Pimlico press close
to him, and stood very upright.  The car came on.

Suddenly Mrs. Coates looked up into the sky.  The sound of an
aeroplane bored ominously into the ears of the crowd.  There it was
coming over the trees, letting out white smoke from behind, which
curled and twisted, actually writing something! making letters in
the sky!  Every one looked up.

Dropping dead down the aeroplane soared straight up, curved in a
loop, raced, sank, rose, and whatever it did, wherever it went, out
fluttered behind it a thick ruffled bar of white smoke which curled
and wreathed upon the sky in letters.  But what letters?  A C was
it? an E, then an L?  Only for a moment did they lie still; then
they moved and melted and were rubbed out up in the sky, and the
aeroplane shot further away and again, in a fresh space of sky,
began writing a K, an E, a Y perhaps?

"Glaxo," said Mrs. Coates in a strained, awe-stricken voice, gazing
straight up, and her baby, lying stiff and white in her arms, gazed
straight up.

"Kreemo," murmured Mrs. Bletchley, like a sleep-walker.  With his
hat held out perfectly still in his hand, Mr. Bowley gazed straight
up.  All down the Mall people were standing and looking up into the
sky.  As they looked the whole world became perfectly silent, and a
flight of gulls crossed the sky, first one gull leading, then
another, and in this extraordinary silence and peace, in this
pallor, in this purity, bells struck eleven times, the sound fading
up there among the gulls.

The aeroplane turned and raced and swooped exactly where it liked,
swiftly, freely, like a skater--

"That's an E," said Mrs. Bletchley--or a dancer--

"It's toffee," murmured Mr. Bowley--(and the car went in at the
gates and nobody looked at it), and shutting off the smoke, away
and away it rushed, and the smoke faded and assembled itself round
the broad white shapes of the clouds.

It had gone; it was behind the clouds.  There was no sound.  The
clouds to which the letters E, G, or L had attached themselves
moved freely, as if destined to cross from West to East on a
mission of the greatest importance which would never be revealed,
and yet certainly so it was--a mission of the greatest importance.
Then suddenly, as a train comes out of a tunnel, the aeroplane
rushed out of the clouds again, the sound boring into the ears of
all people in the Mall, in the Green Park, in Piccadilly, in Regent
Street, in Regent's Park, and the bar of smoke curved behind and it
dropped down, and it soared up and wrote one letter after another--
but what word was it writing?

Lucrezia Warren Smith, sitting by her husband's side on a seat in
Regent's Park in the Broad Walk, looked up.

"Look, look, Septimus!" she cried.  For Dr. Holmes had told her to
make her husband (who had nothing whatever seriously the matter
with him but was a little out of sorts) take an interest in things
outside himself.

So, thought Septimus, looking up, they are signalling to me.  Not
indeed in actual words; that is, he could not read the language
yet; but it was plain enough, this beauty, this exquisite beauty,
and tears filled his eyes as he looked at the smoke words
languishing and melting in the sky and bestowing upon him in their
inexhaustible charity and laughing goodness one shape after another
of unimaginable beauty and signalling their intention to provide
him, for nothing, for ever, for looking merely, with beauty, more
beauty!  Tears ran down his cheeks.

It was toffee; they were advertising toffee, a nursemaid told
Rezia.  Together they began to spell t . . . o . . . f . . .

"K . . . R . . ." said the nursemaid, and Septimus heard her say
"Kay Arr" close to his ear, deeply, softly, like a mellow organ,
but with a roughness in her voice like a grasshopper's, which
rasped his spine deliciously and sent running up into his brain
waves of sound which, concussing, broke.  A marvellous discovery
indeed--that the human voice in certain atmospheric conditions (for
one must be scientific, above all scientific) can quicken trees
into life!  Happily Rezia put her hand with a tremendous weight on
his knee so that he was weighted down, transfixed, or the
excitement of the elm trees rising and falling, rising and falling
with all their leaves alight and the colour thinning and thickening
from blue to the green of a hollow wave, like plumes on horses'
heads, feathers on ladies', so proudly they rose and fell, so
superbly, would have sent him mad.  But he would not go mad.  He
would shut his eyes; he would see no more.

But they beckoned; leaves were alive; trees were alive.  And the
leaves being connected by millions of fibres with his own body,
there on the seat, fanned it up and down; when the branch stretched
he, too, made that statement.  The sparrows fluttering, rising, and
falling in jagged fountains were part of the pattern; the white and
blue, barred with black branches.  Sounds made harmonies with
premeditation; the spaces between them were as significant as the
sounds.  A child cried.  Rightly far away a horn sounded.  All
taken together meant the birth of a new religion--

"Septimus!" said Rezia.  He started violently.  People must notice.

"I am going to walk to the fountain and back," she said.

For she could stand it no longer.  Dr. Holmes might say there was
nothing the matter.  Far rather would she that he were dead!  She
could not sit beside him when he stared so and did not see her and
made everything terrible; sky and tree, children playing, dragging
carts, blowing whistles, falling down; all were terrible.  And he
would not kill himself; and she could tell no one.  "Septimus has
been working too hard"--that was all she could say to her own
mother.  To love makes one solitary, she thought.  She could tell
nobody, not even Septimus now, and looking back, she saw him
sitting in his shabby overcoat alone, on the seat, hunched up,
staring.  And it was cowardly for a man to say he would kill
himself, but Septimus had fought; he was brave; he was not Septimus
now.  She put on her lace collar.  She put on her new hat and he
never noticed; and he was happy without her.  Nothing could make
her happy without him!  Nothing!  He was selfish.  So men are.  For
he was not ill.  Dr. Holmes said there was nothing the matter with
him.  She spread her hand before her.  Look!  Her wedding ring
slipped--she had grown so thin.  It was she who suffered--but she
had nobody to tell.

Far was Italy and the white houses and the room where her sisters
sat making hats, and the streets crowded every evening with people
walking, laughing out loud, not half alive like people here,
huddled up in Bath chairs, looking at a few ugly flowers stuck in
pots!

"For you should see the Milan gardens," she said aloud.  But to
whom?

There was nobody.  Her words faded.  So a rocket fades.  Its
sparks, having grazed their way into the night, surrender to it,
dark descends, pours over the outlines of houses and towers; bleak
hillsides soften and fall in.  But though they are gone, the night
is full of them; robbed of colour, blank of windows, they exist
more ponderously, give out what the frank daylight fails to
transmit--the trouble and suspense of things conglomerated there in
the darkness; huddled together in the darkness; reft of the relief
which dawn brings when, washing the walls white and grey, spotting
each window-pane, lifting the mist from the fields, showing the
red-brown cows peacefully grazing, all is once more decked out to
the eye; exists again.  I am alone; I am alone! she cried, by the
fountain in Regent's Park (staring at the Indian and his cross), as
perhaps at midnight, when all boundaries are lost, the country
reverts to its ancient shape, as the Romans saw it, lying cloudy,
when they landed, and the hills had no names and rivers wound they
knew not where--such was her darkness; when suddenly, as if a shelf
were shot forth and she stood on it, she said how she was his wife,
married years ago in Milan, his wife, and would never, never tell
that he was mad!  Turning, the shelf fell; down, down she dropped.
For he was gone, she thought--gone, as he threatened, to kill
himself--to throw himself under a cart!  But no; there he was;
still sitting alone on the seat, in his shabby overcoat, his legs
crossed, staring, talking aloud.

Men must not cut down trees.  There is a God.  (He noted such
revelations on the backs of envelopes.)  Change the world.  No one
kills from hatred.  Make it known (he wrote it down).  He waited.
He listened.  A sparrow perched on the railing opposite chirped
Septimus, Septimus, four or five times over and went on, drawing
its notes out, to sing freshly and piercingly in Greek words how
there is no crime and, joined by another sparrow, they sang in
voices prolonged and piercing in Greek words, from trees in the
meadow of life beyond a river where the dead walk, how there is no
death.

There was his hand; there the dead.  White things were assembling
behind the railings opposite.  But he dared not look.  Evans was
behind the railings!

"What are you saying?" said Rezia suddenly, sitting down by him.

Interrupted again!  She was always interrupting.

Away from people--they must get away from people, he said (jumping
up), right away over there, where there were chairs beneath a tree
and the long slope of the park dipped like a length of green stuff
with a ceiling cloth of blue and pink smoke high above, and there
was a rampart of far irregular houses hazed in smoke, the traffic
hummed in a circle, and on the right, dun-coloured animals
stretched long necks over the Zoo palings, barking, howling.  There
they sat down under a tree.

"Look," she implored him, pointing at a little troop of boys
carrying cricket stumps, and one shuffled, spun round on his heel
and shuffled, as if he were acting a clown at the music hall.

"Look," she implored him, for Dr. Holmes had told her to make him
notice real things, go to a music hall, play cricket--that was the
very game, Dr. Holmes said, a nice out-of-door game, the very game
for her husband.

"Look," she repeated.

Look the unseen bade him, the voice which now communicated with him
who was the greatest of mankind, Septimus, lately taken from life
to death, the Lord who had come to renew society, who lay like a
coverlet, a snow blanket smitten only by the sun, for ever
unwasted, suffering for ever, the scapegoat, the eternal sufferer,
but he did not want it, he moaned, putting from him with a wave of
his hand that eternal suffering, that eternal loneliness.

"Look," she repeated, for he must not talk aloud to himself out of
doors.

"Oh look," she implored him.  But what was there to look at?  A few
sheep.  That was all.

The way to Regent's Park Tube station--could they tell her the way
to Regent's Park Tube station--Maisie Johnson wanted to know.  She
was only up from Edinburgh two days ago.

"Not this way--over there!" Rezia exclaimed, waving her aside, lest
she should see Septimus.

Both seemed queer, Maisie Johnson thought.  Everything seemed very
queer.  In London for the first time, come to take up a post at her
uncle's in Leadenhall Street, and now walking through Regent's Park
in the morning, this couple on the chairs gave her quite a turn;
the young woman seeming foreign, the man looking queer; so that
should she be very old she would still remember and make it jangle
again among her memories how she had walked through Regent's Park
on a fine summer's morning fifty years ago.  For she was only
nineteen and had got her way at last, to come to London; and now
how queer it was, this couple she had asked the way of, and the
girl started and jerked her hand, and the man--he seemed awfully
odd; quarrelling, perhaps; parting for ever, perhaps; something was
up, she knew; and now all these people (for she returned to the
Broad Walk), the stone basins, the prim flowers, the old men and
women, invalids most of them in Bath chairs--all seemed, after
Edinburgh, so queer.  And Maisie Johnson, as she joined that gently
trudging, vaguely gazing, breeze-kissed company--squirrels perching
and preening, sparrow fountains fluttering for crumbs, dogs busy
with the railings, busy with each other, while the soft warm air
washed over them and lent to the fixed unsurprised gaze with which
they received life something whimsical and mollified--Maisie
Johnson positively felt she must cry Oh! (for that young man on the
seat had given her quite a turn.  Something was up, she knew.)

Horror!  horror!  she wanted to cry.  (She had left her people;
they had warned her what would happen.)

Why hadn't she stayed at home? she cried, twisting the knob of the
iron railing.

That girl, thought Mrs. Dempster (who saved crusts for the
squirrels and often ate her lunch in Regent's Park), don't know a
thing yet; and really it seemed to her better to be a little stout,
a little slack, a little moderate in one's expectations.  Percy
drank.  Well, better to have a son, thought Mrs. Dempster.  She had
had a hard time of it, and couldn't help smiling at a girl like
that.  You'll get married, for you're pretty enough, thought Mrs.
Dempster.  Get married, she thought, and then you'll know.  Oh, the
cooks, and so on.  Every man has his ways.  But whether I'd have
chosen quite like that if I could have known, thought Mrs.
Dempster, and could not help wishing to whisper a word to Maisie
Johnson; to feel on the creased pouch of her worn old face the kiss
of pity.  For it's been a hard life, thought Mrs. Dempster.  What
hadn't she given to it?  Roses; figure; her feet too.  (She drew
the knobbed lumps beneath her skirt.)

Roses, she thought sardonically.  All trash, m'dear.  For really,
what with eating, drinking, and mating, the bad days and good, life
had been no mere matter of roses, and what was more, let me tell
you, Carrie Dempster had no wish to change her lot with any woman's
in Kentish Town!  But, she implored, pity.  Pity, for the loss of
roses.  Pity she asked of Maisie Johnson, standing by the hyacinth
beds.

Ah, but that aeroplane!  Hadn't Mrs. Dempster always longed to see
foreign parts?  She had a nephew, a missionary.  It soared and
shot.  She always went on the sea at Margate, not out o' sight of
land, but she had no patience with women who were afraid of water.
It swept and fell.  Her stomach was in her mouth.  Up again.
There's a fine young feller aboard of it, Mrs. Dempster wagered,
and away and away it went, fast and fading, away and away the
aeroplane shot; soaring over Greenwich and all the masts; over the
little island of grey churches, St. Paul's and the rest till, on
either side of London, fields spread out and dark brown woods where
adventurous thrushes hopping boldly, glancing quickly, snatched the
snail and tapped him on a stone, once, twice, thrice.

Away and away the aeroplane shot, till it was nothing but a bright
spark; an aspiration; a concentration; a symbol (so it seemed to
Mr. Bentley, vigorously rolling his strip of turf at Greenwich) of
man's soul; of his determination, thought Mr. Bentley, sweeping
round the cedar tree, to get outside his body, beyond his house, by
means of thought, Einstein, speculation, mathematics, the Mendelian
theory--away the aeroplane shot.

Then, while a seedy-looking nondescript man carrying a leather bag
stood on the steps of St. Paul's Cathedral, and hesitated, for
within was what balm, how great a welcome, how many tombs with
banners waving over them, tokens of victories not over armies, but
over, he thought, that plaguy spirit of truth seeking which leaves
me at present without a situation, and more than that, the
cathedral offers company, he thought, invites you to membership of
a society; great men belong to it; martyrs have died for it; why
not enter in, he thought, put this leather bag stuffed with
pamphlets before an altar, a cross, the symbol of something which
has soared beyond seeking and questing and knocking of words
together and has become all spirit, disembodied, ghostly--why not
enter in? he thought and while he hesitated out flew the aeroplane
over Ludgate Circus.

It was strange; it was still.  Not a sound was to be heard above
the traffic.  Unguided it seemed; sped of its own free will.  And
now, curving up and up, straight up, like something mounting in
ecstasy, in pure delight, out from behind poured white smoke
looping, writing a T, an O, an F.



"What are they looking at?" said Clarissa Dalloway to the maid who
opened her door.

The hall of the house was cool as a vault.  Mrs. Dalloway raised
her hand to her eyes, and, as the maid shut the door to, and she
heard the swish of Lucy's skirts, she felt like a nun who has left
the world and feels fold round her the familiar veils and the
response to old devotions.  The cook whistled in the kitchen.  She
heard the click of the typewriter.  It was her life, and, bending
her head over the hall table, she bowed beneath the influence, felt
blessed and purified, saying to herself, as she took the pad with
the telephone message on it, how moments like this are buds on the
tree of life, flowers of darkness they are, she thought (as if some
lovely rose had blossomed for her eyes only); not for a moment did
she believe in God; but all the more, she thought, taking up the
pad, must one repay in daily life to servants, yes, to dogs and
canaries, above all to Richard her husband, who was the foundation
of it--of the gay sounds, of the green lights, of the cook even
whistling, for Mrs. Walker was Irish and whistled all day long--one
must pay back from this secret deposit of exquisite moments, she
thought, lifting the pad, while Lucy stood by her, trying to
explain how

"Mr. Dalloway, ma'am"--

Clarissa read on the telephone pad, "Lady Bruton wishes to know if
Mr. Dalloway will lunch with her to-day."

"Mr. Dalloway, ma'am, told me to tell you he would be lunching
out."

"Dear!" said Clarissa, and Lucy shared as she meant her to her
disappointment (but not the pang); felt the concord between them;
took the hint; thought how the gentry love; gilded her own future
with calm; and, taking Mrs. Dalloway's parasol, handled it like a
sacred weapon which a Goddess, having acquitted herself honourably
in the field of battle, sheds, and placed it in the umbrella stand.

"Fear no more," said Clarissa.  Fear no more the heat o' the sun;
for the shock of Lady Bruton asking Richard to lunch without her
made the moment in which she had stood shiver, as a plant on the
river-bed feels the shock of a passing oar and shivers: so she
rocked: so she shivered.

Millicent Bruton, whose lunch parties were said to be extraordinarily
amusing, had not asked her.  No vulgar jealousy could separate
her from Richard.  But she feared time itself, and read on Lady
Bruton's face, as if it had been a dial cut in impassive stone, the
dwindling of life; how year by year her share was sliced; how little
the margin that remained was capable any longer of stretching, of
absorbing, as in the youthful years, the colours, salts, tones of
existence, so that she filled the room she entered, and felt often
as she stood hesitating one moment on the threshold of her drawing-
room, an exquisite suspense, such as might stay a diver before
plunging while the sea darkens and brightens beneath him, and the
waves which threaten to break, but only gently split their surface,
roll and conceal and encrust as they just turn over the weeds with
pearl.

She put the pad on the hall table.  She began to go slowly
upstairs, with her hand on the bannisters, as if she had left a
party, where now this friend now that had flashed back her face,
her voice; had shut the door and gone out and stood alone, a single
figure against the appalling night, or rather, to be accurate,
against the stare of this matter-of-fact June morning; soft with
the glow of rose petals for some, she knew, and felt it, as she
paused by the open staircase window which let in blinds flapping,
dogs barking, let in, she thought, feeling herself suddenly
shrivelled, aged, breastless, the grinding, blowing, flowering of
the day, out of doors, out of the window, out of her body and brain
which now failed, since Lady Bruton, whose lunch parties were said
to be extraordinarily amusing, had not asked her.

Like a nun withdrawing, or a child exploring a tower, she went
upstairs, paused at the window, came to the bathroom.  There was
the green linoleum and a tap dripping.  There was an emptiness
about the heart of life; an attic room.  Women must put off their
rich apparel.  At midday they must disrobe.  She pierced the
pincushion and laid her feathered yellow hat on the bed.  The
sheets were clean, tight stretched in a broad white band from side
to side.  Narrower and narrower would her bed be.  The candle was
half burnt down and she had read deep in Baron Marbot's Memoirs.
She had read late at night of the retreat from Moscow.  For the
House sat so long that Richard insisted, after her illness, that
she must sleep undisturbed.  And really she preferred to read of
the retreat from Moscow.  He knew it.  So the room was an attic;
the bed narrow; and lying there reading, for she slept badly, she
could not dispel a virginity preserved through childbirth which
clung to her like a sheet.  Lovely in girlhood, suddenly there came
a moment--for example on the river beneath the woods at Clieveden--
when, through some contraction of this cold spirit, she had failed
him.  And then at Constantinople, and again and again.  She could
see what she lacked.  It was not beauty; it was not mind.  It was
something central which permeated; something warm which broke up
surfaces and rippled the cold contact of man and woman, or of women
together.  For THAT she could dimly perceive.  She resented it, had
a scruple picked up Heaven knows where, or, as she felt, sent by
Nature (who is invariably wise); yet she could not resist sometimes
yielding to the charm of a woman, not a girl, of a woman
confessing, as to her they often did, some scrape, some folly.  And
whether it was pity, or their beauty, or that she was older, or
some accident--like a faint scent, or a violin next door (so
strange is the power of sounds at certain moments), she did
undoubtedly then feel what men felt.  Only for a moment; but it was
enough.  It was a sudden revelation, a tinge like a blush which one
tried to check and then, as it spread, one yielded to its
expansion, and rushed to the farthest verge and there quivered and
felt the world come closer, swollen with some astonishing
significance, some pressure of rapture, which split its thin skin
and gushed and poured with an extraordinary alleviation over the
cracks and sores!  Then, for that moment, she had seen an
illumination; a match burning in a crocus; an inner meaning almost
expressed.  But the close withdrew; the hard softened.  It was
over--the moment.  Against such moments (with women too) there
contrasted (as she laid her hat down) the bed and Baron Marbot and
the candle half-burnt.  Lying awake, the floor creaked; the lit
house was suddenly darkened, and if she raised her head she could
just hear the click of the handle released as gently as possible by
Richard, who slipped upstairs in his socks and then, as often as
not, dropped his hot-water bottle and swore!  How she laughed!

But this question of love (she thought, putting her coat away),
this falling in love with women.  Take Sally Seton; her relation in
the old days with Sally Seton.  Had not that, after all, been love?

She sat on the floor--that was her first impression of Sally--she
sat on the floor with her arms round her knees, smoking a
cigarette.  Where could it have been?  The Mannings?  The Kinloch-
Jones's?  At some party (where, she could not be certain), for she
had a distinct recollection of saying to the man she was with, "Who
is THAT?"  And he had told her, and said that Sally's parents did
not get on (how that shocked her--that one's parents should
quarrel!).  But all that evening she could not take her eyes off
Sally.  It was an extraordinary beauty of the kind she most
admired, dark, large-eyed, with that quality which, since she
hadn't got it herself, she always envied--a sort of abandonment, as
if she could say anything, do anything; a quality much commoner in
foreigners than in Englishwomen.  Sally always said she had French
blood in her veins, an ancestor had been with Marie Antoinette, had
his head cut off, left a ruby ring.  Perhaps that summer she came
to stay at Bourton, walking in quite unexpectedly without a penny
in her pocket, one night after dinner, and upsetting poor Aunt
Helena to such an extent that she never forgave her.  There had
been some quarrel at home.  She literally hadn't a penny that night
when she came to them--had pawned a brooch to come down.  She had
rushed off in a passion.  They sat up till all hours of the night
talking.  Sally it was who made her feel, for the first time, how
sheltered the life at Bourton was.  She knew nothing about sex--
nothing about social problems.  She had once seen an old man who
had dropped dead in a field--she had seen cows just after their
calves were born.  But Aunt Helena never liked discussion of
anything (when Sally gave her William Morris, it had to be wrapped
in brown paper).  There they sat, hour after hour, talking in her
bedroom at the top of the house, talking about life, how they were
to reform the world.  They meant to found a society to abolish
private property, and actually had a letter written, though not
sent out.  The ideas were Sally's, of course--but very soon she was
just as excited--read Plato in bed before breakfast; read Morris;
read Shelley by the hour.

Sally's power was amazing, her gift, her personality.  There was
her way with flowers, for instance.  At Bourton they always had
stiff little vases all the way down the table.  Sally went out,
picked hollyhocks, dahlias--all sorts of flowers that had never
been seen together--cut their heads off, and made them swim on the
top of water in bowls.  The effect was extraordinary--coming in to
dinner in the sunset.  (Of course Aunt Helena thought it wicked to
treat flowers like that.)  Then she forgot her sponge, and ran
along the passage naked.  That grim old housemaid, Ellen Atkins,
went about grumbling--"Suppose any of the gentlemen had seen?"
Indeed she did shock people.  She was untidy, Papa said.

The strange thing, on looking back, was the purity, the integrity,
of her feeling for Sally.  It was not like one's feeling for a man.
It was completely disinterested, and besides, it had a quality
which could only exist between women, between women just grown up.
It was protective, on her side; sprang from a sense of being in
league together, a presentiment of something that was bound to part
them (they spoke of marriage always as a catastrophe), which led to
this chivalry, this protective feeling which was much more on her
side than Sally's.  For in those days she was completely reckless;
did the most idiotic things out of bravado; bicycled round the
parapet on the terrace; smoked cigars.  Absurd, she was--very
absurd.  But the charm was overpowering, to her at least, so that
she could remember standing in her bedroom at the top of the house
holding the hot-water can in her hands and saying aloud, "She is
beneath this roof. . . .  She is beneath this roof!"

No, the words meant absolutely nothing to her now.  She could not
even get an echo of her old emotion.  But she could remember going
cold with excitement, and doing her hair in a kind of ecstasy (now
the old feeling began to come back to her, as she took out her
hairpins, laid them on the dressing-table, began to do her hair),
with the rooks flaunting up and down in the pink evening light, and
dressing, and going downstairs, and feeling as she crossed the hall
"if it were now to die 'twere now to be most happy."  That was her
feeling--Othello's feeling, and she felt it, she was convinced, as
strongly as Shakespeare meant Othello to feel it, all because she
was coming down to dinner in a white frock to meet Sally Seton!

She was wearing pink gauze--was that possible?  She SEEMED, anyhow,
all light, glowing, like some bird or air ball that has flown in,
attached itself for a moment to a bramble.  But nothing is so
strange when one is in love (and what was this except being in
love?) as the complete indifference of other people.  Aunt Helena
just wandered off after dinner; Papa read the paper.  Peter Walsh
might have been there, and old Miss Cummings; Joseph Breitkopf
certainly was, for he came every summer, poor old man, for weeks
and weeks, and pretended to read German with her, but really played
the piano and sang Brahms without any voice.

All this was only a background for Sally.  She stood by the
fireplace talking, in that beautiful voice which made everything
she said sound like a caress, to Papa, who had begun to be
attracted rather against his will (he never got over lending her
one of his books and finding it soaked on the terrace), when
suddenly she said, "What a shame to sit indoors!" and they all went
out on to the terrace and walked up and down.  Peter Walsh and
Joseph Breitkopf went on about Wagner.  She and Sally fell a little
behind.  Then came the most exquisite moment of her whole life
passing a stone urn with flowers in it.  Sally stopped; picked a
flower; kissed her on the lips.  The whole world might have turned
upside down!  The others disappeared; there she was alone with
Sally.  And she felt that she had been given a present, wrapped up,
and told just to keep it, not to look at it--a diamond, something
infinitely precious, wrapped up, which, as they walked (up and
down, up and down), she uncovered, or the radiance burnt through,
the revelation, the religious feeling!--when old Joseph and Peter
faced them:

"Star-gazing?" said Peter.

It was like running one's face against a granite wall in the
darkness!  It was shocking; it was horrible!

Not for herself.  She felt only how Sally was being mauled already,
maltreated; she felt his hostility; his jealousy; his determination
to break into their companionship.  All this she saw as one sees a
landscape in a flash of lightning--and Sally (never had she admired
her so much!) gallantly taking her way unvanquished.  She laughed.
She made old Joseph tell her the names of the stars, which he liked
doing very seriously.  She stood there: she listened.  She heard
the names of the stars.

"Oh this horror!" she said to herself, as if she had known all
along that something would interrupt, would embitter her moment of
happiness.

Yet, after all, how much she owed to him later.  Always when she
thought of him she thought of their quarrels for some reason--
because she wanted his good opinion so much, perhaps.  She owed him
words: "sentimental," "civilised"; they started up every day of her
life as if he guarded her.  A book was sentimental; an attitude to
life sentimental.  "Sentimental," perhaps she was to be thinking of
the past.  What would he think, she wondered, when he came back?

That she had grown older?  Would he say that, or would she see him
thinking when he came back, that she had grown older?  It was true.
Since her illness she had turned almost white.

Laying her brooch on the table, she had a sudden spasm, as if,
while she mused, the icy claws had had the chance to fix in her.
She was not old yet.  She had just broken into her fifty-second
year.  Months and months of it were still untouched.  June, July,
August!  Each still remained almost whole, and, as if to catch the
falling drop, Clarissa (crossing to the dressing-table) plunged
into the very heart of the moment, transfixed it, there--the moment
of this June morning on which was the pressure of all the other
mornings, seeing the glass, the dressing-table, and all the bottles
afresh, collecting the whole of her at one point (as she looked
into the glass), seeing the delicate pink face of the woman who was
that very night to give a party; of Clarissa Dalloway; of herself.

How many million times she had seen her face, and always with the
same imperceptible contraction!  She pursed her lips when she
looked in the glass.  It was to give her face point.  That was her
self--pointed; dartlike; definite.  That was her self when some
effort, some call on her to be her self, drew the parts together,
she alone knew how different, how incompatible and composed so for
the world only into one centre, one diamond, one woman who sat in
her drawing-room and made a meeting-point, a radiancy no doubt in
some dull lives, a refuge for the lonely to come to, perhaps; she
had helped young people, who were grateful to her; had tried to be
the same always, never showing a sign of all the other sides of
her--faults, jealousies, vanities, suspicions, like this of Lady
Bruton not asking her to lunch; which, she thought (combing her
hair finally), is utterly base!  Now, where was her dress?

Her evening dresses hung in the cupboard.  Clarissa, plunging her
hand into the softness, gently detached the green dress and carried
it to the window.  She had torn it.  Some one had trod on the
skirt.  She had felt it give at the Embassy party at the top among
the folds.  By artificial light the green shone, but lost its
colour now in the sun.  She would mend it.  Her maids had too much
to do.  She would wear it to-night.  She would take her silks, her
scissors, her--what was it?--her thimble, of course, down into the
drawing-room, for she must also write, and see that things
generally were more or less in order.

Strange, she thought, pausing on the landing, and assembling that
diamond shape, that single person, strange how a mistress knows the
very moment, the very temper of her house!  Faint sounds rose in
spirals up the well of the stairs; the swish of a mop; tapping;
knocking; a loudness when the front door opened; a voice repeating
a message in the basement; the chink of silver on a tray; clean
silver for the party.  All was for the party.

(And Lucy, coming into the drawing-room with her tray held out, put
the giant candlesticks on the mantelpiece, the silver casket in the
middle, turned the crystal dolphin towards the clock.  They would
come; they would stand; they would talk in the mincing tones which
she could imitate, ladies and gentlemen.  Of all, her mistress was
loveliest--mistress of silver, of linen, of china, for the sun, the
silver, doors off their hinges, Rumpelmayer's men, gave her a
sense, as she laid the paper-knife on the inlaid table, of
something achieved.  Behold!  Behold! she said, speaking to her old
friends in the baker's shop, where she had first seen service at
Caterham, prying into the glass.  She was Lady Angela, attending
Princess Mary, when in came Mrs. Dalloway.)

"Oh Lucy," she said, "the silver does look nice!"

"And how," she said, turning the crystal dolphin to stand straight,
"how did you enjoy the play last night?"  "Oh, they had to go
before the end!" she said.  "They had to be back at ten!" she said.
"So they don't know what happened," she said.  "That does seem hard
luck," she said (for her servants stayed later, if they asked her).
"That does seem rather a shame," she said, taking the old bald-
looking cushion in the middle of the sofa and putting it in Lucy's
arms, and giving her a little push, and crying:

"Take it away!  Give it to Mrs. Walker with my compliments!  Take
it away!" she cried.

And Lucy stopped at the drawing-room door, holding the cushion, and
said, very shyly, turning a little pink, Couldn't she help to mend
that dress?

But, said Mrs. Dalloway, she had enough on her hands already, quite
enough of her own to do without that.

"But, thank you, Lucy, oh, thank you," said Mrs. Dalloway, and
thank you, thank you, she went on saying (sitting down on the sofa
with her dress over her knees, her scissors, her silks), thank you,
thank you, she went on saying in gratitude to her servants
generally for helping her to be like this, to be what she wanted,
gentle, generous-hearted.  Her servants liked her.  And then this
dress of hers--where was the tear? and now her needle to be
threaded.  This was a favourite dress, one of Sally Parker's, the
last almost she ever made, alas, for Sally had now retired, living
at Ealing, and if ever I have a moment, thought Clarissa (but never
would she have a moment any more), I shall go and see her at
Ealing.  For she was a character, thought Clarissa, a real artist.
She thought of little out-of-the-way things; yet her dresses were
never queer.  You could wear them at Hatfield; at Buckingham
Palace.  She had worn them at Hatfield; at Buckingham Palace.

Quiet descended on her, calm, content, as her needle, drawing the
silk smoothly to its gentle pause, collected the green folds
together and attached them, very lightly, to the belt.  So on a
summer's day waves collect, overbalance, and fall; collect and
fall; and the whole world seems to be saying "that is all" more and
more ponderously, until even the heart in the body which lies in
the sun on the beach says too, That is all.  Fear no more, says the
heart.  Fear no more, says the heart, committing its burden to some
sea, which sighs collectively for all sorrows, and renews, begins,
collects, lets fall.  And the body alone listens to the passing
bee; the wave breaking; the dog barking, far away barking and
barking.

"Heavens, the front-door bell!" exclaimed Clarissa, staying her
needle.  Roused, she listened.

"Mrs. Dalloway will see me," said the elderly man in the hall.  "Oh
yes, she will see ME," he repeated, putting Lucy aside very
benevolently, and running upstairs ever so quickly.  "Yes, yes,
yes," he muttered as he ran upstairs.  "She will see me.  After
five years in India, Clarissa will see me."

"Who can--what can," asked Mrs. Dalloway (thinking it was
outrageous to be interrupted at eleven o'clock on the morning of
the day she was giving a party), hearing a step on the stairs.  She
heard a hand upon the door.  She made to hide her dress, like a
virgin protecting chastity, respecting privacy.  Now the brass knob
slipped.  Now the door opened, and in came--for a single second she
could not remember what he was called! so surprised she was to see
him, so glad, so shy, so utterly taken aback to have Peter Walsh
come to her unexpectedly in the morning!  (She had not read his
letter.)

"And how are you?" said Peter Walsh, positively trembling; taking
both her hands; kissing both her hands.  She's grown older, he
thought, sitting down.  I shan't tell her anything about it, he
thought, for she's grown older.  She's looking at me, he thought,
a sudden embarrassment coming over him, though he had kissed her
hands.  Putting his hand into his pocket, he took out a large
pocket-knife and half opened the blade.

Exactly the same, thought Clarissa; the same queer look; the same
check suit; a little out of the straight his face is, a little
thinner, dryer, perhaps, but he looks awfully well, and just the
same.

"How heavenly it is to see you again!" she exclaimed.  He had his
knife out.  That's so like him, she thought.

He had only reached town last night, he said; would have to go down
into the country at once; and how was everything, how was
everybody--Richard?  Elizabeth?

"And what's all this?" he said, tilting his pen-knife towards her
green dress.

He's very well dressed, thought Clarissa; yet he always criticises
ME.

Here she is mending her dress; mending her dress as usual, he
thought; here she's been sitting all the time I've been in India;
mending her dress; playing about; going to parties; running to the
House and back and all that, he thought, growing more and more
irritated, more and more agitated, for there's nothing in the world
so bad for some women as marriage, he thought; and politics; and
having a Conservative husband, like the admirable Richard.  So it
is, so it is, he thought, shutting his knife with a snap.

"Richard's very well.  Richard's at a Committee," said Clarissa.

And she opened her scissors, and said, did he mind her just
finishing what she was doing to her dress, for they had a party
that night?

"Which I shan't ask you to," she said.  "My dear Peter!" she said.

But it was delicious to hear her say that--my dear Peter!  Indeed,
it was all so delicious--the silver, the chairs; all so delicious!

Why wouldn't she ask him to her party? he asked.

Now of course, thought Clarissa, he's enchanting! perfectly
enchanting!  Now I remember how impossible it was ever to make up
my mind--and why did I make up my mind--not to marry him? she
wondered, that awful summer?

"But it's so extraordinary that you should have come this morning!"
she cried, putting her hands, one on top of another, down on her
dress.

"Do you remember," she said, "how the blinds used to flap at
Bourton?"

"They did," he said; and he remembered breakfasting alone, very
awkwardly, with her father; who had died; and he had not written to
Clarissa.  But he had never got on well with old Parry, that
querulous, weak-kneed old man, Clarissa's father, Justin Parry.

"I often wish I'd got on better with your father," he said.

"But he never liked any one who--our friends," said Clarissa; and
could have bitten her tongue for thus reminding Peter that he had
wanted to marry her.

Of course I did, thought Peter; it almost broke my heart too, he
thought; and was overcome with his own grief, which rose like a
moon looked at from a terrace, ghastly beautiful with light from
the sunken day.  I was more unhappy than I've ever been since, he
thought.  And as if in truth he were sitting there on the terrace
he edged a little towards Clarissa; put his hand out; raised it;
let it fall.  There above them it hung, that moon.  She too seemed
to be sitting with him on the terrace, in the moonlight.

"Herbert has it now," she said.  "I never go there now," she said.

Then, just as happens on a terrace in the moonlight, when one
person begins to feel ashamed that he is already bored, and yet as
the other sits silent, very quiet, sadly looking at the moon, does
not like to speak, moves his foot, clears his throat, notices some
iron scroll on a table leg, stirs a leaf, but says nothing--so
Peter Walsh did now.  For why go back like this to the past? he
thought.  Why make him think of it again?  Why make him suffer,
when she had tortured him so infernally?  Why?

"Do you remember the lake?" she said, in an abrupt voice, under the
pressure of an emotion which caught her heart, made the muscles of
her throat stiff, and contracted her lips in a spasm as she said
"lake."  For she was a child, throwing bread to the ducks, between
her parents, and at the same time a grown woman coming to her
parents who stood by the lake, holding her life in her arms which,
as she neared them, grew larger and larger in her arms, until it
became a whole life, a complete life, which she put down by them
and said, "This is what I have made of it!  This!"  And what had
she made of it?  What, indeed? sitting there sewing this morning
with Peter.

She looked at Peter Walsh; her look, passing through all that time
and that emotion, reached him doubtfully; settled on him tearfully;
and rose and fluttered away, as a bird touches a branch and rises
and flutters away.  Quite simply she wiped her eyes.

"Yes," said Peter.  "Yes, yes, yes," he said, as if she drew up to
the surface something which positively hurt him as it rose.  Stop!
Stop! he wanted to cry.  For he was not old; his life was not over;
not by any means.  He was only just past fifty.  Shall I tell her,
he thought, or not?  He would like to make a clean breast of it
all.  But she is too cold, he thought; sewing, with her scissors;
Daisy would look ordinary beside Clarissa.  And she would think me
a failure, which I am in their sense, he thought; in the Dalloways'
sense.  Oh yes, he had no doubt about that; he was a failure,
compared with all this--the inlaid table, the mounted paper-knife,
the dolphin and the candlesticks, the chair-covers and the old
valuable English tinted prints--he was a failure!  I detest the
smugness of the whole affair, he thought; Richard's doing, not
Clarissa's; save that she married him.  (Here Lucy came into the
room, carrying silver, more silver, but charming, slender, graceful
she looked, he thought, as she stooped to put it down.)  And this
has been going on all the time! he thought; week after week;
Clarissa's life; while I--he thought; and at once everything seemed
to radiate from him; journeys; rides; quarrels; adventures; bridge
parties; love affairs; work; work, work! and he took out his knife
quite openly--his old horn-handled knife which Clarissa could swear
he had had these thirty years--and clenched his fist upon it.

What an extraordinary habit that was, Clarissa thought; always
playing with a knife.  Always making one feel, too, frivolous;
empty-minded; a mere silly chatterbox, as he used.  But I too, she
thought, and, taking up her needle, summoned, like a Queen whose
guards have fallen asleep and left her unprotected (she had been
quite taken aback by this visit--it had upset her) so that any one
can stroll in and have a look at her where she lies with the
brambles curving over her, summoned to her help the things she did;
the things she liked; her husband; Elizabeth; her self, in short,
which Peter hardly knew now, all to come about her and beat off the
enemy.

"Well, and what's happened to you?" she said.  So before a battle
begins, the horses paw the ground; toss their heads; the light
shines on their flanks; their necks curve.  So Peter Walsh and
Clarissa, sitting side by side on the blue sofa, challenged each
other.  His powers chafed and tossed in him.  He assembled from
different quarters all sorts of things; praise; his career at
Oxford; his marriage, which she knew nothing whatever about; how he
had loved; and altogether done his job.

"Millions of things!" he exclaimed, and, urged by the assembly of
powers which were now charging this way and that and giving him the
feeling at once frightening and extremely exhilarating of being
rushed through the air on the shoulders of people he could no
longer see, he raised his hands to his forehead.

Clarissa sat very upright; drew in her breath.

"I am in love," he said, not to her however, but to some one raised
up in the dark so that you could not touch her but must lay your
garland down on the grass in the dark.

"In love," he repeated, now speaking rather dryly to Clarissa
Dalloway; "in love with a girl in India."  He had deposited his
garland.  Clarissa could make what she would of it.

"In love!" she said.  That he at his age should be sucked under in
his little bow-tie by that monster!  And there's no flesh on his
neck; his hands are red; and he's six months older than I am! her
eye flashed back to her; but in her heart she felt, all the same,
he is in love.  He has that, she felt; he is in love.

But the indomitable egotism which for ever rides down the hosts
opposed to it, the river which says on, on, on; even though, it
admits, there may be no goal for us whatever, still on, on; this
indomitable egotism charged her cheeks with colour; made her look
very young; very pink; very bright-eyed as she sat with her dress
upon her knee, and her needle held to the end of green silk,
trembling a little.  He was in love!  Not with her.  With some
younger woman, of course.

"And who is she?" she asked.

Now this statue must be brought from its height and set down
between them.

"A married woman, unfortunately," he said; "the wife of a Major in
the Indian Army."

And with a curious ironical sweetness he smiled as he placed her in
this ridiculous way before Clarissa.

(All the same, he is in love, thought Clarissa.)

"She has," he continued, very reasonably, "two small children; a
boy and a girl; and I have come over to see my lawyers about the
divorce."

There they are! he thought.  Do what you like with them, Clarissa!
There they are!  And second by second it seemed to him that the
wife of the Major in the Indian Army (his Daisy) and her two small
children became more and more lovely as Clarissa looked at them; as
if he had set light to a grey pellet on a plate and there had risen
up a lovely tree in the brisk sea-salted air of their intimacy (for
in some ways no one understood him, felt with him, as Clarissa
did)--their exquisite intimacy.

She flattered him; she fooled him, thought Clarissa; shaping the
woman, the wife of the Major in the Indian Army, with three strokes
of a knife.  What a waste!  What a folly!  All his life long Peter
had been fooled like that; first getting sent down from Oxford;
next marrying the girl on the boat going out to India; now the wife
of a Major in the Indian Army--thank Heaven she had refused to
marry him!  Still, he was in love; her old friend, her dear Peter,
he was in love.

"But what are you going to do?" she asked him.  Oh the lawyers and
solicitors, Messrs. Hooper and Grateley of Lincoln's Inn, they were
going to do it, he said.  And he actually pared his nails with his
pocket-knife.

For Heaven's sake, leave your knife alone! she cried to herself in
irrepressible irritation; it was his silly unconventionality, his
weakness; his lack of the ghost of a notion what any one else was
feeling that annoyed her, had always annoyed her; and now at his
age, how silly!

I know all that, Peter thought; I know what I'm up against, he
thought, running his finger along the blade of his knife, Clarissa
and Dalloway and all the rest of them; but I'll show Clarissa--and
then to his utter surprise, suddenly thrown by those uncontrollable
forces thrown through the air, he burst into tears; wept; wept
without the least shame, sitting on the sofa, the tears running
down his cheeks.

And Clarissa had leant forward, taken his hand, drawn him to her,
kissed him,--actually had felt his face on hers before she could
down the brandishing of silver flashing--plumes like pampas grass
in a tropic gale in her breast, which, subsiding, left her
holding his hand, patting his knee and, feeling as she sat back
extraordinarily at her ease with him and light-hearted, all in a
clap it came over her, If I had married him, this gaiety would have
been mine all day!

It was all over for her.  The sheet was stretched and the bed
narrow.  She had gone up into the tower alone and left them
blackberrying in the sun.  The door had shut, and there among the
dust of fallen plaster and the litter of birds' nests how distant
the view had looked, and the sounds came thin and chill (once on
Leith Hill, she remembered), and Richard, Richard! she cried, as a
sleeper in the night starts and stretches a hand in the dark for
help.  Lunching with Lady Bruton, it came back to her.  He has left
me; I am alone for ever, she thought, folding her hands upon her
knee.

Peter Walsh had got up and crossed to the window and stood with his
back to her, flicking a bandanna handkerchief from side to side.
Masterly and dry and desolate he looked, his thin shoulder-blades
lifting his coat slightly; blowing his nose violently.  Take me
with you, Clarissa thought impulsively, as if he were starting
directly upon some great voyage; and then, next moment, it was as
if the five acts of a play that had been very exciting and moving
were now over and she had lived a lifetime in them and had run
away, had lived with Peter, and it was now over.

Now it was time to move, and, as a woman gathers her things
together, her cloak, her gloves, her opera-glasses, and gets up to
go out of the theatre into the street, she rose from the sofa and
went to Peter.

And it was awfully strange, he thought, how she still had the
power, as she came tinkling, rustling, still had the power as she
came across the room, to make the moon, which he detested, rise at
Bourton on the terrace in the summer sky.

"Tell me," he said, seizing her by the shoulders.  "Are you happy,
Clarissa?  Does Richard--"

The door opened.

"Here is my Elizabeth," said Clarissa, emotionally, histrionically,
perhaps.

"How d'y do?" said Elizabeth coming forward.

The sound of Big Ben striking the half-hour struck out between them
with extraordinary vigour, as if a young man, strong, indifferent,
inconsiderate, were swinging dumb-bells this way and that.

"Hullo, Elizabeth!" cried Peter, stuffing his handkerchief into his
pocket, going quickly to her, saying "Good-bye, Clarissa" without
looking at her, leaving the room quickly, and running downstairs
and opening the hall door.

"Peter!  Peter!" cried Clarissa, following him out on to the
landing.  "My party to-night!  Remember my party to-night!" she
cried, having to raise her voice against the roar of the open air,
and, overwhelmed by the traffic and the sound of all the clocks
striking, her voice crying "Remember my party to-night!" sounded
frail and thin and very far away as Peter Walsh shut the door.



Remember my party, remember my party, said Peter Walsh as he
stepped down the street, speaking to himself rhythmically, in time
with the flow of the sound, the direct downright sound of Big Ben
striking the half-hour.  (The leaden circles dissolved in the air.)
Oh these parties, he thought; Clarissa's parties.  Why does she
give these parties, he thought.  Not that he blamed her or this
effigy of a man in a tail-coat with a carnation in his buttonhole
coming towards him.  Only one person in the world could be as he
was, in love.  And there he was, this fortunate man, himself,
reflected in the plate-glass window of a motor-car manufacturer in
Victoria Street.  All India lay behind him; plains, mountains;
epidemics of cholera; a district twice as big as Ireland; decisions
he had come to alone--he, Peter Walsh; who was now really for the
first time in his life, in love.  Clarissa had grown hard, he
thought; and a trifle sentimental into the bargain, he suspected,
looking at the great motor-cars capable of doing--how many miles on
how many gallons?  For he had a turn for mechanics; had invented a
plough in his district, had ordered wheel-barrows from England, but
the coolies wouldn't use them, all of which Clarissa knew nothing
whatever about.

The way she said "Here is my Elizabeth!"--that annoyed him.  Why
not "Here's Elizabeth" simply?  It was insincere.  And Elizabeth
didn't like it either.  (Still the last tremors of the great
booming voice shook the air round him; the half-hour; still early;
only half-past eleven still.)  For he understood young people; he
liked them.  There was always something cold in Clarissa, he
thought.  She had always, even as a girl, a sort of timidity, which
in middle age becomes conventionality, and then it's all up, it's
all up, he thought, looking rather drearily into the glassy depths,
and wondering whether by calling at that hour he had annoyed her;
overcome with shame suddenly at having been a fool; wept; been
emotional; told her everything, as usual, as usual.

As a cloud crosses the sun, silence falls on London; and falls on
the mind.  Effort ceases.  Time flaps on the mast.  There we stop;
there we stand.  Rigid, the skeleton of habit alone upholds the
human frame.  Where there is nothing, Peter Walsh said to himself;
feeling hollowed out, utterly empty within.  Clarissa refused me,
he thought.  He stood there thinking, Clarissa refused me.

Ah, said St. Margaret's, like a hostess who comes into her drawing-
room on the very stroke of the hour and finds her guests there
already.  I am not late.  No, it is precisely half-past eleven, she
says.  Yet, though she is perfectly right, her voice, being the
voice of the hostess, is reluctant to inflict its individuality.
Some grief for the past holds it back; some concern for the
present.  It is half-past eleven, she says, and the sound of St.
Margaret's glides into the recesses of the heart and buries itself
in ring after ring of sound, like something alive which wants to
confide itself, to disperse itself, to be, with a tremor of
delight, at rest--like Clarissa herself, thought Peter Walsh,
coming down the stairs on the stroke of the hour in white.  It is
Clarissa herself, he thought, with a deep emotion, and an
extraordinarily clear, yet puzzling, recollection of her, as if
this bell had come into the room years ago, where they sat at some
moment of great intimacy, and had gone from one to the other and
had left, like a bee with honey, laden with the moment.  But what
room?  What moment?  And why had he been so profoundly happy when
the clock was striking?  Then, as the sound of St. Margaret's
languished, he thought, She has been ill, and the sound expressed
languor and suffering.  It was her heart, he remembered; and the
sudden loudness of the final stroke tolled for death that surprised
in the midst of life, Clarissa falling where she stood, in her
drawing-room.  No!  No! he cried.  She is not dead!  I am not old,
he cried, and marched up Whitehall, as if there rolled down to him,
vigorous, unending, his future.

He was not old, or set, or dried in the least.  As for caring what
they said of him--the Dalloways, the Whitbreads, and their set, he
cared not a straw--not a straw (though it was true he would have,
some time or other, to see whether Richard couldn't help him to
some job).  Striding, staring, he glared at the statue of the Duke
of Cambridge.  He had been sent down from Oxford--true.  He had
been a Socialist, in some sense a failure--true.  Still the future
of civilisation lies, he thought, in the hands of young men like
that; of young men such as he was, thirty years ago; with their
love of abstract principles; getting books sent out to them all the
way from London to a peak in the Himalayas; reading science;
reading philosophy.  The future lies in the hands of young men like
that, he thought.

A patter like the patter of leaves in a wood came from behind, and
with it a rustling, regular thudding sound, which as it overtook
him drummed his thoughts, strict in step, up Whitehall, without his
doing.  Boys in uniform, carrying guns, marched with their eyes
ahead of them, marched, their arms stiff, and on their faces an
expression like the letters of a legend written round the base of a
statue praising duty, gratitude, fidelity, love of England.

It is, thought Peter Walsh, beginning to keep step with them, a
very fine training.  But they did not look robust.  They were weedy
for the most part, boys of sixteen, who might, to-morrow, stand
behind bowls of rice, cakes of soap on counters.  Now they wore on
them unmixed with sensual pleasure or daily preoccupations the
solemnity of the wreath which they had fetched from Finsbury
Pavement to the empty tomb.  They had taken their vow.  The traffic
respected it; vans were stopped.

I can't keep up with them, Peter Walsh thought, as they marched up
Whitehall, and sure enough, on they marched, past him, past every
one, in their steady way, as if one will worked legs and arms
uniformly, and life, with its varieties, its irreticences, had been
laid under a pavement of monuments and wreaths and drugged into a
stiff yet staring corpse by discipline.  One had to respect it; one
might laugh; but one had to respect it, he thought.  There they go,
thought Peter Walsh, pausing at the edge of the pavement; and all
the exalted statues, Nelson, Gordon, Havelock, the black, the
spectacular images of great soldiers stood looking ahead of them,
as if they too had made the same renunciation (Peter Walsh felt he
too had made it, the great renunciation), trampled under the same
temptations, and achieved at length a marble stare.  But the stare
Peter Walsh did not want for himself in the least; though he could
respect it in others.  He could respect it in boys.  They don't
know the troubles of the flesh yet, he thought, as the marching
boys disappeared in the direction of the Strand--all that I've been
through, he thought, crossing the road, and standing under Gordon's
statue, Gordon whom as a boy he had worshipped; Gordon standing
lonely with one leg raised and his arms crossed,--poor Gordon, he
thought.

And just because nobody yet knew he was in London, except Clarissa,
and the earth, after the voyage, still seemed an island to him, the
strangeness of standing alone, alive, unknown, at half-past eleven
in Trafalgar Square overcame him.  What is it?  Where am I?  And
why, after all, does one do it? he thought, the divorce seeming all
moonshine.  And down his mind went flat as a marsh, and three great
emotions bowled over him; understanding; a vast philanthropy; and
finally, as if the result of the others, an irrepressible,
exquisite delight; as if inside his brain by another hand strings
were pulled, shutters moved, and he, having nothing to do with it,
yet stood at the opening of endless avenues, down which if he chose
he might wander.  He had not felt so young for years.

He had escaped! was utterly free--as happens in the downfall of
habit when the mind, like an unguarded flame, bows and bends and
seems about to blow from its holding.  I haven't felt so young for
years! thought Peter, escaping (only of course for an hour or so)
from being precisely what he was, and feeling like a child who runs
out of doors, and sees, as he runs, his old nurse waving at the
wrong window.  But she's extraordinarily attractive, he thought,
as, walking across Trafalgar Square in the direction of the
Haymarket, came a young woman who, as she passed Gordon's statue,
seemed, Peter Walsh thought (susceptible as he was), to shed veil
after veil, until she became the very woman he had always had in
mind; young, but stately; merry, but discreet; black, but
enchanting.

Straightening himself and stealthily fingering his pocket-knife he
started after her to follow this woman, this excitement, which
seemed even with its back turned to shed on him a light which
connected them, which singled him out, as if the random uproar of
the traffic had whispered through hollowed hands his name, not
Peter, but his private name which he called himself in his own
thoughts.  "You," she said, only "you," saying it with her white
gloves and her shoulders.  Then the thin long cloak which the wind
stirred as she walked past Dent's shop in Cockspur Street blew out
with an enveloping kindness, a mournful tenderness, as of arms that
would open and take the tired--

But she's not married; she's young; quite young, thought Peter, the
red carnation he had seen her wear as she came across Trafalgar
Square burning again in his eyes and making her lips red.  But she
waited at the kerbstone.  There was a dignity about her.  She was
not worldly, like Clarissa; not rich, like Clarissa.  Was she, he
wondered as she moved, respectable?  Witty, with a lizard's
flickering tongue, he thought (for one must invent, must allow
oneself a little diversion), a cool waiting wit, a darting wit; not
noisy.

She moved; she crossed; he followed her.  To embarrass her was the
last thing he wished.  Still if she stopped he would say "Come and
have an ice," he would say, and she would answer, perfectly simply,
"Oh yes."

But other people got between them in the street, obstructing him,
blotting her out.  He pursued; she changed.  There was colour in
her cheeks; mockery in her eyes; he was an adventurer, reckless, he
thought, swift, daring, indeed (landed as he was last night from
India) a romantic buccaneer, careless of all these damned
proprieties, yellow dressing-gowns, pipes, fishing-rods, in the
shop windows; and respectability and evening parties and spruce old
men wearing white slips beneath their waistcoats.  He was a
buccaneer.  On and on she went, across Piccadilly, and up Regent
Street, ahead of him, her cloak, her gloves, her shoulders
combining with the fringes and the laces and the feather boas in
the windows to make the spirit of finery and whimsy which dwindled
out of the shops on to the pavement, as the light of a lamp goes
wavering at night over hedges in the darkness.

Laughing and delightful, she had crossed Oxford Street and Great
Portland Street and turned down one of the little streets, and now,
and now, the great moment was approaching, for now she slackened,
opened her bag, and with one look in his direction, but not at him,
one look that bade farewell, summed up the whole situation and
dismissed it triumphantly, for ever, had fitted her key, opened the
door, and gone!  Clarissa's voice saying, Remember my party,
Remember my party, sang in his ears.  The house was one of those
flat red houses with hanging flower-baskets of vague impropriety.
It was over.

Well, I've had my fun; I've had it, he thought, looking up at the
swinging baskets of pale geraniums.  And it was smashed to atoms--
his fun, for it was half made up, as he knew very well; invented,
this escapade with the girl; made up, as one makes up the better
part of life, he thought--making oneself up; making her up;
creating an exquisite amusement, and something more.  But odd it
was, and quite true; all this one could never share--it smashed to
atoms.

He turned; went up the street, thinking to find somewhere to sit,
till it was time for Lincoln's Inn--for Messrs. Hooper and
Grateley.  Where should he go?  No matter.  Up the street, then,
towards Regent's Park.  His boots on the pavement struck out "no
matter"; for it was early, still very early.

It was a splendid morning too.  Like the pulse of a perfect heart,
life struck straight through the streets.  There was no fumbling--
no hesitation.  Sweeping and swerving, accurately, punctually,
noiselessly, there, precisely at the right instant, the motor-car
stopped at the door.  The girl, silk-stockinged, feathered,
evanescent, but not to him particularly attractive (for he had had
his fling), alighted.  Admirable butlers, tawny chow dogs, halls
laid in black and white lozenges with white blinds blowing, Peter
saw through the opened door and approved of.  A splendid
achievement in its own way, after all, London; the season;
civilisation.  Coming as he did from a respectable Anglo-Indian
family which for at least three generations had administered the
affairs of a continent (it's strange, he thought, what a sentiment
I have about that, disliking India, and empire, and army as he
did), there were moments when civilisation, even of this sort,
seemed dear to him as a personal possession; moments of pride in
England; in butlers; chow dogs; girls in their security.
Ridiculous enough, still there it is, he thought.  And the doctors
and men of business and capable women all going about their
business, punctual, alert, robust, seemed to him wholly admirable,
good fellows, to whom one would entrust one's life, companions in
the art of living, who would see one through.  What with one thing
and another, the show was really very tolerable; and he would sit
down in the shade and smoke.

There was Regent's Park.  Yes.  As a child he had walked in
Regent's Park--odd, he thought, how the thought of childhood keeps
coming back to me--the result of seeing Clarissa, perhaps; for
women live much more in the past than we do, he thought.  They
attach themselves to places; and their fathers--a woman's always
proud of her father.  Bourton was a nice place, a very nice place,
but I could never get on with the old man, he thought.  There was
quite a scene one night--an argument about something or other,
what, he could not remember.  Politics presumably.

Yes, he remembered Regent's Park; the long straight walk; the
little house where one bought air-balls to the left; an absurd
statue with an inscription somewhere or other.  He looked for an
empty seat.  He did not want to be bothered (feeling a little
drowsy as he did) by people asking him the time.  An elderly grey
nurse, with a baby asleep in its perambulator--that was the best he
could do for himself; sit down at the far end of the seat by that
nurse.

She's a queer-looking girl, he thought, suddenly remembering
Elizabeth as she came into the room and stood by her mother.  Grown
big; quite grown-up, not exactly pretty; handsome rather; and she
can't be more than eighteen.  Probably she doesn't get on with
Clarissa.  "There's my Elizabeth"--that sort of thing--why not
"Here's Elizabeth" simply?--trying to make out, like most mothers,
that things are what they're not.  She trusts to her charm too
much, he thought.  She overdoes it.

The rich benignant cigar smoke eddied coolly down his throat; he
puffed it out again in rings which breasted the air bravely for a
moment; blue, circular--I shall try and get a word alone with
Elizabeth to-night, he thought--then began to wobble into hour-
glass shapes and taper away; odd shapes they take, he thought.
Suddenly he closed his eyes, raised his hand with an effort, and
threw away the heavy end of his cigar.  A great brush swept smooth
across his mind, sweeping across it moving branches, children's
voices, the shuffle of feet, and people passing, and humming
traffic, rising and falling traffic.  Down, down he sank into the
plumes and feathers of sleep, sank, and was muffled over.



The grey nurse resumed her knitting as Peter Walsh, on the hot seat
beside her, began snoring.  In her grey dress, moving her hands
indefatigably yet quietly, she seemed like the champion of the
rights of sleepers, like one of those spectral presences which rise
in twilight in woods made of sky and branches.  The solitary
traveller, haunter of lanes, disturber of ferns, and devastator of
great hemlock plants, looking up, suddenly sees the giant figure at
the end of the ride.

By conviction an atheist perhaps, he is taken by surprise with
moments of extraordinary exaltation.  Nothing exists outside us
except a state of mind, he thinks; a desire for solace, for relief,
for something outside these miserable pigmies, these feeble, these
ugly, these craven men and women.  But if he can conceive of her,
then in some sort she exists, he thinks, and advancing down the
path with his eyes upon sky and branches he rapidly endows them
with womanhood; sees with amazement how grave they become; how
majestically, as the breeze stirs them, they dispense with a dark
flutter of the leaves charity, comprehension, absolution, and then,
flinging themselves suddenly aloft, confound the piety of their
aspect with a wild carouse.

Such are the visions which proffer great cornucopias full of fruit
to the solitary traveller, or murmur in his ear like sirens
lolloping away on the green sea waves, or are dashed in his face
like bunches of roses, or rise to the surface like pale faces which
fishermen flounder through floods to embrace.

Such are the visions which ceaselessly float up, pace beside, put
their faces in front of, the actual thing; often overpowering the
solitary traveller and taking away from him the sense of the earth,
the wish to return, and giving him for substitute a general peace,
as if (so he thinks as he advances down the forest ride) all this
fever of living were simplicity itself; and myriads of things
merged in one thing; and this figure, made of sky and branches as
it is, had risen from the troubled sea (he is elderly, past fifty
now) as a shape might be sucked up out of the waves to shower down
from her magnificent hands compassion, comprehension, absolution.
So, he thinks, may I never go back to the lamplight; to the
sitting-room; never finish my book; never knock out my pipe; never
ring for Mrs. Turner to clear away; rather let me walk straight on
to this great figure, who will, with a toss of her head, mount me
on her streamers and let me blow to nothingness with the rest.

Such are the visions.  The solitary traveller is soon beyond the
wood; and there, coming to the door with shaded eyes, possibly to
look for his return, with hands raised, with white apron blowing,
is an elderly woman who seems (so powerful is this infirmity) to
seek, over a desert, a lost son; to search for a rider destroyed;
to be the figure of the mother whose sons have been killed in the
battles of the world.  So, as the solitary traveller advances down
the village street where the women stand knitting and the men dig
in the garden, the evening seems ominous; the figures still; as if
some august fate, known to them, awaited without fear, were about
to sweep them into complete annihilation.

Indoors among ordinary things, the cupboard, the table, the window-
sill with its geraniums, suddenly the outline of the landlady,
bending to remove the cloth, becomes soft with light, an adorable
emblem which only the recollection of cold human contacts forbids
us to embrace.  She takes the marmalade; she shuts it in the
cupboard.

"There is nothing more to-night, sir?"

But to whom does the solitary traveller make reply?



So the elderly nurse knitted over the sleeping baby in Regent's
Park.  So Peter Walsh snored.

He woke with extreme suddenness, saying to himself, "The death of
the soul."

"Lord, Lord!" he said to himself out loud, stretching and opening
his eyes.  "The death of the soul."  The words attached themselves
to some scene, to some room, to some past he had been dreaming of.
It became clearer; the scene, the room, the past he had been
dreaming of.

It was at Bourton that summer, early in the 'nineties, when he was
so passionately in love with Clarissa.  There were a great many
people there, laughing and talking, sitting round a table after tea
and the room was bathed in yellow light and full of cigarette
smoke.  They were talking about a man who had married his
housemaid, one of the neighbouring squires, he had forgotten his
name.  He had married his housemaid, and she had been brought to
Bourton to call--an awful visit it had been.  She was absurdly
over-dressed, "like a cockatoo," Clarissa had said, imitating her,
and she never stopped talking.  On and on she went, on and on.
Clarissa imitated her.  Then somebody said--Sally Seton it was--did
it make any real difference to one's feelings to know that before
they'd married she had had a baby?  (In those days, in mixed
company, it was a bold thing to say.)  He could see Clarissa now,
turning bright pink; somehow contracting; and saying, "Oh, I shall
never be able to speak to her again!"  Whereupon the whole party
sitting round the tea-table seemed to wobble.  It was very
uncomfortable.

He hadn't blamed her for minding the fact, since in those days a
girl brought up as she was, knew nothing, but it was her manner
that annoyed him; timid; hard; something arrogant; unimaginative;
prudish.  "The death of the soul."  He had said that instinctively,
ticketing the moment as he used to do--the death of her soul.

Every one wobbled; every one seemed to bow, as she spoke, and then
to stand up different.  He could see Sally Seton, like a child who
has been in mischief, leaning forward, rather flushed, wanting to
talk, but afraid, and Clarissa did frighten people.  (She was
Clarissa's greatest friend, always about the place, totally unlike
her, an attractive creature, handsome, dark, with the reputation in
those days of great daring and he used to give her cigars, which
she smoked in her bedroom.  She had either been engaged to somebody
or quarrelled with her family and old Parry disliked them both
equally, which was a great bond.)  Then Clarissa, still with an air
of being offended with them all, got up, made some excuse, and went
off, alone.  As she opened the door, in came that great shaggy dog
which ran after sheep.  She flung herself upon him, went into
raptures.  It was as if she said to Peter--it was all aimed at him,
he knew--"I know you thought me absurd about that woman just now;
but see how extraordinarily sympathetic I am; see how I love my
Rob!"

They had always this queer power of communicating without words.
She knew directly he criticised her.  Then she would do something
quite obvious to defend herself, like this fuss with the dog--but
it never took him in, he always saw through Clarissa.  Not that he
said anything, of course; just sat looking glum.  It was the way
their quarrels often began.

She shut the door.  At once he became extremely depressed.  It all
seemed useless--going on being in love; going on quarrelling; going
on making it up, and he wandered off alone, among outhouses,
stables, looking at the horses.  (The place was quite a humble one;
the Parrys were never very well off; but there were always grooms
and stable-boys about--Clarissa loved riding--and an old coachman--
what was his name?--an old nurse, old Moody, old Goody, some such
name they called her, whom one was taken to visit in a little room
with lots of photographs, lots of bird-cages.)

It was an awful evening!  He grew more and more gloomy, not about
that only; about everything.  And he couldn't see her; couldn't
explain to her; couldn't have it out.  There were always people
about--she'd go on as if nothing had happened.  That was the
devilish part of her--this coldness, this woodenness, something
very profound in her, which he had felt again this morning talking
to her; an impenetrability.  Yet Heaven knows he loved her.  She
had some queer power of fiddling on one's nerves, turning one's
nerves to fiddle-strings, yes.

He had gone in to dinner rather late, from some idiotic idea of
making himself felt, and had sat down by old Miss Parry--Aunt
Helena--Mr. Parry's sister, who was supposed to preside.  There she
sat in her white Cashmere shawl, with her head against the window--
a formidable old lady, but kind to him, for he had found her some
rare flower, and she was a great botanist, marching off in thick
boots with a black collecting-box slung between her shoulders.  He
sat down beside her, and couldn't speak.  Everything seemed to race
past him; he just sat there, eating.  And then half-way through
dinner he made himself look across at Clarissa for the first time.
She was talking to a young man on her right.  He had a sudden
revelation.  "She will marry that man," he said to himself.  He
didn't even know his name.

For of course it was that afternoon, that very afternoon, that
Dalloway had come over; and Clarissa called him "Wickham"; that was
the beginning of it all.  Somebody had brought him over; and
Clarissa got his name wrong.  She introduced him to everybody as
Wickham.  At last he said "My name is Dalloway!"--that was his
first view of Richard--a fair young man, rather awkward, sitting on
a deck-chair, and blurting out "My name is Dalloway!"  Sally got
hold of it; always after that she called him "My name is Dalloway!"

He was a prey to revelations at that time.  This one--that she
would marry Dalloway--was blinding--overwhelming at the moment.
There was a sort of--how could he put it?--a sort of ease in her
manner to him; something maternal; something gentle.  They were
talking about politics.  All through dinner he tried to hear what
they were saying.

Afterwards he could remember standing by old Miss Parry's chair in
the drawing-room.  Clarissa came up, with her perfect manners, like
a real hostess, and wanted to introduce him to some one--spoke as
if they had never met before, which enraged him.  Yet even then he
admired her for it.  He admired her courage; her social instinct;
he admired her power of carrying things through.  "The perfect
hostess," he said to her, whereupon she winced all over.  But he
meant her to feel it.  He would have done anything to hurt her
after seeing her with Dalloway.  So she left him.  And he had a
feeling that they were all gathered together in a conspiracy
against him--laughing and talking--behind his back.  There he stood
by Miss Parry's chair as though he had been cut out of wood, he
talking about wild flowers.  Never, never had he suffered so
infernally!  He must have forgotten even to pretend to listen; at
last he woke up; he saw Miss Parry looking rather disturbed, rather
indignant, with her prominent eyes fixed.  He almost cried out that
he couldn't attend because he was in Hell!  People began going out
of the room.  He heard them talking about fetching cloaks; about
its being cold on the water, and so on.  They were going boating on
the lake by moonlight--one of Sally's mad ideas.  He could hear her
describing the moon.  And they all went out.  He was left quite
alone.

"Don't you want to go with them?" said Aunt Helena--old Miss
Parry!--she had guessed.  And he turned round and there was
Clarissa again.  She had come back to fetch him.  He was overcome
by her generosity--her goodness.

"Come along," she said.  "They're waiting."  He had never felt so
happy in the whole of his life!  Without a word they made it up.
They walked down to the lake.  He had twenty minutes of perfect
happiness.  Her voice, her laugh, her dress (something floating,
white, crimson), her spirit, her adventurousness; she made them all
disembark and explore the island; she startled a hen; she laughed;
she sang.  And all the time, he knew perfectly well, Dalloway was
falling in love with her; she was falling in love with Dalloway;
but it didn't seem to matter.  Nothing mattered.  They sat on the
ground and talked--he and Clarissa.  They went in and out of each
other's minds without any effort.  And then in a second it was
over.  He said to himself as they were getting into the boat, "She
will marry that man," dully, without any resentment; but it was an
obvious thing.  Dalloway would marry Clarissa.

Dalloway rowed them in.  He said nothing.  But somehow as they
watched him start, jumping on to his bicycle to ride twenty miles
through the woods, wobbling off down the drive, waving his hand and
disappearing, he obviously did feel, instinctively, tremendously,
strongly, all that; the night; the romance; Clarissa.  He deserved
to have her.

For himself, he was absurd.  His demands upon Clarissa (he could
see it now) were absurd.  He asked impossible things.  He made
terrible scenes.  She would have accepted him still, perhaps, if he
had been less absurd.  Sally thought so.  She wrote him all that
summer long letters; how they had talked of him; how she had
praised him, how Clarissa burst into tears!  It was an extraordinary
summer--all letters, scenes, telegrams--arriving at Bourton early in
the morning, hanging about till the servants were up; appalling
tęte-ŕ-tętes with old Mr. Parry at breakfast; Aunt Helena formidable
but kind; Sally sweeping him off for talks in the vegetable garden;
Clarissa in bed with headaches.

The final scene, the terrible scene which he believed had mattered
more than anything in the whole of his life (it might be an
exaggeration--but still so it did seem now) happened at three
o'clock in the afternoon of a very hot day.  It was a trifle that
led up to it--Sally at lunch saying something about Dalloway, and
calling him "My name is Dalloway"; whereupon Clarissa suddenly
stiffened, coloured, in a way she had, and rapped out sharply,
"We've had enough of that feeble joke."  That was all; but for him
it was precisely as if she had said, "I'm only amusing myself with
you; I've an understanding with Richard Dalloway."  So he took it.
He had not slept for nights.  "It's got to be finished one way or
the other," he said to himself.  He sent a note to her by Sally
asking her to meet him by the fountain at three.  "Something very
important has happened," he scribbled at the end of it.

The fountain was in the middle of a little shrubbery, far from the
house, with shrubs and trees all round it.  There she came, even
before the time, and they stood with the fountain between them, the
spout (it was broken) dribbling water incessantly.  How sights fix
themselves upon the mind!  For example, the vivid green moss.

She did not move.  "Tell me the truth, tell me the truth," he kept
on saying.  He felt as if his forehead would burst.  She seemed
contracted, petrified.  She did not move.  "Tell me the truth," he
repeated, when suddenly that old man Breitkopf popped his head in
carrying the Times; stared at them; gaped; and went away.  They
neither of them moved.  "Tell me the truth," he repeated.  He felt
that he was grinding against something physically hard; she was
unyielding.  She was like iron, like flint, rigid up the backbone.
And when she said, "It's no use.  It's no use.  This is the end"--
after he had spoken for hours, it seemed, with the tears running
down his cheeks--it was as if she had hit him in the face.  She
turned, she left him, went away.

"Clarissa!" he cried.  "Clarissa!"  But she never came back.  It
was over.  He went away that night.  He never saw her again.



It was awful, he cried, awful, awful!

Still, the sun was hot.  Still, one got over things.  Still, life
had a way of adding day to day.  Still, he thought, yawning and
beginning to take notice--Regent's Park had changed very little
since he was a boy, except for the squirrels--still, presumably
there were compensations--when little Elise Mitchell, who had been
picking up pebbles to add to the pebble collection which she and
her brother were making on the nursery mantelpiece, plumped her
handful down on the nurse's knee and scudded off again full tilt
into a lady's legs.  Peter Walsh laughed out.

But Lucrezia Warren Smith was saying to herself, It's wicked; why
should I suffer? she was asking, as she walked down the broad path.
No; I can't stand it any longer, she was saying, having left
Septimus, who wasn't Septimus any longer, to say hard, cruel,
wicked things, to talk to himself, to talk to a dead man, on the
seat over there; when the child ran full tilt into her, fell flat,
and burst out crying.

That was comforting rather.  She stood her upright, dusted her
frock, kissed her.

But for herself she had done nothing wrong; she had loved Septimus;
she had been happy; she had had a beautiful home, and there her
sisters lived still, making hats.  Why should SHE suffer?

The child ran straight back to its nurse, and Rezia saw her
scolded, comforted, taken up by the nurse who put down her
knitting, and the kind-looking man gave her his watch to blow open
to comfort her--but why should SHE be exposed?  Why not left in
Milan?  Why tortured?  Why?

Slightly waved by tears the broad path, the nurse, the man in grey,
the perambulator, rose and fell before her eyes.  To be rocked by
this malignant torturer was her lot.  But why?  She was like a bird
sheltering under the thin hollow of a leaf, who blinks at the sun
when the leaf moves; starts at the crack of a dry twig.  She was
exposed; she was surrounded by the enormous trees, vast clouds of
an indifferent world, exposed; tortured; and why should she suffer?
Why?

She frowned; she stamped her foot.  She must go back again to
Septimus since it was almost time for them to be going to Sir
William Bradshaw.  She must go back and tell him, go back to him
sitting there on the green chair under the tree, talking to
himself, or to that dead man Evans, whom she had only seen once for
a moment in the shop.  He had seemed a nice quiet man; a great
friend of Septimus's, and he had been killed in the War.  But such
things happen to every one.  Every one has friends who were killed
in the War.  Every one gives up something when they marry.  She had
given up her home.  She had come to live here, in this awful city.
But Septimus let himself think about horrible things, as she could
too, if she tried.  He had grown stranger and stranger.  He said
people were talking behind the bedroom walls.  Mrs. Filmer thought
it odd.  He saw things too--he had seen an old woman's head in the
middle of a fern.  Yet he could be happy when he chose.  They went
to Hampton Court on top of a bus, and they were perfectly happy.
All the little red and yellow flowers were out on the grass, like
floating lamps he said, and talked and chattered and laughed,
making up stories.  Suddenly he said, "Now we will kill ourselves,"
when they were standing by the river, and he looked at it with a
look which she had seen in his eyes when a train went by, or an
omnibus--a look as if something fascinated him; and she felt he was
going from her and she caught him by the arm.  But going home he
was perfectly quiet--perfectly reasonable.  He would argue with her
about killing themselves; and explain how wicked people were; how
he could see them making up lies as they passed in the street.  He
knew all their thoughts, he said; he knew everything.  He knew the
meaning of the world, he said.

Then when they got back he could hardly walk.  He lay on the sofa
and made her hold his hand to prevent him from falling down, down,
he cried, into the flames! and saw faces laughing at him, calling
him horrible disgusting names, from the walls, and hands pointing
round the screen.  Yet they were quite alone.  But he began to talk
aloud, answering people, arguing, laughing, crying, getting very
excited and making her write things down.  Perfect nonsense it was;
about death; about Miss Isabel Pole.  She could stand it no longer.
She would go back.

She was close to him now, could see him staring at the sky,
muttering, clasping his hands.  Yet Dr. Holmes said there was
nothing the matter with him.  What then had happened--why had he
gone, then, why, when she sat by him, did he start, frown at her,
move away, and point at her hand, take her hand, look at it
terrified?

Was it that she had taken off her wedding ring?  "My hand has grown
so thin," she said.  "I have put it in my purse," she told him.

He dropped her hand.  Their marriage was over, he thought, with
agony, with relief.  The rope was cut; he mounted; he was free, as
it was decreed that he, Septimus, the lord of men, should be free;
alone (since his wife had thrown away her wedding ring; since she
had left him), he, Septimus, was alone, called forth in advance of
the mass of men to hear the truth, to learn the meaning, which now
at last, after all the toils of civilisation--Greeks, Romans,
Shakespeare, Darwin, and now himself--was to be given whole to. . . .
"To whom?" he asked aloud.  "To the Prime Minister," the voices
which rustled above his head replied.  The supreme secret must be
told to the Cabinet; first that trees are alive; next there is no
crime; next love, universal love, he muttered, gasping, trembling,
painfully drawing out these profound truths which needed, so deep
were they, so difficult, an immense effort to speak out, but the
world was entirely changed by them for ever.

No crime; love; he repeated, fumbling for his card and pencil, when
a Skye terrier snuffed his trousers and he started in an agony of
fear.  It was turning into a man!  He could not watch it happen!
It was horrible, terrible to see a dog become a man!  At once the
dog trotted away.

Heaven was divinely merciful, infinitely benignant.  It spared him,
pardoned his weakness.  But what was the scientific explanation
(for one must be scientific above all things)?  Why could he see
through bodies, see into the future, when dogs will become men?  It
was the heat wave presumably, operating upon a brain made sensitive
by eons of evolution.  Scientifically speaking, the flesh was
melted off the world.  His body was macerated until only the nerve
fibres were left.  It was spread like a veil upon a rock.

He lay back in his chair, exhausted but upheld.  He lay resting,
waiting, before he again interpreted, with effort, with agony, to
mankind.  He lay very high, on the back of the world.  The earth
thrilled beneath him.  Red flowers grew through his flesh; their
stiff leaves rustled by his head.  Music began clanging against the
rocks up here.  It is a motor horn down in the street, he muttered;
but up here it cannoned from rock to rock, divided, met in shocks
of sound which rose in smooth columns (that music should be visible
was a discovery) and became an anthem, an anthem twined round now
by a shepherd boy's piping (That's an old man playing a penny
whistle by the public-house, he muttered) which, as the boy stood
still came bubbling from his pipe, and then, as he climbed higher,
made its exquisite plaint while the traffic passed beneath.  This
boy's elegy is played among the traffic, thought Septimus.  Now he
withdraws up into the snows, and roses hang about him--the thick
red roses which grow on my bedroom wall, he reminded himself.  The
music stopped.  He has his penny, he reasoned it out, and has gone
on to the next public-house.

But he himself remained high on his rock, like a drowned sailor on
a rock.  I leant over the edge of the boat and fell down, he
thought.  I went under the sea.  I have been dead, and yet am now
alive, but let me rest still; he begged (he was talking to himself
again--it was awful, awful!); and as, before waking, the voices of
birds and the sound of wheels chime and chatter in a queer harmony,
grow louder and louder and the sleeper feels himself drawing to the
shores of life, so he felt himself drawing towards life, the sun
growing hotter, cries sounding louder, something tremendous about
to happen.

He had only to open his eyes; but a weight was on them; a fear.  He
strained; he pushed; he looked; he saw Regent's Park before him.
Long streamers of sunlight fawned at his feet.  The trees waved,
brandished.  We welcome, the world seemed to say; we accept; we
create.  Beauty, the world seemed to say.  And as if to prove it
(scientifically) wherever he looked at the houses, at the railings,
at the antelopes stretching over the palings, beauty sprang
instantly.  To watch a leaf quivering in the rush of air was an
exquisite joy.  Up in the sky swallows swooping, swerving, flinging
themselves in and out, round and round, yet always with perfect
control as if elastics held them; and the flies rising and falling;
and the sun spotting now this leaf, now that, in mockery, dazzling
it with soft gold in pure good temper; and now and again some chime
(it might be a motor horn) tinkling divinely on the grass stalks--
all of this, calm and reasonable as it was, made out of ordinary
things as it was, was the truth now; beauty, that was the truth
now.  Beauty was everywhere.

"It is time," said Rezia.

The word "time" split its husk; poured its riches over him; and
from his lips fell like shells, like shavings from a plane, without
his making them, hard, white, imperishable words, and flew to
attach themselves to their places in an ode to Time; an immortal
ode to Time.  He sang.  Evans answered from behind the tree.  The
dead were in Thessaly, Evans sang, among the orchids.  There they
waited till the War was over, and now the dead, now Evans himself--

"For God's sake don't come!" Septimus cried out.  For he could not
look upon the dead.

But the branches parted.  A man in grey was actually walking
towards them.  It was Evans!  But no mud was on him; no wounds; he
was not changed.  I must tell the whole world, Septimus cried,
raising his hand (as the dead man in the grey suit came nearer),
raising his hand like some colossal figure who has lamented the
fate of man for ages in the desert alone with his hands pressed to
his forehead, furrows of despair on his cheeks, and now sees light
on the desert's edge which broadens and strikes the iron-black
figure (and Septimus half rose from his chair), and with legions of
men prostrate behind him he, the giant mourner, receives for one
moment on his face the whole--

"But I am so unhappy, Septimus," said Rezia trying to make him sit
down.

The millions lamented; for ages they had sorrowed.  He would turn
round, he would tell them in a few moments, only a few moments
more, of this relief, of this joy, of this astonishing revelation--

"The time, Septimus," Rezia repeated.  "What is the time?"

He was talking, he was starting, this man must notice him.  He was
looking at them.

"I will tell you the time," said Septimus, very slowly, very
drowsily, smiling mysteriously.  As he sat smiling at the dead man
in the grey suit the quarter struck--the quarter to twelve.

And that is being young, Peter Walsh thought as he passed them.
To be having an awful scene--the poor girl looked absolutely
desperate--in the middle of the morning.  But what was it about, he
wondered, what had the young man in the overcoat been saying to her
to make her look like that; what awful fix had they got themselves
into, both to look so desperate as that on a fine summer morning?
The amusing thing about coming back to England, after five years,
was the way it made, anyhow the first days, things stand out as if
one had never seen them before; lovers squabbling under a tree; the
domestic family life of the parks.  Never had he seen London look
so enchanting--the softness of the distances; the richness; the
greenness; the civilisation, after India, he thought, strolling
across the grass.

This susceptibility to impressions had been his undoing no doubt.
Still at his age he had, like a boy or a girl even, these
alternations of mood; good days, bad days, for no reason whatever,
happiness from a pretty face, downright misery at the sight of a
frump.  After India of course one fell in love with every woman one
met.  There was a freshness about them; even the poorest dressed
better than five years ago surely; and to his eye the fashions had
never been so becoming; the long black cloaks; the slimness; the
elegance; and then the delicious and apparently universal habit of
paint.  Every woman, even the most respectable, had roses blooming
under glass; lips cut with a knife; curls of Indian ink; there was
design, art, everywhere; a change of some sort had undoubtedly
taken place.  What did the young people think about? Peter Walsh
asked himself.

Those five years--1918 to 1923--had been, he suspected, somehow
very important.  People looked different.  Newspapers seemed
different.  Now for instance there was a man writing quite openly
in one of the respectable weeklies about water-closets.  That you
couldn't have done ten years ago--written quite openly about water-
closets in a respectable weekly.  And then this taking out a stick
of rouge, or a powder-puff and making up in public.  On board ship
coming home there were lots of young men and girls--Betty and
Bertie he remembered in particular--carrying on quite openly; the
old mother sitting and watching them with her knitting, cool as a
cucumber.  The girl would stand still and powder her nose in front
of every one.  And they weren't engaged; just having a good time;
no feelings hurt on either side.  As hard as nails she was--Betty
What'shername--; but a thorough good sort.  She would make a very
good wife at thirty--she would marry when it suited her to marry;
marry some rich man and live in a large house near Manchester.

Who was it now who had done that?  Peter Walsh asked himself,
turning into the Broad Walk,--married a rich man and lived in a
large house near Manchester?  Somebody who had written him a long,
gushing letter quite lately about "blue hydrangeas."  It was seeing
blue hydrangeas that made her think of him and the old days--Sally
Seton, of course!  It was Sally Seton--the last person in the world
one would have expected to marry a rich man and live in a large
house near Manchester, the wild, the daring, the romantic Sally!

But of all that ancient lot, Clarissa's friends--Whitbreads,
Kinderleys, Cunninghams, Kinloch-Jones's--Sally was probably the
best.  She tried to get hold of things by the right end anyhow.
She saw through Hugh Whitbread anyhow--the admirable Hugh--when
Clarissa and the rest were at his feet.

"The Whitbreads?" he could hear her saying.  "Who are the
Whitbreads?  Coal merchants.  Respectable tradespeople."

Hugh she detested for some reason.  He thought of nothing but his
own appearance, she said.  He ought to have been a Duke.  He would
be certain to marry one of the Royal Princesses.  And of course
Hugh had the most extraordinary, the most natural, the most sublime
respect for the British aristocracy of any human being he had ever
come across.  Even Clarissa had to own that.  Oh, but he was such a
dear, so unselfish, gave up shooting to please his old mother--
remembered his aunts' birthdays, and so on.

Sally, to do her justice, saw through all that.  One of the things
he remembered best was an argument one Sunday morning at Bourton
about women's rights (that antediluvian topic), when Sally suddenly
lost her temper, flared up, and told Hugh that he represented all
that was most detestable in British middle-class life.  She told
him that she considered him responsible for the state of "those
poor girls in Piccadilly"--Hugh, the perfect gentleman, poor Hugh!--
never did a man look more horrified!  She did it on purpose she
said afterwards (for they used to get together in the vegetable
garden and compare notes).  "He's read nothing, thought nothing,
felt nothing," he could hear her saying in that very emphatic voice
which carried so much farther than she knew.  The stable boys had
more life in them than Hugh, she said.  He was a perfect specimen
of the public school type, she said.  No country but England could
have produced him.  She was really spiteful, for some reason; had
some grudge against him.  Something had happened--he forgot what--
in the smoking-room.  He had insulted her--kissed her?  Incredible!
Nobody believed a word against Hugh of course.  Who could?  Kissing
Sally in the smoking-room!  If it had been some Honourable Edith or
Lady Violet, perhaps; but not that ragamuffin Sally without a penny
to her name, and a father or a mother gambling at Monte Carlo.  For
of all the people he had ever met Hugh was the greatest snob--the
most obsequious--no, he didn't cringe exactly.  He was too much of
a prig for that.  A first-rate valet was the obvious comparison--
somebody who walked behind carrying suit cases; could be trusted to
send telegrams--indispensable to hostesses.  And he'd found his
job--married his Honourable Evelyn; got some little post at Court,
looked after the King's cellars, polished the Imperial shoe-
buckles, went about in knee-breeches and lace ruffles.  How
remorseless life is!  A little job at Court!

He had married this lady, the Honourable Evelyn, and they lived
hereabouts, so he thought (looking at the pompous houses
overlooking the Park), for he had lunched there once in a house
which had, like all Hugh's possessions, something that no other
house could possibly have--linen cupboards it might have been.  You
had to go and look at them--you had to spend a great deal of time
always admiring whatever it was--linen cupboards, pillow-cases, old
oak furniture, pictures, which Hugh had picked up for an old song.
But Mrs. Hugh sometimes gave the show away.  She was one of those
obscure mouse-like little women who admire big men.  She was
almost negligible.  Then suddenly she would say something quite
unexpected--something sharp.  She had the relics of the grand
manner perhaps.  The steam coal was a little too strong for her--it
made the atmosphere thick.  And so there they lived, with their
linen cupboards and their old masters and their pillow-cases
fringed with real lace at the rate of five or ten thousand a year
presumably, while he, who was two years older than Hugh, cadged for
a job.

At fifty-three he had to come and ask them to put him into some
secretary's office, to find him some usher's job teaching little
boys Latin, at the beck and call of some mandarin in an office,
something that brought in five hundred a year; for if he married
Daisy, even with his pension, they could never do on less.
Whitbread could do it presumably; or Dalloway.  He didn't mind what
he asked Dalloway.  He was a thorough good sort; a bit limited; a
bit thick in the head; yes; but a thorough good sort.  Whatever he
took up he did in the same matter-of-fact sensible way; without a
touch of imagination, without a spark of brilliancy, but with the
inexplicable niceness of his type.  He ought to have been a country
gentleman--he was wasted on politics.  He was at his best out of
doors, with horses and dogs--how good he was, for instance, when
that great shaggy dog of Clarissa's got caught in a trap and had
its paw half torn off, and Clarissa turned faint and Dalloway did
the whole thing; bandaged, made splints; told Clarissa not to be a
fool.  That was what she liked him for perhaps--that was what she
needed.  "Now, my dear, don't be a fool.  Hold this--fetch that,"
all the time talking to the dog as if it were a human being.

But how could she swallow all that stuff about poetry?  How could
she let him hold forth about Shakespeare?  Seriously and solemnly
Richard Dalloway got on his hind legs and said that no decent man
ought to read Shakespeare's sonnets because it was like listening
at keyholes (besides the relationship was not one that he
approved).  No decent man ought to let his wife visit a deceased
wife's sister.  Incredible!  The only thing to do was to pelt him
with sugared almonds--it was at dinner.  But Clarissa sucked it all
in; thought it so honest of him; so independent of him; Heaven
knows if she didn't think him the most original mind she'd ever
met!

That was one of the bonds between Sally and himself.  There was a
garden where they used to walk, a walled-in place, with rose-bushes
and giant cauliflowers--he could remember Sally tearing off a rose,
stopping to exclaim at the beauty of the cabbage leaves in the
moonlight (it was extraordinary how vividly it all came back to
him, things he hadn't thought of for years,) while she implored
him, half laughing of course, to carry off Clarissa, to save her
from the Hughs and the Dalloways and all the other "perfect
gentlemen" who would "stifle her soul" (she wrote reams of poetry
in those days), make a mere hostess of her, encourage her
worldliness.  But one must do Clarissa justice.  She wasn't going
to marry Hugh anyhow.  She had a perfectly clear notion of what she
wanted.  Her emotions were all on the surface.  Beneath, she was
very shrewd--a far better judge of character than Sally, for
instance, and with it all, purely feminine; with that extraordinary
gift, that woman's gift, of making a world of her own wherever she
happened to be.  She came into a room; she stood, as he had often
seen her, in a doorway with lots of people round her.  But it was
Clarissa one remembered.  Not that she was striking; not beautiful
at all; there was nothing picturesque about her; she never said
anything specially clever; there she was, however; there she was.

No, no, no!  He was not in love with her any more!  He only felt,
after seeing her that morning, among her scissors and silks, making
ready for the party, unable to get away from the thought of her;
she kept coming back and back like a sleeper jolting against him in
a railway carriage; which was not being in love, of course; it was
thinking of her, criticising her, starting again, after thirty
years, trying to explain her.  The obvious thing to say of her was
that she was worldly; cared too much for rank and society and
getting on in the world--which was true in a sense; she had
admitted it to him.  (You could always get her to own up if you
took the trouble; she was honest.)  What she would say was that she
hated frumps, fogies, failures, like himself presumably; thought
people had no right to slouch about with their hands in their
pockets; must do something, be something; and these great swells,
these Duchesses, these hoary old Countesses one met in her drawing-
room, unspeakably remote as he felt them to be from anything that
mattered a straw, stood for something real to her.  Lady
Bexborough, she said once, held herself upright (so did Clarissa
herself; she never lounged in any sense of the word; she was
straight as a dart, a little rigid in fact).  She said they had a
kind of courage which the older she grew the more she respected.
In all this there was a great deal of Dalloway, of course; a great
deal of the public-spirited, British Empire, tariff-reform,
governing-class spirit, which had grown on her, as it tends to do.
With twice his wits, she had to see things through his eyes--one of
the tragedies of married life.  With a mind of her own, she must
always be quoting Richard--as if one couldn't know to a tittle what
Richard thought by reading the Morning Post of a morning!  These
parties for example were all for him, or for her idea of him (to do
Richard justice he would have been happier farming in Norfolk).
She made her drawing-room a sort of meeting-place; she had a genius
for it.  Over and over again he had seen her take some raw youth,
twist him, turn him, wake him up; set him going.  Infinite numbers
of dull people conglomerated round her of course.  But odd
unexpected people turned up; an artist sometimes; sometimes a
writer; queer fish in that atmosphere.  And behind it all was that
network of visiting, leaving cards, being kind to people; running
about with bunches of flowers, little presents; So-and-so was going
to France--must have an air-cushion; a real drain on her strength;
all that interminable traffic that women of her sort keep up; but
she did it genuinely, from a natural instinct.

Oddly enough, she was one of the most thoroughgoing sceptics he had
ever met, and possibly (this was a theory he used to make up to
account for her, so transparent in some ways, so inscrutable in
others), possibly she said to herself, As we are a doomed race,
chained to a sinking ship (her favourite reading as a girl was
Huxley and Tyndall, and they were fond of these nautical
metaphors), as the whole thing is a bad joke, let us, at any rate,
do our part; mitigate the sufferings of our fellow-prisoners
(Huxley again); decorate the dungeon with flowers and air-cushions;
be as decent as we possibly can.  Those ruffians, the Gods, shan't
have it all their own way,--her notion being that the Gods, who
never lost a chance of hurting, thwarting and spoiling human lives
were seriously put out if, all the same, you behaved like a lady.
That phase came directly after Sylvia's death--that horrible
affair.  To see your own sister killed by a falling tree (all
Justin Parry's fault--all his carelessness) before your very eyes,
a girl too on the verge of life, the most gifted of them, Clarissa
always said, was enough to turn one bitter.  Later she wasn't so
positive perhaps; she thought there were no Gods; no one was to
blame; and so she evolved this atheist's religion of doing good for
the sake of goodness.

And of course she enjoyed life immensely.  It was her nature to
enjoy (though goodness only knows, she had her reserves; it was a
mere sketch, he often felt, that even he, after all these years,
could make of Clarissa).  Anyhow there was no bitterness in her;
none of that sense of moral virtue which is so repulsive in good
women.  She enjoyed practically everything.  If you walked with her
in Hyde Park now it was a bed of tulips, now a child in a
perambulator, now some absurd little drama she made up on the spur
of the moment.  (Very likely, she would have talked to those
lovers, if she had thought them unhappy.)  She had a sense of
comedy that was really exquisite, but she needed people, always
people, to bring it out, with the inevitable result that she
frittered her time away, lunching, dining, giving these incessant
parties of hers, talking nonsense, sayings things she didn't mean,
blunting the edge of her mind, losing her discrimination.  There
she would sit at the head of the table taking infinite pains with
some old buffer who might be useful to Dalloway--they knew the most
appalling bores in Europe--or in came Elizabeth and everything must
give way to HER.  She was at a High School, at the inarticulate
stage last time he was over, a round-eyed, pale-faced girl, with
nothing of her mother in her, a silent stolid creature, who took it
all as a matter of course, let her mother make a fuss of her, and
then said "May I go now?" like a child of four; going off, Clarissa
explained, with that mixture of amusement and pride which Dalloway
himself seemed to rouse in her, to play hockey.  And now Elizabeth
was "out," presumably; thought him an old fogy, laughed at her
mother's friends.  Ah well, so be it.  The compensation of growing
old, Peter Walsh thought, coming out of Regent's Park, and holding
his hat in hand, was simply this; that the passions remain as
strong as ever, but one has gained--at last!--the power which adds
the supreme flavour to existence,--the power of taking hold of
experience, of turning it round, slowly, in the light.

A terrible confession it was (he put his hat on again), but now, at
the age of fifty-three one scarcely needed people any more.  Life
itself, every moment of it, every drop of it, here, this instant,
now, in the sun, in Regent's Park, was enough.  Too much indeed.  A
whole lifetime was too short to bring out, now that one had
acquired the power, the full flavour; to extract every ounce of
pleasure, every shade of meaning; which both were so much more
solid than they used to be, so much less personal.  It was
impossible that he should ever suffer again as Clarissa had made
him suffer.  For hours at a time (pray God that one might say these
things without being overheard!), for hours and days he never
thought of Daisy.

Could it be that he was in love with her then, remembering the
misery, the torture, the extraordinary passion of those days?  It
was a different thing altogether--a much pleasanter thing--the
truth being, of course, that now SHE was in love with HIM.  And
that perhaps was the reason why, when the ship actually sailed, he
felt an extraordinary relief, wanted nothing so much as to be
alone; was annoyed to find all her little attentions--cigars,
notes, a rug for the voyage--in his cabin.  Every one if they were
honest would say the same; one doesn't want people after fifty; one
doesn't want to go on telling women they are pretty; that's what
most men of fifty would say, Peter Walsh thought, if they were
honest.

But then these astonishing accesses of emotion--bursting into tears
this morning, what was all that about?  What could Clarissa have
thought of him? thought him a fool presumably, not for the first
time.  It was jealousy that was at the bottom of it--jealousy which
survives every other passion of mankind, Peter Walsh thought,
holding his pocket-knife at arm's length.  She had been meeting
Major Orde, Daisy said in her last letter; said it on purpose he
knew; said it to make him jealous; he could see her wrinkling her
forehead as she wrote, wondering what she could say to hurt him;
and yet it made no difference; he was furious!  All this pother of
coming to England and seeing lawyers wasn't to marry her, but to
prevent her from marrying anybody else.  That was what tortured
him, that was what came over him when he saw Clarissa so calm, so
cold, so intent on her dress or whatever it was; realising what she
might have spared him, what she had reduced him to--a whimpering,
snivelling old ass.  But women, he thought, shutting his pocket-
knife, don't know what passion is.  They don't know the meaning of
it to men.  Clarissa was as cold as an icicle.  There she would sit
on the sofa by his side, let him take her hand, give him one kiss--
Here he was at the crossing.

A sound interrupted him; a frail quivering sound, a voice bubbling
up without direction, vigour, beginning or end, running weakly and
shrilly and with an absence of all human meaning into


     ee um fah um so
     foo swee too eem oo--


the voice of no age or sex, the voice of an ancient spring spouting
from the earth; which issued, just opposite Regent's Park Tube
station from a tall quivering shape, like a funnel, like a rusty
pump, like a wind-beaten tree for ever barren of leaves which lets
the wind run up and down its branches singing


     ee um fah um so
     foo swee too eem oo


and rocks and creaks and moans in the eternal breeze.

Through all ages--when the pavement was grass, when it was swamp,
through the age of tusk and mammoth, through the age of silent
sunrise, the battered woman--for she wore a skirt--with her right
hand exposed, her left clutching at her side, stood singing of
love--love which has lasted a million years, she sang, love which
prevails, and millions of years ago, her lover, who had been dead
these centuries, had walked, she crooned, with her in May; but in
the course of ages, long as summer days, and flaming, she
remembered, with nothing but red asters, he had gone; death's
enormous sickle had swept those tremendous hills, and when at last
she laid her hoary and immensely aged head on the earth, now become
a mere cinder of ice, she implored the Gods to lay by her side a
bunch of purple-heather, there on her high burial place which the
last rays of the last sun caressed; for then the pageant of the
universe would be over.

As the ancient song bubbled up opposite Regent's Park Tube station
still the earth seemed green and flowery; still, though it issued
from so rude a mouth, a mere hole in the earth, muddy too, matted
with root fibres and tangled grasses, still the old bubbling
burbling song, soaking through the knotted roots of infinite ages,
and skeletons and treasure, streamed away in rivulets over the
pavement and all along the Marylebone Road, and down towards
Euston, fertilising, leaving a damp stain.

Still remembering how once in some primeval May she had walked with
her lover, this rusty pump, this battered old woman with one hand
exposed for coppers the other clutching her side, would still be
there in ten million years, remembering how once she had walked in
May, where the sea flows now, with whom it did not matter--he was a
man, oh yes, a man who had loved her.  But the passage of ages had
blurred the clarity of that ancient May day; the bright petalled
flowers were hoar and silver frosted; and she no longer saw, when
she implored him (as she did now quite clearly) "look in my eyes
with thy sweet eyes intently," she no longer saw brown eyes, black
whiskers or sunburnt face but only a looming shape, a shadow shape,
to which, with the bird-like freshness of the very aged she still
twittered "give me your hand and let me press it gently" (Peter
Walsh couldn't help giving the poor creature a coin as he stepped
into his taxi), "and if some one should see, what matter they?" she
demanded; and her fist clutched at her side, and she smiled,
pocketing her shilling, and all peering inquisitive eyes seemed
blotted out, and the passing generations--the pavement was crowded
with bustling middle-class people--vanished, like leaves, to be
trodden under, to be soaked and steeped and made mould of by that
eternal spring--


     ee um fah um so
     foo swee too eem oo


"Poor old woman," said Rezia Warren Smith, waiting to cross.

Oh poor old wretch!

Suppose it was a wet night?  Suppose one's father, or somebody who
had known one in better days had happened to pass, and saw one
standing there in the gutter?  And where did she sleep at night?

Cheerfully, almost gaily, the invincible thread of sound wound up
into the air like the smoke from a cottage chimney, winding up
clean beech trees and issuing in a tuft of blue smoke among the
topmost leaves.  "And if some one should see, what matter they?"

Since she was so unhappy, for weeks and weeks now, Rezia had given
meanings to things that happened, almost felt sometimes that she
must stop people in the street, if they looked good, kind people,
just to say to them "I am unhappy"; and this old woman singing in
the street "if some one should see, what matter they?" made her
suddenly quite sure that everything was going to be right.  They
were going to Sir William Bradshaw; she thought his name sounded
nice; he would cure Septimus at once.  And then there was a
brewer's cart, and the grey horses had upright bristles of straw in
their tails; there were newspaper placards.  It was a silly, silly
dream, being unhappy.

So they crossed, Mr. and Mrs. Septimus Warren Smith, and was there,
after all, anything to draw attention to them, anything to make a
passer-by suspect here is a young man who carries in him the
greatest message in the world, and is, moreover, the happiest man
in the world, and the most miserable?  Perhaps they walked more
slowly than other people, and there was something hesitating,
trailing, in the man's walk, but what more natural for a clerk, who
has not been in the West End on a weekday at this hour for years,
than to keep looking at the sky, looking at this, that and the
other, as if Portland Place were a room he had come into when the
family are away, the chandeliers being hung in holland bags, and
the caretaker, as she lets in long shafts of dusty light upon
deserted, queer-looking armchairs, lifting one corner of the long
blinds, explains to the visitors what a wonderful place it is; how
wonderful, but at the same time, he thinks, as he looks at chairs
and tables, how strange.

To look at, he might have been a clerk, but of the better sort; for
he wore brown boots; his hands were educated; so, too, his profile--
his angular, big-nosed, intelligent, sensitive profile; but not
his lips altogether, for they were loose; and his eyes (as eyes
tend to be), eyes merely; hazel, large; so that he was, on the
whole, a border case, neither one thing nor the other, might end
with a house at Purley and a motor car, or continue renting
apartments in back streets all his life; one of those half-
educated, self-educated men whose education is all learnt from
books borrowed from public libraries, read in the evening after the
day's work, on the advice of well-known authors consulted by
letter.

As for the other experiences, the solitary ones, which people go
through alone, in their bedrooms, in their offices, walking the
fields and the streets of London, he had them; had left home, a
mere boy, because of his mother; she lied; because he came down to
tea for the fiftieth time with his hands unwashed; because he could
see no future for a poet in Stroud; and so, making a confidant of
his little sister, had gone to London leaving an absurd note behind
him, such as great men have written, and the world has read later
when the story of their struggles has become famous.

London has swallowed up many millions of young men called Smith;
thought nothing of fantastic Christian names like Septimus with
which their parents have thought to distinguish them.  Lodging off
the Euston Road, there were experiences, again experiences, such as
change a face in two years from a pink innocent oval to a face
lean, contracted, hostile.  But of all this what could the most
observant of friends have said except what a gardener says when he
opens the conservatory door in the morning and finds a new blossom
on his plant:--It has flowered; flowered from vanity, ambition,
idealism, passion, loneliness, courage, laziness, the usual seeds,
which all muddled up (in a room off the Euston Road), made him shy,
and stammering, made him anxious to improve himself, made him fall
in love with Miss Isabel Pole, lecturing in the Waterloo Road upon
Shakespeare.

Was he not like Keats? she asked; and reflected how she might give
him a taste of Antony and Cleopatra and the rest; lent him books;
wrote him scraps of letters; and lit in him such a fire as burns
only once in a lifetime, without heat, flickering a red gold flame
infinitely ethereal and insubstantial over Miss Pole; Antony and
Cleopatra; and the Waterloo Road.  He thought her beautiful,
believed her impeccably wise; dreamed of her, wrote poems to her,
which, ignoring the subject, she corrected in red ink; he saw her,
one summer evening, walking in a green dress in a square.  "It has
flowered," the gardener might have said, had he opened the door;
had he come in, that is to say, any night about this time, and
found him writing; found him tearing up his writing; found him
finishing a masterpiece at three o'clock in the morning and running
out to pace the streets, and visiting churches, and fasting one
day, drinking another, devouring Shakespeare, Darwin, The History
of Civilisation, and Bernard Shaw.

Something was up, Mr. Brewer knew; Mr. Brewer, managing clerk at
Sibleys and Arrowsmiths, auctioneers, valuers, land and estate
agents; something was up, he thought, and, being paternal with his
young men, and thinking very highly of Smith's abilities, and
prophesying that he would, in ten or fifteen years, succeed to the
leather arm-chair in the inner room under the skylight with the
deed-boxes round him, "if he keeps his health," said Mr. Brewer,
and that was the danger--he looked weakly; advised football,
invited him to supper and was seeing his way to consider
recommending a rise of salary, when something happened which threw
out many of Mr. Brewer's calculations, took away his ablest young
fellows, and eventually, so prying and insidious were the fingers
of the European War, smashed a plaster cast of Ceres, ploughed a
hole in the geranium beds, and utterly ruined the cook's nerves at
Mr. Brewer's establishment at Muswell Hill.

Septimus was one of the first to volunteer.  He went to France to
save an England which consisted almost entirely of Shakespeare's
plays and Miss Isabel Pole in a green dress walking in a square.
There in the trenches the change which Mr. Brewer desired when he
advised football was produced instantly; he developed manliness; he
was promoted; he drew the attention, indeed the affection of his
officer, Evans by name.  It was a case of two dogs playing on a
hearth-rug; one worrying a paper screw, snarling, snapping, giving
a pinch, now and then, at the old dog's ear; the other lying
somnolent, blinking at the fire, raising a paw, turning and
growling good-temperedly.  They had to be together, share with each
other, fight with each other, quarrel with each other.  But when
Evans (Rezia who had only seen him once called him "a quiet man," a
sturdy red-haired man, undemonstrative in the company of women),
when Evans was killed, just before the Armistice, in Italy,
Septimus, far from showing any emotion or recognising that here was
the end of a friendship, congratulated himself upon feeling very
little and very reasonably.  The War had taught him.  It was
sublime.  He had gone through the whole show, friendship, European
War, death, had won promotion, was still under thirty and was bound
to survive.  He was right there.  The last shells missed him.  He
watched them explode with indifference.  When peace came he was in
Milan, billeted in the house of an innkeeper with a courtyard,
flowers in tubs, little tables in the open, daughters making hats,
and to Lucrezia, the younger daughter, he became engaged one
evening when the panic was on him--that he could not feel.

For now that it was all over, truce signed, and the dead buried, he
had, especially in the evening, these sudden thunder-claps of fear.
He could not feel.  As he opened the door of the room where the
Italian girls sat making hats, he could see them; could hear them;
they were rubbing wires among coloured beads in saucers; they were
turning buckram shapes this way and that; the table was all strewn
with feathers, spangles, silks, ribbons; scissors were rapping on
the table; but something failed him; he could not feel.  Still,
scissors rapping, girls laughing, hats being made protected him; he
was assured of safety; he had a refuge.  But he could not sit there
all night.  There were moments of waking in the early morning.  The
bed was falling; he was falling.  Oh for the scissors and the
lamplight and the buckram shapes!  He asked Lucrezia to marry him,
the younger of the two, the gay, the frivolous, with those little
artist's fingers that she would hold up and say "It is all in
them."  Silk, feathers, what not were alive to them.

"It is the hat that matters most," she would say, when they walked
out together.  Every hat that passed, she would examine; and the
cloak and the dress and the way the woman held herself.  Ill-
dressing, over-dressing she stigmatised, not savagely, rather with
impatient movements of the hands, like those of a painter who puts
from him some obvious well-meant glaring imposture; and then,
generously, but always critically, she would welcome a shopgirl who
had turned her little bit of stuff gallantly, or praise, wholly,
with enthusiastic and professional understanding, a French lady
descending from her carriage, in chinchilla, robes, pearls.

"Beautiful!" she would murmur, nudging Septimus, that he might see.
But beauty was behind a pane of glass.  Even taste (Rezia liked
ices, chocolates, sweet things) had no relish to him.  He put down
his cup on the little marble table.  He looked at people outside;
happy they seemed, collecting in the middle of the street,
shouting, laughing, squabbling over nothing.  But he could not
taste, he could not feel.  In the tea-shop among the tables and the
chattering waiters the appalling fear came over him--he could not
feel.  He could reason; he could read, Dante for example, quite
easily ("Septimus, do put down your book," said Rezia, gently
shutting the Inferno), he could add up his bill; his brain was
perfect; it must be the fault of the world then--that he could not
feel.

"The English are so silent," Rezia said.  She liked it, she said.
She respected these Englishmen, and wanted to see London, and the
English horses, and the tailor-made suits, and could remember
hearing how wonderful the shops were, from an Aunt who had married
and lived in Soho.

It might be possible, Septimus thought, looking at England from the
train window, as they left Newhaven; it might be possible that the
world itself is without meaning.

At the office they advanced him to a post of considerable
responsibility.  They were proud of him; he had won crosses.  "You
have done your duty; it is up to us--" began Mr. Brewer; and could
not finish, so pleasurable was his emotion.  They took admirable
lodgings off the Tottenham Court Road.

Here he opened Shakespeare once more.  That boy's business of the
intoxication of language--Antony and Cleopatra--had shrivelled
utterly.  How Shakespeare loathed humanity--the putting on of
clothes, the getting of children, the sordidity of the mouth and
the belly!  This was now revealed to Septimus; the message hidden
in the beauty of words.  The secret signal which one generation
passes, under disguise, to the next is loathing, hatred, despair.
Dante the same.  Aeschylus (translated) the same.  There Rezia sat
at the table trimming hats.  She trimmed hats for Mrs. Filmer's
friends; she trimmed hats by the hour.  She looked pale,
mysterious, like a lily, drowned, under water, he thought.

"The English are so serious," she would say, putting her arms round
Septimus, her cheek against his.

Love between man and woman was repulsive to Shakespeare.  The
business of copulation was filth to him before the end.  But, Rezia
said, she must have children.  They had been married five years.

They went to the Tower together; to the Victoria and Albert Museum;
stood in the crowd to see the King open Parliament.  And there were
the shops--hat shops, dress shops, shops with leather bags in the
window, where she would stand staring.  But she must have a boy.

She must have a son like Septimus, she said.  But nobody could be
like Septimus; so gentle; so serious; so clever.  Could she not
read Shakespeare too?  Was Shakespeare a difficult author? she
asked.

One cannot bring children into a world like this.  One cannot
perpetuate suffering, or increase the breed of these lustful
animals, who have no lasting emotions, but only whims and vanities,
eddying them now this way, now that.

He watched her snip, shape, as one watches a bird hop, flit in the
grass, without daring to move a finger.  For the truth is (let her
ignore it) that human beings have neither kindness, nor faith, nor
charity beyond what serves to increase the pleasure of the moment.
They hunt in packs.  Their packs scour the desert and vanish
screaming into the wilderness.  They desert the fallen.  They are
plastered over with grimaces.  There was Brewer at the office, with
his waxed moustache, coral tie-pin, white slip, and pleasurable
emotions--all coldness and clamminess within,--his geraniums ruined
in the War--his cook's nerves destroyed; or Amelia What'shername,
handing round cups of tea punctually at five--a leering, sneering
obscene little harpy; and the Toms and Berties in their starched
shirt fronts oozing thick drops of vice.  They never saw him
drawing pictures of them naked at their antics in his notebook.  In
the street, vans roared past him; brutality blared out on placards;
men were trapped in mines; women burnt alive; and once a maimed
file of lunatics being exercised or displayed for the diversion of
the populace (who laughed aloud), ambled and nodded and grinned
past him, in the Tottenham Court Road, each half apologetically,
yet triumphantly, inflicting his hopeless woe.  And would HE go
mad?

At tea Rezia told him that Mrs. Filmer's daughter was expecting a
baby.  SHE could not grow old and have no children!  She was very
lonely, she was very unhappy!  She cried for the first time since
they were married.  Far away he heard her sobbing; he heard it
accurately, he noticed it distinctly; he compared it to a piston
thumping.  But he felt nothing.

His wife was crying, and he felt nothing; only each time she sobbed
in this profound, this silent, this hopeless way, he descended
another step into the pit.

At last, with a melodramatic gesture which he assumed mechanically
and with complete consciousness of its insincerity, he dropped his
head on his hands.  Now he had surrendered; now other people must
help him.  People must be sent for.  He gave in.

Nothing could rouse him.  Rezia put him to bed.  She sent for a
doctor--Mrs. Filmer's Dr. Holmes.  Dr. Holmes examined him.  There
was nothing whatever the matter, said Dr. Holmes.  Oh, what a
relief!  What a kind man, what a good man! thought Rezia.  When he
felt like that he went to the Music Hall, said Dr. Holmes.  He took
a day off with his wife and played golf.  Why not try two tabloids
of bromide dissolved in a glass of water at bedtime?  These old
Bloomsbury houses, said Dr. Holmes, tapping the wall, are often
full of very fine panelling, which the landlords have the folly to
paper over.  Only the other day, visiting a patient, Sir Somebody
Something in Bedford Square--

So there was no excuse; nothing whatever the matter, except the sin
for which human nature had condemned him to death; that he did not
feel.  He had not cared when Evans was killed; that was worst; but
all the other crimes raised their heads and shook their fingers and
jeered and sneered over the rail of the bed in the early hours of
the morning at the prostrate body which lay realising its
degradation; how he had married his wife without loving her; had
lied to her; seduced her; outraged Miss Isabel Pole, and was so
pocked and marked with vice that women shuddered when they saw him
in the street.  The verdict of human nature on such a wretch was
death.

Dr. Holmes came again.  Large, fresh coloured, handsome, flicking
his boots, looking in the glass, he brushed it all aside--
headaches, sleeplessness, fears, dreams--nerve symptoms and nothing
more, he said.  If Dr. Holmes found himself even half a pound below
eleven stone six, he asked his wife for another plate of porridge
at breakfast.  (Rezia would learn to cook porridge.)  But, he
continued, health is largely a matter in our own control.  Throw
yourself into outside interests; take up some hobby.  He opened
Shakespeare--Antony and Cleopatra; pushed Shakespeare aside.  Some
hobby, said Dr. Holmes, for did he not owe his own excellent health
(and he worked as hard as any man in London) to the fact that he
could always switch off from his patients on to old furniture?  And
what a very pretty comb, if he might say so, Mrs. Warren Smith was
wearing!

When the damned fool came again, Septimus refused to see him.  Did
he indeed? said Dr. Holmes, smiling agreeably.  Really he had to
give that charming little lady, Mrs. Smith, a friendly push before
he could get past her into her husband's bedroom.

"So you're in a funk," he said agreeably, sitting down by his
patient's side.  He had actually talked of killing himself to his
wife, quite a girl, a foreigner, wasn't she?  Didn't that give her
a very odd idea of English husbands?  Didn't one owe perhaps a duty
to one's wife?  Wouldn't it be better to do something instead of
lying in bed?  For he had had forty years' experience behind him;
and Septimus could take Dr. Holmes's word for it--there was nothing
whatever the matter with him.  And next time Dr. Holmes came he
hoped to find Smith out of bed and not making that charming little
lady his wife anxious about him.

Human nature, in short, was on him--the repulsive brute, with the
blood-red nostrils.  Holmes was on him.  Dr. Holmes came quite
regularly every day.  Once you stumble, Septimus wrote on the back
of a postcard, human nature is on you.  Holmes is on you.  Their
only chance was to escape, without letting Holmes know; to Italy--
anywhere, anywhere, away from Dr. Holmes.

But Rezia could not understand him.  Dr. Holmes was such a kind
man.  He was so interested in Septimus.  He only wanted to help
them, he said.  He had four little children and he had asked her to
tea, she told Septimus.

So he was deserted.  The whole world was clamouring:  Kill
yourself, kill yourself, for our sakes.  But why should he kill
himself for their sakes?  Food was pleasant; the sun hot; and this
killing oneself, how does one set about it, with a table knife,
uglily, with floods of blood,--by sucking a gaspipe?  He was too
weak; he could scarcely raise his hand.  Besides, now that he was
quite alone, condemned, deserted, as those who are about to die are
alone, there was a luxury in it, an isolation full of sublimity; a
freedom which the attached can never know.  Holmes had won of
course; the brute with the red nostrils had won.  But even Holmes
himself could not touch this last relic straying on the edge of the
world, this outcast, who gazed back at the inhabited regions, who
lay, like a drowned sailor, on the shore of the world.

It was at that moment (Rezia gone shopping) that the great
revelation took place.  A voice spoke from behind the screen.
Evans was speaking.  The dead were with him.

"Evans, Evans!" he cried.

Mr. Smith was talking aloud to himself, Agnes the servant girl
cried to Mrs. Filmer in the kitchen.  "Evans, Evans," he had said
as she brought in the tray.  She jumped, she did.  She scuttled
downstairs.

And Rezia came in, with her flowers, and walked across the room,
and put the roses in a vase, upon which the sun struck directly,
and it went laughing, leaping round the room.

She had had to buy the roses, Rezia said, from a poor man in the
street.  But they were almost dead already, she said, arranging the
roses.

So there was a man outside; Evans presumably; and the roses, which
Rezia said were half dead, had been picked by him in the fields of
Greece.  "Communication is health; communication is happiness,
communication--" he muttered.

"What are you saying, Septimus?" Rezia asked, wild with terror, for
he was talking to himself.

She sent Agnes running for Dr. Holmes.  Her husband, she said, was
mad.  He scarcely knew her.

"You brute!  You brute!" cried Septimus, seeing human nature, that
is Dr. Holmes, enter the room.

"Now what's all this about?" said Dr. Holmes in the most amiable
way in the world.  "Talking nonsense to frighten your wife?"  But
he would give him something to make him sleep.  And if they were
rich people, said Dr. Holmes, looking ironically round the room, by
all means let them go to Harley Street; if they had no confidence
in him, said Dr. Holmes, looking not quite so kind.

It was precisely twelve o'clock; twelve by Big Ben; whose stroke
was wafted over the northern part of London; blent with that of
other clocks, mixed in a thin ethereal way with the clouds and
wisps of smoke, and died up there among the seagulls--twelve
o'clock struck as Clarissa Dalloway laid her green dress on her
bed, and the Warren Smiths walked down Harley Street.  Twelve was
the hour of their appointment.  Probably, Rezia thought, that was
Sir William Bradshaw's house with the grey motor car in front of
it.  The leaden circles dissolved in the air.

Indeed it was--Sir William Bradshaw's motor car; low, powerful,
grey with plain initials' interlocked on the panel, as if the pomps
of heraldry were incongruous, this man being the ghostly helper,
the priest of science; and, as the motor car was grey, so to match
its sober suavity, grey furs, silver grey rugs were heaped in it,
to keep her ladyship warm while she waited.  For often Sir William
would travel sixty miles or more down into the country to visit the
rich, the afflicted, who could afford the very large fee which Sir
William very properly charged for his advice.  Her ladyship waited
with the rugs about her knees an hour or more, leaning back,
thinking sometimes of the patient, sometimes, excusably, of the
wall of gold, mounting minute by minute while she waited; the wall
of gold that was mounting between them and all shifts and anxieties
(she had borne them bravely; they had had their struggles) until
she felt wedged on a calm ocean, where only spice winds blow;
respected, admired, envied, with scarcely anything left to wish
for, though she regretted her stoutness; large dinner-parties every
Thursday night to the profession; an occasional bazaar to be
opened; Royalty greeted; too little time, alas, with her husband,
whose work grew and grew; a boy doing well at Eton; she would have
liked a daughter too; interests she had, however, in plenty; child
welfare; the after-care of the epileptic, and photography, so that
if there was a church building, or a church decaying, she bribed
the sexton, got the key and took photographs, which were scarcely
to be distinguished from the work of professionals, while she
waited.

Sir William himself was no longer young.  He had worked very hard;
he had won his position by sheer ability (being the son of a
shopkeeper); loved his profession; made a fine figurehead at
ceremonies and spoke well--all of which had by the time he was
knighted given him a heavy look, a weary look (the stream of
patients being so incessant, the responsibilities and privileges of
his profession so onerous), which weariness, together with his grey
hairs, increased the extraordinary distinction of his presence and
gave him the reputation (of the utmost importance in dealing with
nerve cases) not merely of lightning skill, and almost infallible
accuracy in diagnosis but of sympathy; tact; understanding of the
human soul.  He could see the first moment they came into the room
(the Warren Smiths they were called); he was certain directly he
saw the man; it was a case of extreme gravity.  It was a case of
complete breakdown--complete physical and nervous breakdown, with
every symptom in an advanced stage, he ascertained in two or three
minutes (writing answers to questions, murmured discreetly, on a
pink card).

How long had Dr. Holmes been attending him?

Six weeks.

Prescribed a little bromide?  Said there was nothing the matter?
Ah yes (those general practitioners! thought Sir William.  It took
half his time to undo their blunders.  Some were irreparable).

"You served with great distinction in the War?"

The patient repeated the word "war" interrogatively.

He was attaching meanings to words of a symbolical kind.  A serious
symptom, to be noted on the card.

"The War?" the patient asked.  The European War--that little shindy
of schoolboys with gunpowder?  Had he served with distinction?  He
really forgot.  In the War itself he had failed.

"Yes, he served with the greatest distinction," Rezia assured the
doctor; "he was promoted."

"And they have the very highest opinion of you at your office?" Sir
William murmured, glancing at Mr. Brewer's very generously worded
letter.  "So that you have nothing to worry you, no financial
anxiety, nothing?"

He had committed an appalling crime and been condemned to death by
human nature.

"I have--I have," he began, "committed a crime--"

"He has done nothing wrong whatever," Rezia assured the doctor.  If
Mr. Smith would wait, said Sir William, he would speak to Mrs.
Smith in the next room.  Her husband was very seriously ill, Sir
William said.  Did he threaten to kill himself?

Oh, he did, she cried.  But he did not mean it, she said.  Of
course not.  It was merely a question of rest, said Sir William; of
rest, rest, rest; a long rest in bed.  There was a delightful home
down in the country where her husband would be perfectly looked
after.  Away from her? she asked.  Unfortunately, yes; the people
we care for most are not good for us when we are ill.  But he was
not mad, was he?  Sir William said he never spoke of "madness"; he
called it not having a sense of proportion.  But her husband did
not like doctors.  He would refuse to go there.  Shortly and kindly
Sir William explained to her the state of the case.  He had
threatened to kill himself.  There was no alternative.  It was a
question of law.  He would lie in bed in a beautiful house in the
country.  The nurses were admirable.  Sir William would visit him
once a week.  If Mrs. Warren Smith was quite sure she had no more
questions to ask--he never hurried his patients--they would return
to her husband.  She had nothing more to ask--not of Sir William.

So they returned to the most exalted of mankind; the criminal who
faced his judges; the victim exposed on the heights; the fugitive;
the drowned sailor; the poet of the immortal ode; the Lord who had
gone from life to death; to Septimus Warren Smith, who sat in the
arm-chair under the skylight staring at a photograph of Lady
Bradshaw in Court dress, muttering messages about beauty.

"We have had our little talk," said Sir William.

"He says you are very, very ill," Rezia cried.

"We have been arranging that you should go into a home," said Sir
William.

"One of Holmes's homes?" sneered Septimus.

The fellow made a distasteful impression.  For there was in Sir
William, whose father had been a tradesman, a natural respect for
breeding and clothing, which shabbiness nettled; again, more
profoundly, there was in Sir William, who had never had time for
reading, a grudge, deeply buried, against cultivated people who
came into his room and intimated that doctors, whose profession is
a constant strain upon all the highest faculties, are not educated
men.

"One of MY homes, Mr. Warren Smith," he said, "where we will teach
you to rest."

And there was just one thing more.

He was quite certain that when Mr. Warren Smith was well he was the
last man in the world to frighten his wife.  But he had talked of
killing himself.

"We all have our moments of depression," said Sir William.

Once you fall, Septimus repeated to himself, human nature is on
you.  Holmes and Bradshaw are on you.  They scour the desert.  They
fly screaming into the wilderness.  The rack and the thumbscrew are
applied.  Human nature is remorseless.

"Impulses came upon him sometimes?" Sir William asked, with his
pencil on a pink card.

That was his own affair, said Septimus.

"Nobody lives for himself alone," said Sir William, glancing at the
photograph of his wife in Court dress.

"And you have a brilliant career before you," said Sir William.
There was Mr. Brewer's letter on the table.  "An exceptionally
brilliant career."

But if he confessed?  If he communicated?  Would they let him off
then, his torturers?

"I--I--" he stammered.

But what was his crime?  He could not remember it.

"Yes?"  Sir William encouraged him.  (But it was growing late.)

Love, trees, there is no crime--what was his message?

He could not remember it.

"I--I--" Septimus stammered.

"Try to think as little about yourself as possible," said Sir
William kindly.  Really, he was not fit to be about.

Was there anything else they wished to ask him?  Sir William would
make all arrangements (he murmured to Rezia) and he would let her
know between five and six that evening he murmured.

"Trust everything to me," he said, and dismissed them.

Never, never had Rezia felt such agony in her life!  She had asked
for help and been deserted!  He had failed them!  Sir William
Bradshaw was not a nice man.

The upkeep of that motor car alone must cost him quite a lot, said
Septimus, when they got out into the street.

She clung to his arm.  They had been deserted.

But what more did she want?

To his patients he gave three-quarters of an hour; and if in this
exacting science which has to do with what, after all, we know
nothing about--the nervous system, the human brain--a doctor loses
his sense of proportion, as a doctor he fails.  Health we must
have; and health is proportion; so that when a man comes into your
room and says he is Christ (a common delusion), and has a message,
as they mostly have, and threatens, as they often do, to kill
himself, you invoke proportion; order rest in bed; rest in
solitude; silence and rest; rest without friends, without books,
without messages; six months' rest; until a man who went in
weighing seven stone six comes out weighing twelve.

Proportion, divine proportion, Sir William's goddess, was acquired
by Sir William walking hospitals, catching salmon, begetting one
son in Harley Street by Lady Bradshaw, who caught salmon herself
and took photographs scarcely to be distinguished from the work of
professionals.  Worshipping proportion, Sir William not only
prospered himself but made England prosper, secluded her lunatics,
forbade childbirth, penalised despair, made it impossible for the
unfit to propagate their views until they, too, shared his sense of
proportion--his, if they were men, Lady Bradshaw's if they were
women (she embroidered, knitted, spent four nights out of seven at
home with her son), so that not only did his colleagues respect
him, his subordinates fear him, but the friends and relations of
his patients felt for him the keenest gratitude for insisting that
these prophetic Christs and Christesses, who prophesied the end of
the world, or the advent of God, should drink milk in bed, as Sir
William ordered; Sir William with his thirty years' experience of
these kinds of cases, and his infallible instinct, this is madness,
this sense; in fact, his sense of proportion.

But Proportion has a sister, less smiling, more formidable, a
Goddess even now engaged--in the heat and sands of India, the mud
and swamp of Africa, the purlieus of London, wherever in short the
climate or the devil tempts men to fall from the true belief which
is her own--is even now engaged in dashing down shrines, smashing
idols, and setting up in their place her own stern countenance.
Conversion is her name and she feasts on the wills of the weakly,
loving to impress, to impose, adoring her own features stamped on
the face of the populace.  At Hyde Park Corner on a tub she stands
preaching; shrouds herself in white and walks penitentially
disguised as brotherly love through factories and parliaments;
offers help, but desires power; smites out of her way roughly the
dissentient, or dissatisfied; bestows her blessing on those who,
looking upward, catch submissively from her eyes the light of their
own.  This lady too (Rezia Warren Smith divined it) had her
dwelling in Sir William's heart, though concealed, as she mostly
is, under some plausible disguise; some venerable name; love, duty,
self sacrifice.  How he would work--how toil to raise funds,
propagate reforms, initiate institutions!  But conversion,
fastidious Goddess, loves blood better than brick, and feasts most
subtly on the human will.  For example, Lady Bradshaw.  Fifteen
years ago she had gone under.  It was nothing you could put your
finger on; there had been no scene, no snap; only the slow sinking,
water-logged, of her will into his.  Sweet was her smile, swift her
submission; dinner in Harley Street, numbering eight or nine
courses, feeding ten or fifteen guests of the professional classes,
was smooth and urbane.  Only as the evening wore on a very slight
dulness, or uneasiness perhaps, a nervous twitch, fumble, stumble
and confusion indicated, what it was really painful to believe--
that the poor lady lied.  Once, long ago, she had caught salmon
freely: now, quick to minister to the craving which lit her
husband's eye so oilily for dominion, for power, she cramped,
squeezed, pared, pruned, drew back, peeped through; so that without
knowing precisely what made the evening disagreeable, and caused
this pressure on the top of the head (which might well be imputed
to the professional conversation, or the fatigue of a great doctor
whose life, Lady Bradshaw said, "is not his own but his patients'")
disagreeable it was: so that guests, when the clock struck ten,
breathed in the air of Harley Street even with rapture; which
relief, however, was denied to his patients.

There in the grey room, with the pictures on the wall, and the
valuable furniture, under the ground glass skylight, they learnt
the extent of their transgressions; huddled up in arm-chairs, they
watched him go through, for their benefit, a curious exercise with
the arms, which he shot out, brought sharply back to his hip, to
prove (if the patient was obstinate) that Sir William was master of
his own actions, which the patient was not.  There some weakly
broke down; sobbed, submitted; others, inspired by Heaven knows
what intemperate madness, called Sir William to his face a damnable
humbug; questioned, even more impiously, life itself.  Why live?
they demanded.  Sir William replied that life was good.  Certainly
Lady Bradshaw in ostrich feathers hung over the mantelpiece, and as
for his income it was quite twelve thousand a year.  But to us,
they protested, life has given no such bounty.  He acquiesced.
They lacked a sense of proportion.  And perhaps, after all, there
is no God?  He shrugged his shoulders.  In short, this living or
not living is an affair of our own?  But there they were mistaken.
Sir William had a friend in Surrey where they taught, what Sir
William frankly admitted was a difficult art--a sense of
proportion.  There were, moreover, family affection; honour;
courage; and a brilliant career.  All of these had in Sir William a
resolute champion.  If they failed him, he had to support police
and the good of society, which, he remarked very quietly, would
take care, down in Surrey, that these unsocial impulses, bred more
than anything by the lack of good blood, were held in control.  And
then stole out from her hiding-place and mounted her throne that
Goddess whose lust is to override opposition, to stamp indelibly
in the sanctuaries of others the image of herself.  Naked,
defenceless, the exhausted, the friendless received the impress of
Sir William's will.  He swooped; he devoured.  He shut people up.
It was this combination of decision and humanity that endeared Sir
William so greatly to the relations of his victims.

But Rezia Warren Smith cried, walking down Harley Street, that she
did not like that man.

Shredding and slicing, dividing and subdividing, the clocks of
Harley Street nibbled at the June day, counselled submission,
upheld authority, and pointed out in chorus the supreme advantages
of a sense of proportion, until the mound of time was so far
diminished that a commercial clock, suspended above a shop in
Oxford Street, announced, genially and fraternally, as if it were a
pleasure to Messrs. Rigby and Lowndes to give the information
gratis, that it was half-past one.

Looking up, it appeared that each letter of their names stood for
one of the hours; subconsciously one was grateful to Rigby and
Lowndes for giving one time ratified by Greenwich; and this
gratitude (so Hugh Whitbread ruminated, dallying there in front of
the shop window), naturally took the form later of buying off Rigby
and Lowndes socks or shoes.  So he ruminated.  It was his habit.
He did not go deeply.  He brushed surfaces; the dead languages, the
living, life in Constantinople, Paris, Rome; riding, shooting,
tennis, it had been once.  The malicious asserted that he now kept
guard at Buckingham Palace, dressed in silk stockings and knee-
breeches, over what nobody knew.  But he did it extremely
efficiently.  He had been afloat on the cream of English society
for fifty-five years.  He had known Prime Ministers.  His
affections were understood to be deep.  And if it were true that he
had not taken part in any of the great movements of the time or
held important office, one or two humble reforms stood to his
credit; an improvement in public shelters was one; the protection
of owls in Norfolk another; servant girls had reason to be grateful
to him; and his name at the end of letters to the Times, asking for
funds, appealing to the public to protect, to preserve, to clear up
litter, to abate smoke, and stamp out immorality in parks,
commanded respect.

A magnificent figure he cut too, pausing for a moment (as the sound
of the half hour died away) to look critically, magisterially, at
socks and shoes; impeccable, substantial, as if he beheld the world
from a certain eminence, and dressed to match; but realised the
obligations which size, wealth, health, entail, and observed
punctiliously even when not absolutely necessary, little
courtesies, old-fashioned ceremonies which gave a quality to his
manner, something to imitate, something to remember him by, for he
would never lunch, for example, with Lady Bruton, whom he had known
these twenty years, without bringing her in his outstretched hand a
bunch of carnations and asking Miss Brush, Lady Bruton's secretary,
after her brother in South Africa, which, for some reason, Miss
Brush, deficient though she was in every attribute of female charm,
so much resented that she said "Thank you, he's doing very well in
South Africa," when, for half a dozen years, he had been doing
badly in Portsmouth.

Lady Bruton herself preferred Richard Dalloway, who arrived at the
next moment.  Indeed they met on the doorstep.

Lady Bruton preferred Richard Dalloway of course.  He was made of
much finer material.  But she wouldn't let them run down her poor
dear Hugh.  She could never forget his kindness--he had been really
remarkably kind--she forgot precisely upon what occasion.  But he
had been--remarkably kind.  Anyhow, the difference between one man
and another does not amount to much.  She had never seen the sense
of cutting people up, as Clarissa Dalloway did--cutting them up and
sticking them together again; not at any rate when one was sixty-
two.  She took Hugh's carnations with her angular grim smile.
There was nobody else coming, she said.  She had got them there on
false pretences, to help her out of a difficulty--

"But let us eat first," she said.

And so there began a soundless and exquisite passing to and fro
through swing doors of aproned white-capped maids, handmaidens not
of necessity, but adepts in a mystery or grand deception practised
by hostesses in Mayfair from one-thirty to two, when, with a wave
of the hand, the traffic ceases, and there rises instead this
profound illusion in the first place about the food--how it is not
paid for; and then that the table spreads itself voluntarily with
glass and silver, little mats, saucers of red fruit; films of brown
cream mask turbot; in casseroles severed chickens swim; coloured,
undomestic, the fire burns; and with the wine and the coffee (not
paid for) rise jocund visions before musing eyes; gently
speculative eyes; eyes to whom life appears musical, mysterious;
eyes now kindled to observe genially the beauty of the red
carnations which Lady Bruton (whose movements were always angular)
had laid beside her plate, so that Hugh Whitbread, feeling at peace
with the entire universe and at the same time completely sure of
his standing, said, resting his fork,

"Wouldn't they look charming against your lace?"

Miss Brush resented this familiarity intensely.  She thought him an
underbred fellow.  She made Lady Bruton laugh.

Lady Bruton raised the carnations, holding them rather stiffly with
much the same attitude with which the General held the scroll in
the picture behind her; she remained fixed, tranced.  Which was she
now, the General's great-grand-daughter? great-great-grand-
daughter? Richard Dalloway asked himself.  Sir Roderick, Sir Miles,
Sir Talbot--that was it.  It was remarkable how in that family the
likeness persisted in the women.  She should have been a general of
dragoons herself.  And Richard would have served under her,
cheerfully; he had the greatest respect for her; he cherished these
romantic views about well-set-up old women of pedigree, and would
have liked, in his good-humoured way, to bring some young hot-heads
of his acquaintance to lunch with her; as if a type like hers could
be bred of amiable tea-drinking enthusiasts!  He knew her country.
He knew her people.  There was a vine, still bearing, which either
Lovelace or Herrick--she never read a word poetry of herself, but
so the story ran--had sat under.  Better wait to put before them
the question that bothered her (about making an appeal to the
public; if so, in what terms and so on), better wait until they
have had their coffee, Lady Bruton thought; and so laid the
carnations down beside her plate.

"How's Clarissa?" she asked abruptly.

Clarissa always said that Lady Bruton did not like her.  Indeed,
Lady Bruton had the reputation of being more interested in politics
than people; of talking like a man; of having had a finger in some
notorious intrigue of the eighties, which was now beginning to be
mentioned in memoirs.  Certainly there was an alcove in her
drawing-room, and a table in that alcove, and a photograph upon
that table of General Sir Talbot Moore, now deceased, who had
written there (one evening in the eighties) in Lady Bruton's
presence, with her cognisance, perhaps advice, a telegram ordering
the British troops to advance upon an historical occasion.  (She
kept the pen and told the story.)  Thus, when she said in her
offhand way "How's Clarissa?" husbands had difficulty in persuading
their wives and indeed, however devoted, were secretly doubtful
themselves, of her interest in women who often got in their
husbands' way, prevented them from accepting posts abroad, and had
to be taken to the seaside in the middle of the session to recover
from influenza.  Nevertheless her inquiry, "How's Clarissa?" was
known by women infallibly, to be a signal from a well-wisher, from
an almost silent companion, whose utterances (half a dozen perhaps
in the course of a lifetime) signified recognition of some feminine
comradeship which went beneath masculine lunch parties and united
Lady Bruton and Mrs. Dalloway, who seldom met, and appeared when
they did meet indifferent and even hostile, in a singular bond.

"I met Clarissa in the Park this morning," said Hugh Whitbread,
diving into the casserole, anxious to pay himself this little
tribute, for he had only to come to London and he met everybody at
once; but greedy, one of the greediest men she had ever known,
Milly Brush thought, who observed men with unflinching rectitude,
and was capable of everlasting devotion, to her own sex in
particular, being knobbed, scraped, angular, and entirely without
feminine charm.

"D'you know who's in town?" said Lady Bruton suddenly bethinking
her.  "Our old friend, Peter Walsh."

They all smiled.  Peter Walsh!  And Mr. Dalloway was genuinely
glad, Milly Brush thought; and Mr. Whitbread thought only of his
chicken.

Peter Walsh!  All three, Lady Bruton, Hugh Whitbread, and Richard
Dalloway, remembered the same thing--how passionately Peter had
been in love; been rejected; gone to India; come a cropper; made a
mess of things; and Richard Dalloway had a very great liking for
the dear old fellow too.  Milly Brush saw that; saw a depth in the
brown of his eyes; saw him hesitate; consider; which interested
her, as Mr. Dalloway always interested her, for what was he
thinking, she wondered, about Peter Walsh?

That Peter Walsh had been in love with Clarissa; that he would go
back directly after lunch and find Clarissa; that he would tell
her, in so many words, that he loved her.  Yes, he would say that.

Milly Brush once might almost have fallen in love with these
silences; and Mr. Dalloway was always so dependable; such a
gentleman too.  Now, being forty, Lady Bruton had only to nod, or
turn her head a little abruptly, and Milly Brush took the signal,
however deeply she might be sunk in these reflections of a detached
spirit, of an uncorrupted soul whom life could not bamboozle,
because life had not offered her a trinket of the slightest value;
not a curl, smile, lip, cheek, nose; nothing whatever; Lady Bruton
had only to nod, and Perkins was instructed to quicken the coffee.

"Yes; Peter Walsh has come back," said Lady Bruton.  It was vaguely
flattering to them all.  He had come back, battered, unsuccessful,
to their secure shores.  But to help him, they reflected, was
impossible; there was some flaw in his character.  Hugh Whitbread
said one might of course mention his name to So-and-so.  He
wrinkled lugubriously, consequentially, at the thought of the
letters he would write to the heads of Government offices about "my
old friend, Peter Walsh," and so on.  But it wouldn't lead to
anything--not to anything permanent, because of his character.

"In trouble with some woman," said Lady Bruton.  They had all
guessed that THAT was at the bottom of it.

"However," said Lady Bruton, anxious to leave the subject, "we
shall hear the whole story from Peter himself."

(The coffee was very slow in coming.)

"The address?" murmured Hugh Whitbread; and there was at once a
ripple in the grey tide of service which washed round Lady Bruton
day in, day out, collecting, intercepting, enveloping her in a fine
tissue which broke concussions, mitigated interruptions, and spread
round the house in Brook Street a fine net where things lodged and
were picked out accurately, instantly, by grey-haired Perkins, who
had been with Lady Bruton these thirty years and now wrote down the
address; handed it to Mr. Whitbread, who took out his pocket-book,
raised his eyebrows, and slipping it in among documents of the
highest importance, said that he would get Evelyn to ask him to
lunch.

(They were waiting to bring the coffee until Mr. Whitbread had
finished.)

Hugh was very slow, Lady Bruton thought.  He was getting fat, she
noticed.  Richard always kept himself in the pink of condition.
She was getting impatient; the whole of her being was setting
positively, undeniably, domineeringly brushing aside all this
unnecessary trifling (Peter Walsh and his affairs) upon that
subject which engaged her attention, and not merely her attention,
but that fibre which was the ramrod of her soul, that essential
part of her without which Millicent Bruton would not have been
Millicent Bruton; that project for emigrating young people of both
sexes born of respectable parents and setting them up with a fair
prospect of doing well in Canada.  She exaggerated.  She had
perhaps lost her sense of proportion.  Emigration was not to others
the obvious remedy, the sublime conception.  It was not to them
(not to Hugh, or Richard, or even to devoted Miss Brush) the
liberator of the pent egotism, which a strong martial woman, well
nourished, well descended, of direct impulses, downright feelings,
and little introspective power (broad and simple--why could not
every one be broad and simple? she asked) feels rise within her,
once youth is past, and must eject upon some object--it may be
Emigration, it may be Emancipation; but whatever it be, this object
round which the essence of her soul is daily secreted, becomes
inevitably prismatic, lustrous, half looking-glass, half precious
stone; now carefully hidden in case people should sneer at it; now
proudly displayed.  Emigration had become, in short, largely Lady
Bruton.

But she had to write.  And one letter to the Times, she used to say
to Miss Brush, cost her more than to organise an expedition to
South Africa (which she had done in the war).  After a morning's
battle beginning, tearing up, beginning again, she used to feel the
futility of her own womanhood as she felt it on no other occasion,
and would turn gratefully to the thought of Hugh Whitbread who
possessed--no one could doubt it--the art of writing letters to the
Times.

A being so differently constituted from herself, with such a
command of language; able to put things as editors like them put;
had passions which one could not call simply greed.  Lady Bruton
often suspended judgement upon men in deference to the mysterious
accord in which they, but no woman, stood to the laws of the
universe; knew how to put things; knew what was said; so that if
Richard advised her, and Hugh wrote for her, she was sure of being
somehow right.  So she let Hugh eat his soufflé; asked after poor
Evelyn; waited until they were smoking, and then said,

"Milly, would you fetch the papers?"

And Miss Brush went out, came back; laid papers on the table; and
Hugh produced his fountain pen; his silver fountain pen, which had
done twenty years' service, he said, unscrewing the cap.  It was
still in perfect order; he had shown it to the makers; there was no
reason, they said, why it should ever wear out; which was somehow
to Hugh's credit, and to the credit of the sentiments which his pen
expressed (so Richard Dalloway felt) as Hugh began carefully
writing capital letters with rings round them in the margin, and
thus marvellously reduced Lady Bruton's tangles to sense, to
grammar such as the editor of the Times, Lady Bruton felt, watching
the marvellous transformation, must respect.  Hugh was slow.  Hugh
was pertinacious.  Richard said one must take risks.  Hugh proposed
modifications in deference to people's feelings, which, he said
rather tartly when Richard laughed, "had to be considered," and
read out "how, therefore, we are of opinion that the times are ripe
. . . the superfluous youth of our ever-increasing population . . .
what we owe to the dead . . ." which Richard thought all stuffing
and bunkum, but no harm in it, of course, and Hugh went on drafting
sentiments in alphabetical order of the highest nobility, brushing
the cigar ash from his waistcoat, and summing up now and then the
progress they had made until, finally, he read out the draft of a
letter which Lady Bruton felt certain was a masterpiece.  Could her
own meaning sound like that?

Hugh could not guarantee that the editor would put it in; but he
would be meeting somebody at luncheon.

Whereupon Lady Bruton, who seldom did a graceful thing, stuffed all
Hugh's carnations into the front of her dress, and flinging her
hands out called him "My Prime Minister!"  What she would have done
without them both she did not know.  They rose.  And Richard
Dalloway strolled off as usual to have a look at the General's
portrait, because he meant, whenever he had a moment of leisure, to
write a history of Lady Bruton's family.

And Millicent Bruton was very proud of her family.  But they could
wait, they could wait, she said, looking at the picture; meaning
that her family, of military men, administrators, admirals, had
been men of action, who had done their duty; and Richard's first
duty was to his country, but it was a fine face, she said; and all
the papers were ready for Richard down at Aldmixton whenever the
time came; the Labour Government she meant.  "Ah, the news from
India!" she cried.

And then, as they stood in the hall taking yellow gloves from the
bowl on the malachite table and Hugh was offering Miss Brush with
quite unnecessary courtesy some discarded ticket or other
compliment, which she loathed from the depths of her heart and
blushed brick red, Richard turned to Lady Bruton, with his hat in
his hand, and said,

"We shall see you at our party to-night?" whereupon Lady Bruton
resumed the magnificence which letter-writing had shattered.  She
might come; or she might not come.  Clarissa had wonderful energy.
Parties terrified Lady Bruton.  But then, she was getting old.  So
she intimated, standing at her doorway; handsome; very erect; while
her chow stretched behind her, and Miss Brush disappeared into the
background with her hands full of papers.

And Lady Bruton went ponderously, majestically, up to her room,
lay, one arm extended, on the sofa.  She sighed, she snored, not
that she was asleep, only drowsy and heavy, drowsy and heavy, like
a field of clover in the sunshine this hot June day, with the bees
going round and about and the yellow butterflies.  Always she went
back to those fields down in Devonshire, where she had jumped the
brooks on Patty, her pony, with Mortimer and Tom, her brothers.
And there were the dogs; there were the rats; there were her father
and mother on the lawn under the trees, with the tea-things out,
and the beds of dahlias, the hollyhocks, the pampas grass; and
they, little wretches, always up to some mischief! stealing back
through the shrubbery, so as not to be seen, all bedraggled from
some roguery.  What old nurse used to say about her frocks!

Ah dear, she remembered--it was Wednesday in Brook Street.  Those
kind good fellows, Richard Dalloway, Hugh Whitbread, had gone this
hot day through the streets whose growl came up to her lying on the
sofa.  Power was hers, position, income.  She had lived in the
forefront of her time.  She had had good friends; known the ablest
men of her day.  Murmuring London flowed up to her, and her hand,
lying on the sofa back, curled upon some imaginary baton such as
her grandfathers might have held, holding which she seemed, drowsy
and heavy, to be commanding battalions marching to Canada, and
those good fellows walking across London, that territory of theirs,
that little bit of carpet, Mayfair.

And they went further and further from her, being attached to her
by a thin thread (since they had lunched with her) which would
stretch and stretch, get thinner and thinner as they walked across
London; as if one's friends were attached to one's body, after
lunching with them, by a thin thread, which (as she dozed there)
became hazy with the sound of bells, striking the hour or ringing
to service, as a single spider's thread is blotted with rain-drops,
and, burdened, sags down.  So she slept.

And Richard Dalloway and Hugh Whitbread hesitated at the corner of
Conduit Street at the very moment that Millicent Bruton, lying on
the sofa, let the thread snap; snored.  Contrary winds buffeted at
the street corner.  They looked in at a shop window; they did not
wish to buy or to talk but to part, only with contrary winds
buffeting the street corner, with some sort of lapse in the tides
of the body, two forces meeting in a swirl, morning and afternoon,
they paused.  Some newspaper placard went up in the air, gallantly,
like a kite at first, then paused, swooped, fluttered; and a lady's
veil hung.  Yellow awnings trembled.  The speed of the morning
traffic slackened, and single carts rattled carelessly down half-
empty streets.  In Norfolk, of which Richard Dalloway was half
thinking, a soft warm wind blew back the petals; confused the
waters; ruffled the flowering grasses.  Haymakers, who had pitched
beneath hedges to sleep away the morning toil, parted curtains of
green blades; moved trembling globes of cow parsley to see the sky;
the blue, the steadfast, the blazing summer sky.

Aware that he was looking at a silver two-handled Jacobean mug, and
that Hugh Whitbread admired condescendingly with airs of
connoisseurship a Spanish necklace which he thought of asking the
price of in case Evelyn might like it--still Richard was torpid;
could not think or move.  Life had thrown up this wreckage; shop
windows full of coloured paste, and one stood stark with the
lethargy of the old, stiff with the rigidity of the old, looking
in.  Evelyn Whitbread might like to buy this Spanish necklace--so
she might.  Yawn he must.  Hugh was going into the shop.

"Right you are!" said Richard, following.

Goodness knows he didn't want to go buying necklaces with Hugh.
But there are tides in the body.  Morning meets afternoon.  Borne
like a frail shallop on deep, deep floods, Lady Bruton's great-
grandfather and his memoir and his campaigns in North America were
whelmed and sunk.  And Millicent Bruton too.  She went under.
Richard didn't care a straw what became of Emigration; about that
letter, whether the editor put it in or not.  The necklace hung
stretched between Hugh's admirable fingers.  Let him give it to a
girl, if he must buy jewels--any girl, any girl in the street.  For
the worthlessness of this life did strike Richard pretty forcibly--
buying necklaces for Evelyn.  If he'd had a boy he'd have said,
Work, work.  But he had his Elizabeth; he adored his Elizabeth.

"I should like to see Mr. Dubonnet," said Hugh in his curt worldly
way.  It appeared that this Dubonnet had the measurements of Mrs.
Whitbread's neck, or, more strangely still, knew her views upon
Spanish jewellery and the extent of her possessions in that line
(which Hugh could not remember).  All of which seemed to Richard
Dalloway awfully odd.  For he never gave Clarissa presents, except
a bracelet two or three years ago, which had not been a success.
She never wore it.  It pained him to remember that she never wore
it.  And as a single spider's thread after wavering here and there
attaches itself to the point of a leaf, so Richard's mind,
recovering from its lethargy, set now on his wife, Clarissa, whom
Peter Walsh had loved so passionately; and Richard had had a sudden
vision of her there at luncheon; of himself and Clarissa; of their
life together; and he drew the tray of old jewels towards him, and
taking up first this brooch then that ring, "How much is that?" he
asked, but doubted his own taste.  He wanted to open the drawing-
room door and come in holding out something; a present for
Clarissa.  Only what?  But Hugh was on his legs again.  He was
unspeakably pompous.  Really, after dealing here for thirty-five
years he was not going to be put off by a mere boy who did not know
his business.  For Dubonnet, it seemed, was out, and Hugh would not
buy anything until Mr. Dubonnet chose to be in; at which the youth
flushed and bowed his correct little bow.  It was all perfectly
correct.  And yet Richard couldn't have said that to save his life!
Why these people stood that damned insolence he could not conceive.
Hugh was becoming an intolerable ass.  Richard Dalloway could not
stand more than an hour of his society.  And, flicking his bowler
hat by way of farewell, Richard turned at the corner of Conduit
Street eager, yes, very eager, to travel that spider's thread of
attachment between himself and Clarissa; he would go straight to
her, in Westminster.

But he wanted to come in holding something.  Flowers?  Yes,
flowers, since he did not trust his taste in gold; any number of
flowers, roses, orchids, to celebrate what was, reckoning things as
you will, an event; this feeling about her when they spoke of Peter
Walsh at luncheon; and they never spoke of it; not for years had
they spoken of it; which, he thought, grasping his red and white
roses together (a vast bunch in tissue paper), is the greatest
mistake in the world.  The time comes when it can't be said; one's
too shy to say it, he thought, pocketing his sixpence or two of
change, setting off with his great bunch held against his body to
Westminster to say straight out in so many words (whatever she
might think of him), holding out his flowers, "I love you."  Why
not?  Really it was a miracle thinking of the war, and thousands of
poor chaps, with all their lives before them, shovelled together,
already half forgotten; it was a miracle.  Here he was walking
across London to say to Clarissa in so many words that he loved
her.  Which one never does say, he thought.  Partly one's lazy;
partly one's shy.  And Clarissa--it was difficult to think of her;
except in starts, as at luncheon, when he saw her quite distinctly;
their whole life.  He stopped at the crossing; and repeated--being
simple by nature, and undebauched, because he had tramped, and
shot; being pertinacious and dogged, having championed the down-
trodden and followed his instincts in the House of Commons; being
preserved in his simplicity yet at the same time grown rather
speechless, rather stiff--he repeated that it was a miracle that he
should have married Clarissa; a miracle--his life had been a
miracle, he thought; hesitating to cross.  But it did make his
blood boil to see little creatures of five or six crossing
Piccadilly alone.  The police ought to have stopped the traffic at
once.  He had no illusions about the London police.  Indeed, he was
collecting evidence of their malpractices; and those costermongers,
not allowed to stand their barrows in the streets; and prostitutes,
good Lord, the fault wasn't in them, nor in young men either, but
in our detestable social system and so forth; all of which he
considered, could be seen considering, grey, dogged, dapper, clean,
as he walked across the Park to tell his wife that he loved her.

For he would say it in so many words, when he came into the room.
Because it is a thousand pities never to say what one feels, he
thought, crossing the Green Park and observing with pleasure how in
the shade of the trees whole families, poor families, were
sprawling; children kicking up their legs; sucking milk; paper bags
thrown about, which could easily be picked up (if people objected)
by one of those fat gentlemen in livery; for he was of opinion that
every park, and every square, during the summer months should be
open to children (the grass of the park flushed and faded, lighting
up the poor mothers of Westminster and their crawling babies, as if
a yellow lamp were moved beneath).  But what could be done for
female vagrants like that poor creature, stretched on her elbow (as
if she had flung herself on the earth, rid of all ties, to observe
curiously, to speculate boldly, to consider the whys and the
wherefores, impudent, loose-lipped, humorous), he did not know.
Bearing his flowers like a weapon, Richard Dalloway approached her;
intent he passed her; still there was time for a spark between
them--she laughed at the sight of him, he smiled good-humouredly,
considering the problem of the female vagrant; not that they would
ever speak.  But he would tell Clarissa that he loved her, in so
many words.  He had, once upon a time, been jealous of Peter Walsh;
jealous of him and Clarissa.  But she had often said to him that
she had been right not to marry Peter Walsh; which, knowing
Clarissa, was obviously true; she wanted support.  Not that she was
weak; but she wanted support.

As for Buckingham Palace (like an old prima donna facing the
audience all in white) you can't deny it a certain dignity, he
considered, nor despise what does, after all, stand to millions of
people (a little crowd was waiting at the gate to see the King
drive out) for a symbol, absurd though it is; a child with a box of
bricks could have done better, he thought; looking at the memorial
to Queen Victoria (whom he could remember in her horn spectacles
driving through Kensington), its white mound, its billowing
motherliness; but he liked being ruled by the descendant of Horsa;
he liked continuity; and the sense of handing on the traditions of
the past.  It was a great age in which to have lived.  Indeed, his
own life was a miracle; let him make no mistake about it; here he
was, in the prime of life, walking to his house in Westminster to
tell Clarissa that he loved her.  Happiness is this he thought.

It is this, he said, as he entered Dean's Yard.  Big Ben was
beginning to strike, first the warning, musical; then the hour,
irrevocable.  Lunch parties waste the entire afternoon, he thought,
approaching his door.

The sound of Big Ben flooded Clarissa's drawing-room, where she
sat, ever so annoyed, at her writing-table; worried; annoyed.  It
was perfectly true that she had not asked Ellie Henderson to her
party; but she had done it on purpose.  Now Mrs. Marsham wrote "she
had told Ellie Henderson she would ask Clarissa--Ellie so much
wanted to come."

But why should she invite all the dull women in London to her
parties?  Why should Mrs. Marsham interfere?  And there was
Elizabeth closeted all this time with Doris Kilman.  Anything more
nauseating she could not conceive.  Prayer at this hour with that
woman.  And the sound of the bell flooded the room with its
melancholy wave; which receded, and gathered itself together to
fall once more, when she heard, distractingly, something fumbling,
something scratching at the door.  Who at this hour?  Three, good
Heavens!  Three already!  For with overpowering directness and
dignity the clock struck three; and she heard nothing else; but the
door handle slipped round and in came Richard!  What a surprise!
In came Richard, holding out flowers.  She had failed him, once at
Constantinople; and Lady Bruton, whose lunch parties were said to
be extraordinarily amusing, had not asked her.  He was holding out
flowers--roses, red and white roses.  (But he could not bring
himself to say he loved her; not in so many words.)

But how lovely, she said, taking his flowers.  She understood; she
understood without his speaking; his Clarissa.  She put them in
vases on the mantelpiece.  How lovely they looked! she said.  And
was it amusing, she asked?  Had Lady Bruton asked after her?  Peter
Walsh was back.  Mrs. Marsham had written.  Must she ask Ellie
Henderson?  That woman Kilman was upstairs.

"But let us sit down for five minutes," said Richard.

It all looked so empty.  All the chairs were against the wall.
What had they been doing?  Oh, it was for the party; no, he had not
forgotten, the party.  Peter Walsh was back.  Oh yes; she had had
him.  And he was going to get a divorce; and he was in love with
some woman out there.  And he hadn't changed in the slightest.
There she was, mending her dress. . . .

"Thinking of Bourton," she said.

"Hugh was at lunch," said Richard.  She had met him too!  Well, he
was getting absolutely intolerable.  Buying Evelyn necklaces;
fatter than ever; an intolerable ass.

"And it came over me 'I might have married you,'" she said,
thinking of Peter sitting there in his little bow-tie; with that
knife, opening it, shutting it.  "Just as he always was, you know."

They were talking about him at lunch, said Richard.  (But he could
not tell her he loved her.  He held her hand.  Happiness is this,
he thought.)  They had been writing a letter to the Times for
Millicent Bruton.  That was about all Hugh was fit for.

"And our dear Miss Kilman?" he asked.  Clarissa thought the roses
absolutely lovely; first bunched together; now of their own accord
starting apart.

"Kilman arrives just as we've done lunch," she said.  "Elizabeth
turns pink.  They shut themselves up.  I suppose they're praying."

Lord!  He didn't like it; but these things pass over if you let
them.

"In a mackintosh with an umbrella," said Clarissa.

He had not said "I love you"; but he held her hand.  Happiness is
this, is this, he thought.

"But why should I ask all the dull women in London to my parties?"
said Clarissa.  And if Mrs. Marsham gave a party, did SHE invite
her guests?

"Poor Ellie Henderson," said Richard--it was a very odd thing how
much Clarissa minded about her parties, he thought.

But Richard had no notion of the look of a room.  However--what was
he going to say?

If she worried about these parties he would not let her give them.
Did she wish she had married Peter?  But he must go.

He must be off, he said, getting up.  But he stood for a moment as
if he were about to say something; and she wondered what?  Why?
There were the roses.

"Some Committee?" she asked, as he opened the door.

"Armenians," he said; or perhaps it was "Albanians."

And there is a dignity in people; a solitude; even between husband
and wife a gulf; and that one must respect, thought Clarissa,
watching him open the door; for one would not part with it oneself,
or take it, against his will, from one's husband, without losing
one's independence, one's self-respect--something, after all,
priceless.

He returned with a pillow and a quilt.

"An hour's complete rest after luncheon," he said.  And he went.

How like him!  He would go on saying "An hour's complete rest after
luncheon" to the end of time, because a doctor had ordered it once.
It was like him to take what doctors said literally; part of his
adorable, divine simplicity, which no one had to the same extent;
which made him go and do the thing while she and Peter frittered
their time away bickering.  He was already halfway to the House of
Commons, to his Armenians, his Albanians, having settled her on the
sofa, looking at his roses.  And people would say, "Clarissa
Dalloway is spoilt."  She cared much more for her roses than for
the Armenians.  Hunted out of existence, maimed, frozen, the
victims of cruelty and injustice (she had heard Richard say so over
and over again)--no, she could feel nothing for the Albanians, or
was it the Armenians? but she loved her roses (didn't that help the
Armenians?)--the only flowers she could bear to see cut.  But
Richard was already at the House of Commons; at his Committee,
having settled all her difficulties.  But no; alas, that was not
true.  He did not see the reasons against asking Ellie Henderson.
She would do it, of course, as he wished it.  Since he had brought
the pillows, she would lie down. . . .  But--but--why did she
suddenly feel, for no reason that she could discover, desperately
unhappy?  As a person who has dropped some grain of pearl or
diamond into the grass and parts the tall blades very carefully,
this way and that, and searches here and there vainly, and at last
spies it there at the roots, so she went through one thing and
another; no, it was not Sally Seton saying that Richard would never
be in the Cabinet because he had a second-class brain (it came back
to her); no, she did not mind that; nor was it to do with Elizabeth
either and Doris Kilman; those were facts.  It was a feeling, some
unpleasant feeling, earlier in the day perhaps; something that
Peter had said, combined with some depression of her own, in her
bedroom, taking off her hat; and what Richard had said had added to
it, but what had he said?  There were his roses.  Her parties!
That was it!  Her parties!  Both of them criticised her very
unfairly, laughed at her very unjustly, for her parties.  That was
it!  That was it!

Well, how was she going to defend herself?  Now that she knew what
it was, she felt perfectly happy.  They thought, or Peter at any
rate thought, that she enjoyed imposing herself; liked to have
famous people about her; great names; was simply a snob in short.
Well, Peter might think so.  Richard merely thought it foolish of
her to like excitement when she knew it was bad for her heart.  It
was childish, he thought.  And both were quite wrong.  What she
liked was simply life.

"That's what I do it for," she said, speaking aloud, to life.

Since she was lying on the sofa, cloistered, exempt, the presence
of this thing which she felt to be so obvious became physically
existent; with robes of sound from the street, sunny, with hot
breath, whispering, blowing out the blinds.  But suppose Peter said
to her, "Yes, yes, but your parties--what's the sense of your
parties?" all she could say was (and nobody could be expected to
understand):  They're an offering; which sounded horribly vague.
But who was Peter to make out that life was all plain sailing?--
Peter always in love, always in love with the wrong woman?  What's
your love? she might say to him.  And she knew his answer; how it
is the most important thing in the world and no woman possibly
understood it.  Very well.  But could any man understand what she
meant either? about life?  She could not imagine Peter or Richard
taking the trouble to give a party for no reason whatever.

But to go deeper, beneath what people said (and these judgements,
how superficial, how fragmentary they are!) in her own mind now,
what did it mean to her, this thing she called life?  Oh, it was
very queer.  Here was So-and-so in South Kensington; some one up in
Bayswater; and somebody else, say, in Mayfair.  And she felt quite
continuously a sense of their existence; and she felt what a waste;
and she felt what a pity; and she felt if only they could be
brought together; so she did it.  And it was an offering; to
combine, to create; but to whom?

An offering for the sake of offering, perhaps.  Anyhow, it was her
gift.  Nothing else had she of the slightest importance; could not
think, write, even play the piano.  She muddled Armenians and
Turks; loved success; hated discomfort; must be liked; talked
oceans of nonsense: and to this day, ask her what the Equator was,
and she did not know.  All the same, that one day should follow
another; Wednesday, Thursday, Friday, Saturday; that one should
wake up in the morning; see the sky; walk in the park; meet Hugh
Whitbread; then suddenly in came Peter; then these roses; it was
enough.  After that, how unbelievable death was!--that it must end;
and no one in the whole world would know how she had loved it all;
how, every instant . . .

The door opened.  Elizabeth knew that her mother was resting.  She
came in very quietly.  She stood perfectly still.  Was it that some
Mongol had been wrecked on the coast of Norfolk (as Mrs. Hilbery
said), had mixed with the Dalloway ladies, perhaps, a hundred years
ago?  For the Dalloways, in general, were fair-haired; blue-eyed;
Elizabeth, on the contrary, was dark; had Chinese eyes in a pale
face; an Oriental mystery; was gentle, considerate, still.  As a
child, she had had a perfect sense of humour; but now at seventeen,
why, Clarissa could not in the least understand, she had become
very serious; like a hyacinth, sheathed in glossy green, with buds
just tinted, a hyacinth which has had no sun.

She stood quite still and looked at her mother; but the door was
ajar, and outside the door was Miss Kilman, as Clarissa knew; Miss
Kilman in her mackintosh, listening to whatever they said.

Yes, Miss Kilman stood on the landing, and wore a mackintosh; but
had her reasons.  First, it was cheap; second, she was over forty;
and did not, after all, dress to please.  She was poor, moreover;
degradingly poor.  Otherwise she would not be taking jobs from
people like the Dalloways; from rich people, who liked to be kind.
Mr. Dalloway, to do him justice, had been kind.  But Mrs. Dalloway
had not.  She had been merely condescending.  She came from the
most worthless of all classes--the rich, with a smattering of
culture.  They had expensive things everywhere; pictures, carpets,
lots of servants.  She considered that she had a perfect right to
anything that the Dalloways did for her.

She had been cheated.  Yes, the word was no exaggeration, for
surely a girl has a right to some kind of happiness?  And she had
never been happy, what with being so clumsy and so poor.  And then,
just as she might have had a chance at Miss Dolby's school, the war
came; and she had never been able to tell lies.  Miss Dolby thought
she would be happier with people who shared her views about the
Germans.  She had had to go.  It was true that the family was of
German origin; spelt the name Kiehlman in the eighteenth century;
but her brother had been killed.  They turned her out because she
would not pretend that the Germans were all villains--when she had
German friends, when the only happy days of her life had been spent
in Germany!  And after all, she could read history.  She had had to
take whatever she could get.  Mr. Dalloway had come across her
working for the Friends.  He had allowed her (and that was really
generous of him) to teach his daughter history.  Also she did a
little Extension lecturing and so on.  Then Our Lord had come to
her (and here she always bowed her head).  She had seen the light
two years and three months ago.  Now she did not envy women like
Clarissa Dalloway; she pitied them.

She pitied and despised them from the bottom of her heart, as she
stood on the soft carpet, looking at the old engraving of a little
girl with a muff.  With all this luxury going on, what hope was
there for a better state of things?  Instead of lying on a sofa--
"My mother is resting," Elizabeth had said--she should have been in
a factory; behind a counter; Mrs. Dalloway and all the other fine
ladies!

Bitter and burning, Miss Kilman had turned into a church two years
three months ago.  She had heard the Rev. Edward Whittaker preach;
the boys sing; had seen the solemn lights descend, and whether it
was the music, or the voices (she herself when alone in the evening
found comfort in a violin; but the sound was excruciating; she had
no ear), the hot and turbulent feelings which boiled and surged in
her had been assuaged as she sat there, and she had wept copiously,
and gone to call on Mr. Whittaker at his private house in
Kensington.  It was the hand of God, he said.  The Lord had shown
her the way.  So now, whenever the hot and painful feelings boiled
within her, this hatred of Mrs. Dalloway, this grudge against the
world, she thought of God.  She thought of Mr. Whittaker.  Rage was
succeeded by calm.  A sweet savour filled her veins, her lips
parted, and, standing formidable upon the landing in her
mackintosh, she looked with steady and sinister serenity at Mrs.
Dalloway, who came out with her daughter.

Elizabeth said she had forgotten her gloves.  That was because Miss
Kilman and her mother hated each other.  She could not bear to see
them together.  She ran upstairs to find her gloves.

But Miss Kilman did not hate Mrs. Dalloway.  Turning her large
gooseberry-coloured eyes upon Clarissa, observing her small pink
face, her delicate body, her air of freshness and fashion, Miss
Kilman felt, Fool!  Simpleton!  You who have known neither sorrow
nor pleasure; who have trifled your life away!  And there rose in
her an overmastering desire to overcome her; to unmask her.  If she
could have felled her it would have eased her.  But it was not the
body; it was the soul and its mockery that she wished to subdue;
make feel her mastery.  If only she could make her weep; could ruin
her; humiliate her; bring her to her knees crying, You are right!
But this was God's will, not Miss Kilman's.  It was to be a
religious victory.  So she glared; so she glowered.

Clarissa was really shocked.  This a Christian--this woman!  This
woman had taken her daughter from her!  She in touch with invisible
presences!  Heavy, ugly, commonplace, without kindness or grace,
she know the meaning of life!

"You are taking Elizabeth to the Stores?" Mrs. Dalloway said.

Miss Kilman said she was.  They stood there.  Miss Kilman was not
going to make herself agreeable.  She had always earned her living.
Her knowledge of modern history was thorough in the extreme.  She
did out of her meagre income set aside so much for causes she
believed in; whereas this woman did nothing, believed nothing;
brought up her daughter--but here was Elizabeth, rather out of
breath, the beautiful girl.

So they were going to the Stores.  Odd it was, as Miss Kilman stood
there (and stand she did, with the power and taciturnity of some
prehistoric monster armoured for primeval warfare), how, second by
second, the idea of her diminished, how hatred (which was for
ideas, not people) crumbled, how she lost her malignity, her size,
became second by second merely Miss Kilman, in a mackintosh, whom
Heaven knows Clarissa would have liked to help.

At this dwindling of the monster, Clarissa laughed.  Saying good-
bye, she laughed.

Off they went together, Miss Kilman and Elizabeth, downstairs.

With a sudden impulse, with a violent anguish, for this woman was
taking her daughter from her, Clarissa leant over the bannisters
and cried out, "Remember the party!  Remember our party tonight!"

But Elizabeth had already opened the front door; there was a van
passing; she did not answer.

Love and religion! thought Clarissa, going back into the drawing-
room, tingling all over.  How detestable, how detestable they are!
For now that the body of Miss Kilman was not before her, it
overwhelmed her--the idea.  The cruelest things in the world, she
thought, seeing them clumsy, hot, domineering, hypocritical,
eavesdropping, jealous, infinitely cruel and unscrupulous, dressed
in a mackintosh coat, on the landing; love and religion.  Had she
ever tried to convert any one herself?  Did she not wish everybody
merely to be themselves?  And she watched out of the window the old
lady opposite climbing upstairs.  Let her climb upstairs if she
wanted to; let her stop; then let her, as Clarissa had often seen
her, gain her bedroom, part her curtains, and disappear again into
the background.  Somehow one respected that--that old woman looking
out of the window, quite unconscious that she was being watched.
There was something solemn in it--but love and religion would
destroy that, whatever it was, the privacy of the soul.  The odious
Kilman would destroy it.  Yet it was a sight that made her want to
cry.

Love destroyed too.  Everything that was fine, everything that was
true went.  Take Peter Walsh now.  There was a man, charming,
clever, with ideas about everything.  If you wanted to know about
Pope, say, or Addison, or just to talk nonsense, what people were
like, what things meant, Peter knew better than any one.  It was
Peter who had helped her; Peter who had lent her books.  But look
at the women he loved--vulgar, trivial, commonplace.  Think of
Peter in love--he came to see her after all these years, and what
did he talk about?  Himself.  Horrible passion! she thought.
Degrading passion! she thought, thinking of Kilman and her
Elizabeth walking to the Army and Navy Stores.

Big Ben struck the half-hour.

How extraordinary it was, strange, yes, touching, to see the old
lady (they had been neighbours ever so many years) move away from
the window, as if she were attached to that sound, that string.
Gigantic as it was, it had something to do with her.  Down, down,
into the midst of ordinary things the finger fell making the moment
solemn.  She was forced, so Clarissa imagined, by that sound, to
move, to go--but where?  Clarissa tried to follow her as she turned
and disappeared, and could still just see her white cap moving at
the back of the bedroom.  She was still there moving about at the
other end of the room.  Why creeds and prayers and mackintoshes?
when, thought Clarissa, that's the miracle, that's the mystery;
that old lady, she meant, whom she could see going from chest of
drawers to dressing-table.  She could still see her.  And the
supreme mystery which Kilman might say she had solved, or Peter
might say he had solved, but Clarissa didn't believe either of them
had the ghost of an idea of solving, was simply this: here was one
room; there another.  Did religion solve that, or love?

Love--but here the other clock, the clock which always struck two
minutes after Big Ben, came shuffling in with its lap full of odds
and ends, which it dumped down as if Big Ben were all very well
with his majesty laying down the law, so solemn, so just, but she
must remember all sorts of little things besides--Mrs. Marsham,
Ellie Henderson, glasses for ices--all sorts of little things came
flooding and lapping and dancing in on the wake of that solemn
stroke which lay flat like a bar of gold on the sea.  Mrs. Marsham,
Ellie Henderson, glasses for ices.  She must telephone now at once.

Volubly, troublously, the late clock sounded, coming in on the wake
of Big Ben, with its lap full of trifles.  Beaten up, broken up by
the assault of carriages, the brutality of vans, the eager advance
of myriads of angular men, of flaunting women, the domes and spires
of offices and hospitals, the last relics of this lap full of odds
and ends seemed to break, like the spray of an exhausted wave, upon
the body of Miss Kilman standing still in the street for a moment
to mutter "It is the flesh."

It was the flesh that she must control.  Clarissa Dalloway had
insulted her.  That she expected.  But she had not triumphed; she
had not mastered the flesh.  Ugly, clumsy, Clarissa Dalloway had
laughed at her for being that; and had revived the fleshly desires,
for she minded looking as she did beside Clarissa.  Nor could she
talk as she did.  But why wish to resemble her?  Why?  She despised
Mrs. Dalloway from the bottom of her heart.  She was not serious.
She was not good.  Her life was a tissue of vanity and deceit.  Yet
Doris Kilman had been overcome.  She had, as a matter of fact, very
nearly burst into tears when Clarissa Dalloway laughed at her.  "It
is the flesh, it is the flesh," she muttered (it being her habit to
talk aloud) trying to subdue this turbulent and painful feeling as
she walked down Victoria Street.  She prayed to God.  She could not
help being ugly; she could not afford to buy pretty clothes.
Clarissa Dalloway had laughed--but she would concentrate her mind
upon something else until she had reached the pillar-box.  At any
rate she had got Elizabeth.  But she would think of something else;
she would think of Russia; until she reached the pillar-box.

How nice it must be, she said, in the country, struggling, as Mr.
Whittaker had told her, with that violent grudge against the world
which had scorned her, sneered at her, cast her off, beginning with
this indignity--the infliction of her unlovable body which people
could not bear to see.  Do her hair as she might, her forehead
remained like an egg, bald, white.  No clothes suited her.  She
might buy anything.  And for a woman, of course, that meant never
meeting the opposite sex.  Never would she come first with any one.
Sometimes lately it had seemed to her that, except for Elizabeth,
her food was all that she lived for; her comforts; her dinner, her
tea; her hot-water bottle at night.  But one must fight; vanquish;
have faith in God.  Mr. Whittaker had said she was there for a
purpose.  But no one knew the agony!  He said, pointing to the
crucifix, that God knew.  But why should she have to suffer when
other women, like Clarissa Dalloway, escaped?  Knowledge comes
through suffering, said Mr. Whittaker.

She had passed the pillar-box, and Elizabeth had turned into the
cool brown tobacco department of the Army and Navy Stores while she
was still muttering to herself what Mr. Whittaker had said about
knowledge coming through suffering and the flesh.  "The flesh," she
muttered.

What department did she want?  Elizabeth interrupted her.

"Petticoats," she said abruptly, and stalked straight on to the
lift.

Up they went.  Elizabeth guided her this way and that; guided her
in her abstraction as if she had been a great child, an unwieldy
battleship.  There were the petticoats, brown, decorous, striped,
frivolous, solid, flimsy; and she chose, in her abstraction,
portentously, and the girl serving thought her mad.

Elizabeth rather wondered, as they did up the parcel, what Miss
Kilman was thinking.  They must have their tea, said Miss Kilman,
rousing, collecting herself.  They had their tea.

Elizabeth rather wondered whether Miss Kilman could be hungry.  It
was her way of eating, eating with intensity, then looking, again
and again, at a plate of sugared cakes on the table next them;
then, when a lady and a child sat down and the child took the cake,
could Miss Kilman really mind it?  Yes, Miss Kilman did mind it.
She had wanted that cake--the pink one.  The pleasure of eating was
almost the only pure pleasure left her, and then to be baffled even
in that!

When people are happy, they have a reserve, she had told Elizabeth,
upon which to draw, whereas she was like a wheel without a tyre
(she was fond of such metaphors), jolted by every pebble, so she
would say staying on after the lesson standing by the fire-place
with her bag of books, her "satchel," she called it, on a Tuesday
morning, after the lesson was over.  And she talked too about the
war.  After all, there were people who did not think the English
invariably right.  There were books.  There were meetings.  There
were other points of view.  Would Elizabeth like to come with her
to listen to So-and-so (a most extraordinary looking old man)?
Then Miss Kilman took her to some church in Kensington and they had
tea with a clergyman.  She had lent her books.  Law, medicine,
politics, all professions are open to women of your generation,
said Miss Kilman.  But for herself, her career was absolutely
ruined and was it her fault?  Good gracious, said Elizabeth, no.

And her mother would come calling to say that a hamper had come
from Bourton and would Miss Kilman like some flowers?  To Miss
Kilman she was always very, very nice, but Miss Kilman squashed the
flowers all in a bunch, and hadn't any small talk, and what
interested Miss Kilman bored her mother, and Miss Kilman and she
were terrible together; and Miss Kilman swelled and looked very
plain.  But then Miss Kilman was frightfully clever.  Elizabeth had
never thought about the poor.  They lived with everything they
wanted,--her mother had breakfast in bed every day; Lucy carried it
up; and she liked old women because they were Duchesses, and being
descended from some Lord.  But Miss Kilman said (one of those
Tuesday mornings when the lesson was over), "My grandfather kept an
oil and colour shop in Kensington."  Miss Kilman made one feel so
small.

Miss Kilman took another cup of tea.  Elizabeth, with her oriental
bearing, her inscrutable mystery, sat perfectly upright; no, she
did not want anything more.  She looked for her gloves--her white
gloves.  They were under the table.  Ah, but she must not go!  Miss
Kilman could not let her go! this youth, that was so beautiful,
this girl, whom she genuinely loved!  Her large hand opened and
shut on the table.

But perhaps it was a little flat somehow, Elizabeth felt.  And
really she would like to go.

But said Miss Kilman, "I've not quite finished yet."

Of course, then, Elizabeth would wait.  But it was rather stuffy in
here.

"Are you going to the party to-night?" Miss Kilman said.  Elizabeth
supposed she was going; her mother wanted her to go.  She must not
let parties absorb her, Miss Kilman said, fingering the last two
inches of a chocolate éclair.

She did not much like parties, Elizabeth said.  Miss Kilman opened
her mouth, slightly projected her chin, and swallowed down the last
inches of the chocolate éclair, then wiped her fingers, and washed
the tea round in her cup.

She was about to split asunder, she felt.  The agony was so
terrific.  If she could grasp her, if she could clasp her, if she
could make her hers absolutely and forever and then die; that was
all she wanted.  But to sit here, unable to think of anything to
say; to see Elizabeth turning against her; to be felt repulsive
even by her--it was too much; she could not stand it.  The thick
fingers curled inwards.

"I never go to parties," said Miss Kilman, just to keep Elizabeth
from going.  "People don't ask me to parties"--and she knew as she
said it that it was this egotism that was her undoing; Mr.
Whittaker had warned her; but she could not help it.  She had
suffered so horribly.  "Why should they ask me?" she said.  "I'm
plain, I'm unhappy."  She knew it was idiotic.  But it was all
those people passing--people with parcels who despised her, who
made her say it.  However, she was Doris Kilman.  She had her
degree.  She was a woman who had made her way in the world.  Her
knowledge of modern history was more than respectable.

"I don't pity myself," she said.  "I pity"--she meant to say "your
mother" but no, she could not, not to Elizabeth.  "I pity other
people," she said, "more."

Like some dumb creature who has been brought up to a gate for an
unknown purpose, and stands there longing to gallop away, Elizabeth
Dalloway sat silent.  Was Miss Kilman going to say anything more?

"Don't quite forget me," said Doris Kilman; her voice quivered.
Right away to the end of the field the dumb creature galloped in
terror.

The great hand opened and shut.

Elizabeth turned her head.  The waitress came.  One had to pay at
the desk, Elizabeth said, and went off, drawing out, so Miss Kilman
felt, the very entrails in her body, stretching them as she crossed
the room, and then, with a final twist, bowing her head very
politely, she went.

She had gone.  Miss Kilman sat at the marble table among the
éclairs, stricken once, twice, thrice by shocks of suffering.  She
had gone.  Mrs. Dalloway had triumphed.  Elizabeth had gone.
Beauty had gone, youth had gone.

So she sat.  She got up, blundered off among the little tables,
rocking slightly from side to side, and somebody came after her
with her petticoat, and she lost her way, and was hemmed in by
trunks specially prepared for taking to India; next got among the
accouchement sets, and baby linen; through all the commodities of
the world, perishable and permanent, hams, drugs, flowers,
stationery, variously smelling, now sweet, now sour she lurched;
saw herself thus lurching with her hat askew, very red in the face,
full length in a looking-glass; and at last came out into the
street.

The tower of Westminster Cathedral rose in front of her, the
habitation of God.  In the midst of the traffic, there was the
habitation of God.  Doggedly she set off with her parcel to that
other sanctuary, the Abbey, where, raising her hands in a tent
before her face, she sat beside those driven into shelter too; the
variously assorted worshippers, now divested of social rank, almost
of sex, as they raised their hands before their faces; but once
they removed them, instantly reverent, middle class, English men
and women, some of them desirous of seeing the wax works.

But Miss Kilman held her tent before her face.  Now she was
deserted; now rejoined.  New worshippers came in from the street to
replace the strollers, and still, as people gazed round and
shuffled past the tomb of the Unknown Warrior, still she barred her
eyes with her fingers and tried in this double darkness, for the
light in the Abbey was bodiless, to aspire above the vanities, the
desires, the commodities, to rid herself both of hatred and of
love.  Her hands twitched.  She seemed to struggle.  Yet to others
God was accessible and the path to Him smooth.  Mr. Fletcher,
retired, of the Treasury, Mrs. Gorham, widow of the famous K.C.,
approached Him simply, and having done their praying, leant back,
enjoyed the music (the organ pealed sweetly), and saw Miss Kilman
at the end of the row, praying, praying, and, being still on the
threshold of their underworld, thought of her sympathetically as a
soul haunting the same territory; a soul cut out of immaterial
substance; not a woman, a soul.

But Mr. Fletcher had to go.  He had to pass her, and being himself
neat as a new pin, could not help being a little distressed by the
poor lady's disorder; her hair down; her parcel on the floor.  She
did not at once let him pass.  But, as he stood gazing about him,
at the white marbles, grey window panes, and accumulated treasures
(for he was extremely proud of the Abbey), her largeness,
robustness, and power as she sat there shifting her knees from time
to time (it was so rough the approach to her God--so tough her
desires) impressed him, as they had impressed Mrs. Dalloway (she
could not get the thought of her out of her mind that afternoon),
the Rev. Edward Whittaker, and Elizabeth too.

And Elizabeth waited in Victoria Street for an omnibus.  It was so
nice to be out of doors.  She thought perhaps she need not go home
just yet.  It was so nice to be out in the air.  So she would get
on to an omnibus.  And already, even as she stood there, in her
very well cut clothes, it was beginning. . . .  People were
beginning to compare her to poplar trees, early dawn, hyacinths,
fawns, running water, and garden lilies; and it made her life a
burden to her, for she so much preferred being left alone to do
what she liked in the country, but they would compare her to
lilies, and she had to go to parties, and London was so dreary
compared with being alone in the country with her father and the
dogs.

Buses swooped, settled, were off--garish caravans, glistening with
red and yellow varnish.  But which should she get on to?  She had
no preferences.  Of course, she would not push her way.  She
inclined to be passive.  It was expression she needed, but her eyes
were fine, Chinese, oriental, and, as her mother said, with such
nice shoulders and holding herself so straight, she was always
charming to look at; and lately, in the evening especially, when
she was interested, for she never seemed excited, she looked almost
beautiful, very stately, very serene.  What could she be thinking?
Every man fell in love with her, and she was really awfully bored.
For it was beginning.  Her mother could see that--the compliments
were beginning.  That she did not care more about it--for instance
for her clothes--sometimes worried Clarissa, but perhaps it was as
well with all those puppies and guinea pigs about having distemper,
and it gave her a charm.  And now there was this odd friendship
with Miss Kilman.  Well, thought Clarissa about three o'clock in
the morning, reading Baron Marbot for she could not sleep, it
proves she has a heart.

Suddenly Elizabeth stepped forward and most competently boarded the
omnibus, in front of everybody.  She took a seat on top.  The
impetuous creature--a pirate--started forward, sprang away; she had
to hold the rail to steady herself, for a pirate it was, reckless,
unscrupulous, bearing down ruthlessly, circumventing dangerously,
boldly snatching a passenger, or ignoring a passenger, squeezing
eel-like and arrogant in between, and then rushing insolently all
sails spread up Whitehall.  And did Elizabeth give one thought to
poor Miss Kilman who loved her without jealousy, to whom she had
been a fawn in the open, a moon in a glade?  She was delighted to
be free.  The fresh air was so delicious.  It had been so stuffy in
the Army and Navy Stores.  And now it was like riding, to be
rushing up Whitehall; and to each movement of the omnibus the
beautiful body in the fawn-coloured coat responded freely like a
rider, like the figure-head of a ship, for the breeze slightly
disarrayed her; the heat gave her cheeks the pallor of white
painted wood; and her fine eyes, having no eyes to meet, gazed
ahead, blank, bright, with the staring incredible innocence of
sculpture.

It was always talking about her own sufferings that made Miss
Kilman so difficult.  And was she right?  If it was being on
committees and giving up hours and hours every day (she hardly ever
saw him in London) that helped the poor, her father did that,
goodness knows,--if that was what Miss Kilman meant about being a
Christian; but it was so difficult to say.  Oh, she would like to
go a little further.  Another penny was it to the Strand?  Here was
another penny then.  She would go up the Strand.

She liked people who were ill.  And every profession is open to the
women of your generation, said Miss Kilman.  So she might be a
doctor.  She might be a farmer.  Animals are often ill.  She might
own a thousand acres and have people under her.  She would go and
see them in their cottages.  This was Somerset House.  One might be
a very good farmer--and that, strangely enough though Miss Kilman
had her share in it, was almost entirely due to Somerset House.  It
looked so splendid, so serious, that great grey building.  And she
liked the feeling of people working.  She liked those churches,
like shapes of grey paper, breasting the stream of the Strand.  It
was quite different here from Westminster, she thought, getting off
at Chancery Lane.  It was so serious; it was so busy.  In short,
she would like to have a profession.  She would become a doctor, a
farmer, possibly go into Parliament, if she found it necessary, all
because of the Strand.

The feet of those people busy about their activities, hands putting
stone to stone, minds eternally occupied not with trivial
chatterings (comparing women to poplars--which was rather exciting,
of course, but very silly), but with thoughts of ships, of
business, of law, of administration, and with it all so stately
(she was in the Temple), gay (there was the river), pious (there
was the Church), made her quite determined, whatever her mother
might say, to become either a farmer or a doctor.  But she was, of
course, rather lazy.

And it was much better to say nothing about it.  It seemed so
silly.  It was the sort of thing that did sometimes happen, when
one was alone--buildings without architects' names, crowds of
people coming back from the city having more power than single
clergymen in Kensington, than any of the books Miss Kilman had lent
her, to stimulate what lay slumbrous, clumsy, and shy on the mind's
sandy floor to break surface, as a child suddenly stretches its
arms; it was just that, perhaps, a sigh, a stretch of the arms, an
impulse, a revelation, which has its effects for ever, and then
down again it went to the sandy floor.  She must go home.  She must
dress for dinner.  But what was the time?--where was a clock?

She looked up Fleet Street.  She walked just a little way towards
St. Paul's, shyly, like some one penetrating on tiptoe, exploring a
strange house by night with a candle, on edge lest the owner should
suddenly fling wide his bedroom door and ask her business, nor did
she dare wander off into queer alleys, tempting bye-streets, any
more than in a strange house open doors which might be bedroom
doors, or sitting-room doors, or lead straight to the larder.  For
no Dalloways came down the Strand daily; she was a pioneer, a
stray, venturing, trusting.

In many ways, her mother felt, she was extremely immature, like a
child still, attached to dolls, to old slippers; a perfect baby;
and that was charming.  But then, of course, there was in the
Dalloway family the tradition of public service.  Abbesses,
principals, head mistresses, dignitaries, in the republic of women--
without being brilliant, any of them, they were that.  She
penetrated a little further in the direction of St. Paul's.  She
liked the geniality, sisterhood, motherhood, brotherhood of this
uproar.  It seemed to her good.  The noise was tremendous; and
suddenly there were trumpets (the unemployed) blaring, rattling
about in the uproar; military music; as if people were marching;
yet had they been dying--had some woman breathed her last and
whoever was watching, opening the window of the room where she had
just brought off that act of supreme dignity, looked down on Fleet
Street, that uproar, that military music would have come triumphing
up to him, consolatory, indifferent.

It was not conscious.  There was no recognition in it of one
fortune, or fate, and for that very reason even to those dazed with
watching for the last shivers of consciousness on the faces of the
dying, consoling.  Forgetfulness in people might wound, their
ingratitude corrode, but this voice, pouring endlessly, year in
year out, would take whatever it might be; this vow; this van; this
life; this procession, would wrap them all about and carry them on,
as in the rough stream of a glacier the ice holds a splinter of
bone, a blue petal, some oak trees, and rolls them on.

But it was later than she thought.  Her mother would not like her
to be wandering off alone like this.  She turned back down the
Strand.

A puff of wind (in spite of the heat, there was quite a wind) blew
a thin black veil over the sun and over the Strand.  The faces
faded; the omnibuses suddenly lost their glow.  For although the
clouds were of mountainous white so that one could fancy hacking
hard chips off with a hatchet, with broad golden slopes, lawns of
celestial pleasure gardens, on their flanks, and had all the
appearance of settled habitations assembled for the conference of
gods above the world, there was a perpetual movement among them.
Signs were interchanged, when, as if to fulfil some scheme arranged
already, now a summit dwindled, now a whole block of pyramidal size
which had kept its station inalterably advanced into the midst or
gravely led the procession to fresh anchorage.  Fixed though they
seemed at their posts, at rest in perfect unanimity, nothing could
be fresher, freer, more sensitive superficially than the snow-white
or gold-kindled surface; to change, to go, to dismantle the solemn
assemblage was immediately possible; and in spite of the grave
fixity, the accumulated robustness and solidity, now they struck
light to the earth, now darkness.

Calmly and competently, Elizabeth Dalloway mounted the Westminster
omnibus.

Going and coming, beckoning, signalling, so the light and shadow
which now made the wall grey, now the bananas bright yellow, now
made the Strand grey, now made the omnibuses bright yellow, seemed
to Septimus Warren Smith lying on the sofa in the sitting-room;
watching the watery gold glow and fade with the astonishing
sensibility of some live creature on the roses, on the wall-paper.
Outside the trees dragged their leaves like nets through the depths
of the air; the sound of water was in the room and through the
waves came the voices of birds singing.  Every power poured its
treasures on his head, and his hand lay there on the back of the
sofa, as he had seen his hand lie when he was bathing, floating, on
the top of the waves, while far away on shore he heard dogs barking
and barking far away.  Fear no more, says the heart in the body;
fear no more.

He was not afraid.  At every moment Nature signified by some
laughing hint like that gold spot which went round the wall--there,
there, there--her determination to show, by brandishing her plumes,
shaking her tresses, flinging her mantle this way and that,
beautifully, always beautifully, and standing close up to breathe
through her hollowed hands Shakespeare's words, her meaning.

Rezia, sitting at the table twisting a hat in her hands, watched
him; saw him smiling.  He was happy then.  But she could not bear
to see him smiling.  It was not marriage; it was not being one's
husband to look strange like that, always to be starting, laughing,
sitting hour after hour silent, or clutching her and telling her to
write.  The table drawer was full of those writings; about war;
about Shakespeare; about great discoveries; how there is no death.
Lately he had become excited suddenly for no reason (and both Dr.
Holmes and Sir William Bradshaw said excitement was the worst thing
for him), and waved his hands and cried out that he knew the truth!
He knew everything!  That man, his friend who was killed, Evans,
had come, he said.  He was singing behind the screen.  She wrote it
down just as he spoke it.  Some things were very beautiful; others
sheer nonsense.  And he was always stopping in the middle, changing
his mind; wanting to add something; hearing something new;
listening with his hand up.

But she heard nothing.

And once they found the girl who did the room reading one of these
papers in fits of laughter.  It was a dreadful pity.  For that made
Septimus cry out about human cruelty--how they tear each other to
pieces.  The fallen, he said, they tear to pieces.  "Holmes is on
us," he would say, and he would invent stories about Holmes; Holmes
eating porridge; Holmes reading Shakespeare--making himself roar
with laughter or rage, for Dr. Holmes seemed to stand for something
horrible to him.  "Human nature," he called him.  Then there were
the visions.  He was drowned, he used to say, and lying on a cliff
with the gulls screaming over him.  He would look over the edge of
the sofa down into the sea.  Or he was hearing music.  Really it
was only a barrel organ or some man crying in the street.  But
"Lovely!" he used to cry, and the tears would run down his cheeks,
which was to her the most dreadful thing of all, to see a man like
Septimus, who had fought, who was brave, crying.  And he would lie
listening until suddenly he would cry that he was falling down,
down into the flames!  Actually she would look for flames, it was
so vivid.  But there was nothing.  They were alone in the room.  It
was a dream, she would tell him and so quiet him at last, but
sometimes she was frightened too.  She sighed as she sat sewing.

Her sigh was tender and enchanting, like the wind outside a wood in
the evening.  Now she put down her scissors; now she turned to take
something from the table.  A little stir, a little crinkling, a
little tapping built up something on the table there, where she sat
sewing.  Through his eyelashes he could see her blurred outline;
her little black body; her face and hands; her turning movements at
the table, as she took up a reel, or looked (she was apt to lose
things) for her silk.  She was making a hat for Mrs. Filmer's
married daughter, whose name was--he had forgotten her name.

"What is the name of Mrs. Filmer's married daughter?" he asked.

"Mrs. Peters," said Rezia.  She was afraid it was too small, she
said, holding it before her.  Mrs. Peters was a big woman; but she
did not like her.  It was only because Mrs. Filmer had been so good
to them.  "She gave me grapes this morning," she said--that Rezia
wanted to do something to show that they were grateful.  She had
come into the room the other evening and found Mrs. Peters, who
thought they were out, playing the gramophone.

"Was it true?" he asked.  She was playing the gramophone?  Yes; she
had told him about it at the time; she had found Mrs. Peters
playing the gramophone.

He began, very cautiously, to open his eyes, to see whether a
gramophone was really there.  But real things--real things were too
exciting.  He must be cautious.  He would not go mad.  First he
looked at the fashion papers on the lower shelf, then, gradually at
the gramophone with the green trumpet.  Nothing could be more
exact.  And so, gathering courage, he looked at the sideboard; the
plate of bananas; the engraving of Queen Victoria and the Prince
Consort; at the mantelpiece, with the jar of roses.  None of these
things moved.  All were still; all were real.

"She is a woman with a spiteful tongue," said Rezia.

"What does Mr. Peters do?" Septimus asked.

"Ah," said Rezia, trying to remember.  She thought Mrs. Filmer had
said that he travelled for some company.  "Just now he is in Hull,"
she said.

"Just now!"  She said that with her Italian accent.  She said that
herself.  He shaded his eyes so that he might see only a little of
her face at a time, first the chin, then the nose, then the
forehead, in case it were deformed, or had some terrible mark on
it.  But no, there she was, perfectly natural, sewing, with the
pursed lips that women have, the set, the melancholy expression,
when sewing.  But there was nothing terrible about it, he assured
himself, looking a second time, a third time at her face, her
hands, for what was frightening or disgusting in her as she sat
there in broad daylight, sewing?  Mrs. Peters had a spiteful
tongue.  Mr. Peters was in Hull.  Why then rage and prophesy?  Why
fly scourged and outcast?  Why be made to tremble and sob by the
clouds?  Why seek truths and deliver messages when Rezia sat
sticking pins into the front of her dress, and Mr. Peters was in
Hull?  Miracles, revelations, agonies, loneliness, falling through
the sea, down, down into the flames, all were burnt out, for he had
a sense, as he watched Rezia trimming the straw hat for Mrs.
Peters, of a coverlet of flowers.

"It's too small for Mrs. Peters," said Septimus.

For the first time for days he was speaking as he used to do!  Of
course it was--absurdly small, she said.  But Mrs. Peters had
chosen it.

He took it out of her hands.  He said it was an organ grinder's
monkey's hat.

How it rejoiced her that!  Not for weeks had they laughed like this
together, poking fun privately like married people.  What she meant
was that if Mrs. Filmer had come in, or Mrs. Peters or anybody they
would not have understood what she and Septimus were laughing at.

"There," she said, pinning a rose to one side of the hat.  Never
had she felt so happy!  Never in her life!

But that was still more ridiculous, Septimus said.  Now the poor
woman looked like a pig at a fair.  (Nobody ever made her laugh as
Septimus did.)

What had she got in her work-box?  She had ribbons and beads,
tassels, artificial flowers.  She tumbled them out on the table.
He began putting odd colours together--for though he had no
fingers, could not even do up a parcel, he had a wonderful eye, and
often he was right, sometimes absurd, of course, but sometimes
wonderfully right.

"She shall have a beautiful hat!" he murmured, taking up this and
that, Rezia kneeling by his side, looking over his shoulder.  Now
it was finished--that is to say the design; she must stitch it
together.  But she must be very, very careful, he said, to keep it
just as he had made it.

So she sewed.  When she sewed, he thought, she made a sound like a
kettle on the hob; bubbling, murmuring, always busy, her strong
little pointed fingers pinching and poking; her needle flashing
straight.  The sun might go in and out, on the tassels, on the
wall-paper, but he would wait, he thought, stretching out his feet,
looking at his ringed sock at the end of the sofa; he would wait in
this warm place, this pocket of still air, which one comes on at
the edge of a wood sometimes in the evening, when, because of a
fall in the ground, or some arrangement of the trees (one must be
scientific above all, scientific), warmth lingers, and the air
buffets the cheek like the wing of a bird.

"There it is," said Rezia, twirling Mrs. Peters' hat on the tips of
her fingers.  "That'll do for the moment.  Later . . ." her
sentence bubbled away drip, drip, drip, like a contented tap left
running.

It was wonderful.  Never had he done anything which made him feel
so proud.  It was so real, it was so substantial, Mrs. Peters' hat.

"Just look at it," he said.

Yes, it would always make her happy to see that hat.  He had become
himself then, he had laughed then.  They had been alone together.
Always she would like that hat.

He told her to try it on.

"But I must look so queer!" she cried, running over to the glass
and looking first this side then that.  Then she snatched it off
again, for there was a tap at the door.  Could it be Sir William
Bradshaw?  Had he sent already?

No! it was only the small girl with the evening paper.

What always happened, then happened--what happened every night of
their lives.  The small girl sucked her thumb at the door; Rezia
went down on her knees; Rezia cooed and kissed; Rezia got a bag of
sweets out of the table drawer.  For so it always happened.  First
one thing, then another.  So she built it up, first one thing and
then another.  Dancing, skipping, round and round the room they
went.  He took the paper.  Surrey was all out, he read.  There was
a heat wave.  Rezia repeated:  Surrey was all out.  There was a
heat wave, making it part of the game she was playing with Mrs.
Filmer's grandchild, both of them laughing, chattering at the same
time, at their game.  He was very tired.  He was very happy.  He
would sleep.  He shut his eyes.  But directly he saw nothing the
sounds of the game became fainter and stranger and sounded like the
cries of people seeking and not finding, and passing further and
further away.  They had lost him!

He started up in terror.  What did he see?  The plate of bananas on
the sideboard.  Nobody was there (Rezia had taken the child to its
mother.  It was bedtime).  That was it: to be alone forever.  That
was the doom pronounced in Milan when he came into the room and saw
them cutting out buckram shapes with their scissors; to be alone
forever.

He was alone with the sideboard and the bananas.  He was alone,
exposed on this bleak eminence, stretched out--but not on a hill-
top; not on a crag; on Mrs. Filmer's sitting-room sofa.  As for the
visions, the faces, the voices of the dead, where were they?  There
was a screen in front of him, with black bulrushes and blue
swallows.  Where he had once seen mountains, where he had seen
faces, where he had seen beauty, there was a screen.

"Evans!" he cried.  There was no answer.  A mouse had squeaked, or
a curtain rustled.  Those were the voices of the dead.  The screen,
the coalscuttle, the sideboard remained to him.  Let him then face
the screen, the coal-scuttle and the sideboard . . . but Rezia
burst into the room chattering.

Some letter had come.  Everybody's plans were changed.  Mrs. Filmer
would not be able to go to Brighton after all.  There was no time
to let Mrs. Williams know, and really Rezia thought it very, very
annoying, when she caught sight of the hat and thought . . .
perhaps . . . she . . . might just make a little. . . .  Her voice
died out in contented melody.

"Ah, damn!" she cried (it was a joke of theirs, her swearing), the
needle had broken.  Hat, child, Brighton, needle.  She built it up;
first one thing, then another, she built it up, sewing.

She wanted him to say whether by moving the rose she had improved
the hat.  She sat on the end of the sofa.

They were perfectly happy now, she said, suddenly, putting the hat
down.  For she could say anything to him now.  She could say
whatever came into her head.  That was almost the first thing she
had felt about him, that night in the café when he had come in with
his English friends.  He had come in, rather shyly, looking round
him, and his hat had fallen when he hung it up.  That she could
remember.  She knew he was English, though not one of the large
Englishmen her sister admired, for he was always thin; but he had a
beautiful fresh colour; and with his big nose, his bright eyes, his
way of sitting a little hunched made her think, she had often told
him, of a young hawk, that first evening she saw him, when they
were playing dominoes, and he had come in--of a young hawk; but
with her he was always very gentle.  She had never seen him wild or
drunk, only suffering sometimes through this terrible war, but even
so, when she came in, he would put it all away.  Anything, anything
in the whole world, any little bother with her work, anything that
struck her to say she would tell him, and he understood at once.
Her own family even were not the same.  Being older than she was
and being so clever--how serious he was, wanting her to read
Shakespeare before she could even read a child's story in English!--
being so much more experienced, he could help her.  And she too
could help him.

But this hat now.  And then (it was getting late) Sir William
Bradshaw.

She held her hands to her head, waiting for him to say did he like
the hat or not, and as she sat there, waiting, looking down, he
could feel her mind, like a bird, falling from branch to branch,
and always alighting, quite rightly; he could follow her mind, as
she sat there in one of those loose lax poses that came to her
naturally and, if he should say anything, at once she smiled, like
a bird alighting with all its claws firm upon the bough.

But he remembered Bradshaw said, "The people we are most fond of
are not good for us when we are ill."  Bradshaw said, he must be
taught to rest.  Bradshaw said they must be separated.

"Must," "must," why "must"?  What power had Bradshaw over him?
"What right has Bradshaw to say 'must' to me?" he demanded.

"It is because you talked of killing yourself," said Rezia.
(Mercifully, she could now say anything to Septimus.)

So he was in their power!  Holmes and Bradshaw were on him!  The
brute with the red nostrils was snuffing into every secret place!
"Must" it could say!  Where were his papers? the things he had
written?

She brought him his papers, the things he had written, things she
had written for him.  She tumbled them out on to the sofa.  They
looked at them together.  Diagrams, designs, little men and women
brandishing sticks for arms, with wings--were they?--on their
backs; circles traced round shillings and sixpences--the suns and
stars; zigzagging precipices with mountaineers ascending roped
together, exactly like knives and forks; sea pieces with little
faces laughing out of what might perhaps be waves: the map of the
world.  Burn them! he cried.  Now for his writings; how the dead
sing behind rhododendron bushes; odes to Time; conversations with
Shakespeare; Evans, Evans, Evans--his messages from the dead; do
not cut down trees; tell the Prime Minister.  Universal love: the
meaning of the world.  Burn them! he cried.

But Rezia laid her hands on them.  Some were very beautiful, she
thought.  She would tie them up (for she had no envelope) with a
piece of silk.

Even if they took him, she said, she would go with him.  They could
not separate them against their wills, she said.

Shuffling the edges straight, she did up the papers, and tied the
parcel almost without looking, sitting beside him, he thought, as
if all her petals were about her.  She was a flowering tree; and
through her branches looked out the face of a lawgiver, who had
reached a sanctuary where she feared no one; not Holmes; not
Bradshaw; a miracle, a triumph, the last and greatest.  Staggering
he saw her mount the appalling staircase, laden with Holmes and
Bradshaw, men who never weighed less than eleven stone six, who
sent their wives to Court, men who made ten thousand a year and
talked of proportion; who different in their verdicts (for Holmes
said one thing, Bradshaw another), yet judges they were; who mixed
the vision and the sideboard; saw nothing clear, yet ruled, yet
inflicted.  "Must" they said.  Over them she triumphed.

"There!" she said.  The papers were tied up.  No one should get at
them.  She would put them away.

And, she said, nothing should separate them.  She sat down beside
him and called him by the name of that hawk or crow which being
malicious and a great destroyer of crops was precisely like him.
No one could separate them, she said.

Then she got up to go into the bedroom to pack their things, but
hearing voices downstairs and thinking that Dr. Holmes had perhaps
called, ran down to prevent him coming up.

Septimus could hear her talking to Holmes on the staircase.

"My dear lady, I have come as a friend," Holmes was saying.

"No.  I will not allow you to see my husband," she said.

He could see her, like a little hen, with her wings spread barring
his passage.  But Holmes persevered.

"My dear lady, allow me . . ." Holmes said, putting her aside
(Holmes was a powerfully built man).

Holmes was coming upstairs.  Holmes would burst open the door.
Holmes would say "In a funk, eh?"  Holmes would get him.  But no;
not Holmes; not Bradshaw.  Getting up rather unsteadily, hopping
indeed from foot to foot, he considered Mrs. Filmer's nice clean
bread knife with "Bread" carved on the handle.  Ah, but one mustn't
spoil that.  The gas fire?  But it was too late now.  Holmes was
coming.  Razors he might have got, but Rezia, who always did that
sort of thing, had packed them.  There remained only the window,
the large Bloomsbury-lodging house window, the tiresome, the
troublesome, and rather melodramatic business of opening the window
and throwing himself out.  It was their idea of tragedy, not his or
Rezia's (for she was with him).  Holmes and Bradshaw like that sort
of thing.  (He sat on the sill.)  But he would wait till the very
last moment.  He did not want to die.  Life was good.  The sun hot.
Only human beings--what did THEY want?  Coming down the staircase
opposite an old man stopped and stared at him.  Holmes was at the
door.  "I'll give it you!" he cried, and flung himself vigorously,
violently down on to Mrs. Filmer's area railings.

"The coward!" cried Dr. Holmes, bursting the door open.  Rezia ran
to the window, she saw; she understood.  Dr. Holmes and Mrs. Filmer
collided with each other.  Mrs. Filmer flapped her apron and made
her hide her eyes in the bedroom.  There was a great deal of
running up and down stairs.  Dr. Holmes came in--white as a sheet,
shaking all over, with a glass in his hand.  She must be brave and
drink something, he said (What was it?  Something sweet), for her
husband was horribly mangled, would not recover consciousness, she
must not see him, must be spared as much as possible, would have
the inquest to go through, poor young woman.  Who could have
foretold it?  A sudden impulse, no one was in the least to blame
(he told Mrs. Filmer).  And why the devil he did it, Dr. Holmes
could not conceive.

It seemed to her as she drank the sweet stuff that she was opening
long windows, stepping out into some garden.  But where?  The clock
was striking--one, two, three: how sensible the sound was; compared
with all this thumping and whispering; like Septimus himself.  She
was falling asleep.  But the clock went on striking, four, five,
six and Mrs. Filmer waving her apron (they wouldn't bring the body
in here, would they?) seemed part of that garden; or a flag.  She
had once seen a flag slowly rippling out from a mast when she
stayed with her aunt at Venice.  Men killed in battle were thus
saluted, and Septimus had been through the War.  Of her memories,
most were happy.

She put on her hat, and ran through cornfields--where could it have
been?--on to some hill, somewhere near the sea, for there were
ships, gulls, butterflies; they sat on a cliff.  In London too,
there they sat, and, half dreaming, came to her through the bedroom
door, rain falling, whisperings, stirrings among dry corn, the
caress of the sea, as it seemed to her, hollowing them in its
arched shell and murmuring to her laid on shore, strewn she felt,
like flying flowers over some tomb.

"He is dead," she said, smiling at the poor old woman who guarded
her with her honest light-blue eyes fixed on the door.  (They
wouldn't bring him in here, would they?)  But Mrs. Filmer pooh-
poohed.  Oh no, oh no!  They were carrying him away now.  Ought she
not to be told?  Married people ought to be together, Mrs. Filmer
thought.  But they must do as the doctor said.

"Let her sleep," said Dr. Holmes, feeling her pulse.  She saw the
large outline of his body standing dark against the window.  So
that was Dr. Holmes.

One of the triumphs of civilisation, Peter Walsh thought.  It is
one of the triumphs of civilisation, as the light high bell of the
ambulance sounded.  Swiftly, cleanly the ambulance sped to the
hospital, having picked up instantly, humanely, some poor devil;
some one hit on the head, struck down by disease, knocked over
perhaps a minute or so ago at one of these crossings, as might
happen to oneself.  That was civilisation.  It struck him coming
back from the East--the efficiency, the organisation, the communal
spirit of London.  Every cart or carriage of its own accord drew
aside to let the ambulance pass.  Perhaps it was morbid; or was it
not touching rather, the respect which they showed this ambulance
with its victim inside--busy men hurrying home yet instantly
bethinking them as it passed of some wife; or presumably how easily
it might have been them there, stretched on a shelf with a doctor
and a nurse. . . .  Ah, but thinking became morbid, sentimental,
directly one began conjuring up doctors, dead bodies; a little glow
of pleasure, a sort of lust too over the visual impression warned
one not to go on with that sort of thing any more--fatal to art,
fatal to friendship.  True.  And yet, thought Peter Walsh, as the
ambulance turned the corner though the light high bell could be
heard down the next street and still farther as it crossed the
Tottenham Court Road, chiming constantly, it is the privilege of
loneliness; in privacy one may do as one chooses.  One might weep
if no one saw.  It had been his undoing--this susceptibility--in
Anglo-Indian society; not weeping at the right time, or laughing
either.  I have that in me, he thought standing by the pillar-box,
which could now dissolve in tears.  Why, Heaven knows.  Beauty of
some sort probably, and the weight of the day, which beginning with
that visit to Clarissa had exhausted him with its heat, its
intensity, and the drip, drip, of one impression after another down
into that cellar where they stood, deep, dark, and no one would
ever know.  Partly for that reason, its secrecy, complete and
inviolable, he had found life like an unknown garden, full of turns
and corners, surprising, yes; really it took one's breath away,
these moments; there coming to him by the pillar-box opposite the
British Museum one of them, a moment, in which things came
together; this ambulance; and life and death.  It was as if he were
sucked up to some very high roof by that rush of emotion and the
rest of him, like a white shell-sprinkled beach, left bare.  It had
been his undoing in Anglo-Indian society--this susceptibility.

Clarissa once, going on top of an omnibus with him somewhere,
Clarissa superficially at least, so easily moved, now in despair,
now in the best of spirits, all aquiver in those days and such good
company, spotting queer little scenes, names, people from the top
of a bus, for they used to explore London and bring back bags full
of treasures from the Caledonian market--Clarissa had a theory in
those days--they had heaps of theories, always theories, as
young people have.  It was to explain the feeling they had of
dissatisfaction; not knowing people; not being known.  For how
could they know each other?  You met every day; then not for six
months, or years.  It was unsatisfactory, they agreed, how little
one knew people.  But she said, sitting on the bus going up
Shaftesbury Avenue, she felt herself everywhere; not "here, here,
here"; and she tapped the back of the seat; but everywhere.  She
waved her hand, going up Shaftesbury Avenue.  She was all that.  So
that to know her, or any one, one must seek out the people who
completed them; even the places.  Odd affinities she had with
people she had never spoken to, some woman in the street, some man
behind a counter--even trees, or barns.  It ended in a transcendental
theory which, with her horror of death, allowed her to believe, or
say that she believed (for all her scepticism), that since our
apparitions, the part of us which appears, are so momentary compared
with the other, the unseen part of us, which spreads wide, the
unseen might survive, be recovered somehow attached to this person
or that, or even haunting certain places after death . . . perhaps--
perhaps.

Looking back over that long friendship of almost thirty years her
theory worked to this extent.  Brief, broken, often painful as
their actual meetings had been what with his absences and
interruptions (this morning, for instance, in came Elizabeth, like
a long-legged colt, handsome, dumb, just as he was beginning to
talk to Clarissa) the effect of them on his life was immeasurable.
There was a mystery about it.  You were given a sharp, acute,
uncomfortable grain--the actual meeting; horribly painful as often
as not; yet in absence, in the most unlikely places, it would
flower out, open, shed its scent, let you touch, taste, look about
you, get the whole feel of it and understanding, after years of
lying lost.  Thus she had come to him; on board ship; in the
Himalayas; suggested by the oddest things (so Sally Seton,
generous, enthusiastic goose! thought of HIM when she saw blue
hydrangeas).  She had influenced him more than any person he had
ever known.  And always in this way coming before him without his
wishing it, cool, lady-like, critical; or ravishing, romantic,
recalling some field or English harvest.  He saw her most often
in the country, not in London.  One scene after another at
Bourton. . . .

He had reached his hotel.  He crossed the hall, with its mounds of
reddish chairs and sofas, its spike-leaved, withered-looking
plants.  He got his key off the hook.  The young lady handed him
some letters.  He went upstairs--he saw her most often at Bourton,
in the late summer, when he stayed there for a week, or fortnight
even, as people did in those days.  First on top of some hill there
she would stand, hands clapped to her hair, her cloak blowing out,
pointing, crying to them--she saw the Severn beneath.  Or in a
wood, making the kettle boil--very ineffective with her fingers;
the smoke curtseying, blowing in their faces; her little pink face
showing through; begging water from an old woman in a cottage, who
came to the door to watch them go.  They walked always; the others
drove.  She was bored driving, disliked all animals, except that
dog.  They tramped miles along roads.  She would break off to get
her bearings, pilot him back across country; and all the time they
argued, discussed poetry, discussed people, discussed politics (she
was a Radical then); never noticing a thing except when she
stopped, cried out at a view or a tree, and made him look with her;
and so on again, through stubble fields, she walking ahead, with a
flower for her aunt, never tired of walking for all her delicacy;
to drop down on Bourton in the dusk.  Then, after dinner, old
Breitkopf would open the piano and sing without any voice, and they
would lie sunk in arm-chairs, trying not to laugh, but always
breaking down and laughing, laughing--laughing at nothing.
Breitkopf was supposed not to see.  And then in the morning,
flirting up and down like a wagtail in front of the house. . . .

Oh it was a letter from her!  This blue envelope; that was her
hand.  And he would have to read it.  Here was another of those
meetings, bound to be painful!  To read her letter needed the devil
of an effort.  "How heavenly it was to see him.  She must tell him
that."  That was all.

But it upset him.  It annoyed him.  He wished she hadn't written
it.  Coming on top of his thoughts, it was like a nudge in the
ribs.  Why couldn't she let him be?  After all, she had married
Dalloway, and lived with him in perfect happiness all these years.

These hotels are not consoling places.  Far from it.  Any number of
people had hung up their hats on those pegs.  Even the flies, if
you thought of it, had settled on other people's noses.  As for the
cleanliness which hit him in the face, it wasn't cleanliness, so
much as bareness, frigidity; a thing that had to be.  Some arid
matron made her rounds at dawn sniffing, peering, causing blue-
nosed maids to scour, for all the world as if the next visitor were
a joint of meat to be served on a perfectly clean platter.  For
sleep, one bed; for sitting in, one armchair; for cleaning one's
teeth and shaving one's chin, one tumbler, one looking-glass.
Books, letters, dressing-gown, slipped about on the impersonality
of the horsehair like incongruous impertinences.  And it was
Clarissa's letter that made him see all this.  "Heavenly to see
you.  She must say so!"  He folded the paper; pushed it away;
nothing would induce him to read it again!

To get that letter to him by six o'clock she must have sat down and
written it directly he left her; stamped it; sent somebody to the
post.  It was, as people say, very like her.  She was upset by his
visit.  She had felt a great deal; had for a moment, when she
kissed his hand, regretted, envied him even, remembered possibly
(for he saw her look it) something he had said--how they would
change the world if she married him perhaps; whereas, it was this;
it was middle age; it was mediocrity; then forced herself with her
indomitable vitality to put all that aside, there being in her a
thread of life which for toughness, endurance, power to overcome
obstacles, and carry her triumphantly through he had never known
the like of.  Yes; but there would come a reaction directly he left
the room.  She would be frightfully sorry for him; she would think
what in the world she could do to give him pleasure (short always
of the one thing) and he could see her with the tears running down
her cheeks going to her writing-table and dashing off that one line
which he was to find greeting him. . . .  "Heavenly to see you!"
And she meant it.

Peter Walsh had now unlaced his boots.

But it would not have been a success, their marriage.  The other
thing, after all, came so much more naturally.

It was odd; it was true; lots of people felt it.  Peter Walsh, who
had done just respectably, filled the usual posts adequately, was
liked, but thought a little cranky, gave himself airs--it was odd
that HE should have had, especially now that his hair was grey, a
contented look; a look of having reserves.  It was this that made
him attractive to women who liked the sense that he was not
altogether manly.  There was something unusual about him, or
something behind him.  It might be that he was bookish--never came
to see you without taking up the book on the table (he was now
reading, with his bootlaces trailing on the floor); or that he was
a gentleman, which showed itself in the way he knocked the ashes
out of his pipe, and in his manners of course to women.  For it was
very charming and quite ridiculous how easily some girl without a
grain of sense could twist him round her finger.  But at her own
risk.  That is to say, though he might be ever so easy, and indeed
with his gaiety and good-breeding fascinating to be with, it was
only up to a point.  She said something--no, no; he saw through
that.  He wouldn't stand that--no, no.  Then he could shout and
rock and hold his sides together over some joke with men.  He was
the best judge of cooking in India.  He was a man.  But not the
sort of man one had to respect--which was a mercy; not like Major
Simmons, for instance; not in the least like that, Daisy thought,
when, in spite of her two small children, she used to compare them.

He pulled off his boots.  He emptied his pockets.  Out came with
his pocket-knife a snapshot of Daisy on the verandah; Daisy all in
white, with a fox-terrier on her knee; very charming, very dark;
the best he had ever seen of her.  It did come, after all so
naturally; so much more naturally than Clarissa.  No fuss.  No
bother.  No finicking and fidgeting.  All plain sailing.  And the
dark, adorably pretty girl on the verandah exclaimed (he could hear
her).  Of course, of course she would give him everything! she
cried (she had no sense of discretion) everything he wanted! she
cried, running to meet him, whoever might be looking.  And she was
only twenty-four.  And she had two children.  Well, well!

Well indeed he had got himself into a mess at his age.  And it came
over him when he woke in the night pretty forcibly.  Suppose they
did marry?  For him it would be all very well, but what about her?
Mrs. Burgess, a good sort and no chatterbox, in whom he had
confided, thought this absence of his in England, ostensibly to see
lawyers might serve to make Daisy reconsider, think what it meant.
It was a question of her position, Mrs. Burgess said; the social
barrier; giving up her children.  She'd be a widow with a past one
of these days, draggling about in the suburbs, or more likely,
indiscriminate (you know, she said, what such women get like, with
too much paint).  But Peter Walsh pooh-poohed all that.  He didn't
mean to die yet.  Anyhow she must settle for herself; judge for
herself, he thought, padding about the room in his socks, smoothing
out his dress-shirt, for he might go to Clarissa's party, or he
might go to one of the Halls, or he might settle in and read an
absorbing book written by a man he used to know at Oxford.  And if
he did retire, that's what he'd do--write books.  He would go to
Oxford and poke about in the Bodleian.  Vainly the dark, adorably
pretty girl ran to the end of the terrace; vainly waved her hand;
vainly cried she didn't care a straw what people said.  There he
was, the man she thought the world of, the perfect gentleman, the
fascinating, the distinguished (and his age made not the least
difference to her), padding about a room in an hotel in Bloomsbury,
shaving, washing, continuing, as he took up cans, put down razors,
to poke about in the Bodleian, and get at the truth about one or
two little matters that interested him.  And he would have a chat
with whoever it might be, and so come to disregard more and more
precise hours for lunch, and miss engagements, and when Daisy asked
him, as she would, for a kiss, a scene, fail to come up to the
scratch (though he was genuinely devoted to her)--in short it might
be happier, as Mrs. Burgess said, that she should forget him, or
merely remember him as he was in August 1922, like a figure
standing at the cross roads at dusk, which grows more and more
remote as the dog-cart spins away, carrying her securely fastened
to the back seat, though her arms are outstretched, and as she sees
the figure dwindle and disappear still she cries out how she would
do anything in the world, anything, anything, anything. . . .

He never knew what people thought.  It became more and more
difficult for him to concentrate.  He became absorbed; he became
busied with his own concerns; now surly, now gay; dependent on
women, absent-minded, moody, less and less able (so he thought as
he shaved) to understand why Clarissa couldn't simply find them a
lodging and be nice to Daisy; introduce her.  And then he could
just--just do what? just haunt and hover (he was at the moment
actually engaged in sorting out various keys, papers), swoop and
taste, be alone, in short, sufficient to himself; and yet nobody of
course was more dependent upon others (he buttoned his waistcoat);
it had been his undoing.  He could not keep out of smoking-rooms,
liked colonels, liked golf, liked bridge, and above all women's
society, and the fineness of their companionship, and their
faithfulness and audacity and greatness in loving which though it
had its drawbacks seemed to him (and the dark, adorably pretty face
was on top of the envelopes) so wholly admirable, so splendid a
flower to grow on the crest of human life, and yet he could not
come up to the scratch, being always apt to see round things
(Clarissa had sapped something in him permanently), and to tire
very easily of mute devotion and to want variety in love, though it
would make him furious if Daisy loved anybody else, furious! for he
was jealous, uncontrollably jealous by temperament.  He suffered
tortures!  But where was his knife; his watch; his seals, his note-
case, and Clarissa's letter which he would not read again but liked
to think of, and Daisy's photograph?  And now for dinner.

They were eating.

Sitting at little tables round vases, dressed or not dressed, with
their shawls and bags laid beside them, with their air of false
composure, for they were not used to so many courses at dinner, and
confidence, for they were able to pay for it, and strain, for they
had been running about London all day shopping, sightseeing; and
their natural curiosity, for they looked round and up as the nice-
looking gentleman in horn-rimmed spectacles came in, and their good
nature, for they would have been glad to do any little service,
such as lend a time-table or impart useful information, and their
desire, pulsing in them, tugging at them subterraneously, somehow
to establish connections if it were only a birthplace (Liverpool,
for example) in common or friends of the same name; with their
furtive glances, odd silences, and sudden withdrawals into family
jocularity and isolation; there they sat eating dinner when Mr.
Walsh came in and took his seat at a little table by the curtain.

It was not that he said anything, for being solitary he could only
address himself to the waiter; it was his way of looking at the
menu, of pointing his forefinger to a particular wine, of hitching
himself up to the table, of addressing himself seriously, not
gluttonously to dinner, that won him their respect; which, having
to remain unexpressed for the greater part of the meal, flared up
at the table where the Morrises sat when Mr. Walsh was heard to say
at the end of the meal, "Bartlett pears."  Why he should have
spoken so moderately yet firmly, with the air of a disciplinarian
well within his rights which are founded upon justice, neither
young Charles Morris, nor old Charles, neither Miss Elaine nor Mrs.
Morris knew.  But when he said, "Bartlett pears," sitting alone at
his table, they felt that he counted on their support in some
lawful demand; was champion of a cause which immediately became
their own, so that their eyes met his eyes sympathetically, and
when they all reached the smoking-room simultaneously, a little
talk between them became inevitable.

It was not very profound--only to the effect that London was
crowded; had changed in thirty years; that Mr. Morris preferred
Liverpool; that Mrs. Morris had been to the Westminster flower-
show, and that they had all seen the Prince of Wales.  Yet, thought
Peter Walsh, no family in the world can compare with the Morrises;
none whatever; and their relations to each other are perfect, and
they don't care a hang for the upper classes, and they like what
they like, and Elaine is training for the family business, and the
boy has won a scholarship at Leeds, and the old lady (who is about
his own age) has three more children at home; and they have two
motor cars, but Mr. Morris still mends the boots on Sunday: it is
superb, it is absolutely superb, thought Peter Walsh, swaying a
little backwards and forwards with his liqueur glass in his hand
among the hairy red chairs and ash-trays, feeling very well pleased
with himself, for the Morrises liked him.  Yes, they liked a man
who said, "Bartlett pears."  They liked him, he felt.

He would go to Clarissa's party.  (The Morrises moved off; but they
would meet again.)  He would go to Clarissa's party, because he
wanted to ask Richard what they were doing in India--the
conservative duffers.  And what's being acted?  And music. . . .
Oh yes, and mere gossip.

For this is the truth about our soul, he thought, our self, who
fish-like inhabits deep seas and plies among obscurities threading
her way between the boles of giant weeds, over sun-flickered spaces
and on and on into gloom, cold, deep, inscrutable; suddenly she
shoots to the surface and sports on the wind-wrinkled waves; that
is, has a positive need to brush, scrape, kindle herself,
gossiping.  What did the Government mean--Richard Dalloway would
know--to do about India?

Since it was a very hot night and the paper boys went by with
placards proclaiming in huge red letters that there was a heat-
wave, wicker chairs were placed on the hotel steps and there,
sipping, smoking, detached gentlemen sat.  Peter Walsh sat there.
One might fancy that day, the London day, was just beginning.  Like
a woman who had slipped off her print dress and white apron to
array herself in blue and pearls, the day changed, put off stuff,
took gauze, changed to evening, and with the same sigh of
exhilaration that a woman breathes, tumbling petticoats on the
floor, it too shed dust, heat, colour; the traffic thinned; motor
cars, tinkling, darting, succeeded the lumber of vans; and here and
there among the thick foliage of the squares an intense light hung.
I resign, the evening seemed to say, as it paled and faded above
the battlements and prominences, moulded, pointed, of hotel, flat,
and block of shops, I fade, she was beginning, I disappear, but
London would have none of it, and rushed her bayonets into the sky,
pinioned her, constrained her to partnership in her revelry.

For the great revolution of Mr. Willett's summer time had taken
place since Peter Walsh's last visit to England.  The prolonged
evening was new to him.  It was inspiriting, rather.  For as the
young people went by with their despatch-boxes, awfully glad to be
free, proud too, dumbly, of stepping this famous pavement, joy of a
kind, cheap, tinselly, if you like, but all the same rapture,
flushed their faces.  They dressed well too; pink stockings; pretty
shoes.  They would now have two hours at the pictures.  It
sharpened, it refined them, the yellow-blue evening light; and on
the leaves in the square shone lurid, livid--they looked as if
dipped in sea water--the foliage of a submerged city.  He was
astonished by the beauty; it was encouraging too, for where the
returned Anglo-Indian sat by rights (he knew crowds of them) in the
Oriental Club biliously summing up the ruin of the world, here was
he, as young as ever; envying young people their summer time and
the rest of it, and more than suspecting from the words of a girl,
from a housemaid's laughter--intangible things you couldn't lay
your hands on--that shift in the whole pyramidal accumulation which
in his youth had seemed immovable.  On top of them it had pressed;
weighed them down, the women especially, like those flowers
Clarissa's Aunt Helena used to press between sheets of grey
blotting-paper with Littré's dictionary on top, sitting under the
lamp after dinner.  She was dead now.  He had heard of her, from
Clarissa, losing the sight of one eye.  It seemed so fitting--one
of nature's masterpieces--that old Miss Parry should turn to glass.
She would die like some bird in a frost gripping her perch.  She
belonged to a different age, but being so entire, so complete,
would always stand up on the horizon, stone-white, eminent, like a
lighthouse marking some past stage on this adventurous, long, long
voyage, this interminable (he felt for a copper to buy a paper and
read about Surrey and Yorkshire--he had held out that copper
millions of times.  Surrey was all out once more)--this
interminable life.  But cricket was no mere game.  Cricket was
important.  He could never help reading about cricket.  He read the
scores in the stop press first, then how it was a hot day; then
about a murder case.  Having done things millions of times enriched
them, though it might be said to take the surface off.  The past
enriched, and experience, and having cared for one or two people,
and so having acquired the power which the young lack, of cutting
short, doing what one likes, not caring a rap what people say and
coming and going without any very great expectations (he left his
paper on the table and moved off), which however (and he looked for
his hat and coat) was not altogether true of him, not to-night, for
here he was starting to go to a party, at his age, with the belief
upon him that he was about to have an experience.  But what?

Beauty anyhow.  Not the crude beauty of the eye.  It was not beauty
pure and simple--Bedford Place leading into Russell Square.  It was
straightness and emptiness of course; the symmetry of a corridor;
but it was also windows lit up, a piano, a gramophone sounding; a
sense of pleasure-making hidden, but now and again emerging when,
through the uncurtained window, the window left open, one saw
parties sitting over tables, young people slowly circling,
conversations between men and women, maids idly looking out (a
strange comment theirs, when work was done), stockings drying on
top ledges, a parrot, a few plants.  Absorbing, mysterious, of
infinite richness, this life.  And in the large square where the
cabs shot and swerved so quick, there were loitering couples,
dallying, embracing, shrunk up under the shower of a tree; that was
moving; so silent, so absorbed, that one passed, discreetly,
timidly, as if in the presence of some sacred ceremony to interrupt
which would have been impious.  That was interesting.  And so on
into the flare and glare.

His light overcoat blew open, he stepped with indescribable
idiosyncrasy, lent a little forward, tripped, with his hands behind
his back and his eyes still a little hawklike; he tripped through
London, towards Westminster, observing.

Was everybody dining out, then?  Doors were being opened here by a
footman to let issue a high-stepping old dame, in buckled shoes,
with three purple ostrich feathers in her hair.  Doors were being
opened for ladies wrapped like mummies in shawls with bright
flowers on them, ladies with bare heads.  And in respectable
quarters with stucco pillars through small front gardens lightly
swathed with combs in their hair (having run up to see the
children), women came; men waited for them, with their coats
blowing open, and the motor started.  Everybody was going out.
What with these doors being opened, and the descent and the start,
it seemed as if the whole of London were embarking in little boats
moored to the bank, tossing on the waters, as if the whole place
were floating off in carnival.  And Whitehall was skated over,
silver beaten as it was, skated over by spiders, and there was a
sense of midges round the arc lamps; it was so hot that people
stood about talking.  And here in Westminster was a retired Judge,
presumably, sitting four square at his house door dressed all in
white.  An Anglo-Indian presumably.

And here a shindy of brawling women, drunken women; here only a
policeman and looming houses, high houses, domed houses, churches,
parliaments, and the hoot of a steamer on the river, a hollow misty
cry.  But it was her street, this, Clarissa's; cabs were rushing
round the corner, like water round the piers of a bridge, drawn
together, it seemed to him because they bore people going to her
party, Clarissa's party.

The cold stream of visual impressions failed him now as if the eye
were a cup that overflowed and let the rest run down its china
walls unrecorded.  The brain must wake now.  The body must contract
now, entering the house, the lighted house, where the door stood
open, where the motor cars were standing, and bright women
descending: the soul must brave itself to endure.  He opened the
big blade of his pocket-knife.



Lucy came running full tilt downstairs, having just nipped in to
the drawing-room to smooth a cover, to straighten a chair, to pause
a moment and feel whoever came in must think how clean, how bright,
how beautifully cared for, when they saw the beautiful silver, the
brass fire-irons, the new chair-covers, and the curtains of yellow
chintz: she appraised each; heard a roar of voices; people already
coming up from dinner; she must fly!

The Prime Minister was coming, Agnes said: so she had heard them
say in the dining-room, she said, coming in with a tray of glasses.
Did it matter, did it matter in the least, one Prime Minister more
or less?  It made no difference at this hour of the night to Mrs.
Walker among the plates, saucepans, cullenders, frying-pans,
chicken in aspic, ice-cream freezers, pared crusts of bread,
lemons, soup tureens, and pudding basins which, however hard they
washed up in the scullery seemed to be all on top of her, on the
kitchen table, on chairs, while the fire blared and roared, the
electric lights glared, and still supper had to be laid.  All she
felt was, one Prime Minister more or less made not a scrap of
difference to Mrs. Walker.

The ladies were going upstairs already, said Lucy; the ladies were
going up, one by one, Mrs. Dalloway walking last and almost always
sending back some message to the kitchen, "My love to Mrs. Walker,"
that was it one night.  Next morning they would go over the dishes--
the soup, the salmon; the salmon, Mrs. Walker knew, as usual
underdone, for she always got nervous about the pudding and left it
to Jenny; so it happened, the salmon was always underdone.  But
some lady with fair hair and silver ornaments had said, Lucy said,
about the entrée, was it really made at home?  But it was the
salmon that bothered Mrs. Walker, as she spun the plates round and
round, and pulled in dampers and pulled out dampers; and there came
a burst of laughter from the dining-room; a voice speaking; then
another burst of laughter--the gentlemen enjoying themselves when
the ladies had gone.  The tokay, said Lucy running in.  Mr.
Dalloway had sent for the tokay, from the Emperor's cellars, the
Imperial Tokay.

It was borne through the kitchen.  Over her shoulder Lucy reported
how Miss Elizabeth looked quite lovely; she couldn't take her eyes
off her; in her pink dress, wearing the necklace Mr. Dalloway had
given her.  Jenny must remember the dog, Miss Elizabeth's fox-
terrier, which, since it bit, had to be shut up and might,
Elizabeth thought, want something.  Jenny must remember the dog.
But Jenny was not going upstairs with all those people about.
There was a motor at the door already!  There was a ring at the
bell--and the gentlemen still in the dining-room, drinking tokay!

There, they were going upstairs; that was the first to come, and
now they would come faster and faster, so that Mrs. Parkinson
(hired for parties) would leave the hall door ajar, and the hall
would be full of gentlemen waiting (they stood waiting, sleeking
down their hair) while the ladies took their cloaks off in the room
along the passage; where Mrs. Barnet helped them, old Ellen Barnet,
who had been with the family for forty years, and came every summer
to help the ladies, and remembered mothers when they were girls,
and though very unassuming did shake hands; said "milady" very
respectfully, yet had a humorous way with her, looking at the young
ladies, and ever so tactfully helping Lady Lovejoy, who had some
trouble with her underbodice.  And they could not help feeling,
Lady Lovejoy and Miss Alice, that some little privilege in the
matter of brush and comb, was awarded them having known Mrs.
Barnet--"thirty years, milady," Mrs. Barnet supplied her.  Young
ladies did not use to rouge, said Lady Lovejoy, when they stayed at
Bourton in the old days.  And Miss Alice didn't need rouge, said
Mrs. Barnet, looking at her fondly.  There Mrs. Barnet would sit,
in the cloakroom, patting down the furs, smoothing out the Spanish
shawls, tidying the dressing-table, and knowing perfectly well, in
spite of the furs and the embroideries, which were nice ladies,
which were not.  The dear old body, said Lady Lovejoy, mounting the
stairs, Clarissa's old nurse.

And then Lady Lovejoy stiffened.  "Lady and Miss Lovejoy," she said
to Mr. Wilkins (hired for parties).  He had an admirable manner, as
he bent and straightened himself, bent and straightened himself and
announced with perfect impartiality "Lady and Miss Lovejoy . . .
Sir John and Lady Needham . . . Miss Weld . . . Mr. Walsh."  His
manner was admirable; his family life must be irreproachable,
except that it seemed impossible that a being with greenish lips
and shaven cheeks could ever have blundered into the nuisance of
children.

"How delightful to see you!" said Clarissa.  She said it to every
one.  How delightful to see you!  She was at her worst--effusive,
insincere.  It was a great mistake to have come.  He should have
stayed at home and read his book, thought Peter Walsh; should have
gone to a music hall; he should have stayed at home, for he knew no
one.

Oh dear, it was going to be a failure; a complete failure, Clarissa
felt it in her bones as dear old Lord Lexham stood there
apologising for his wife who had caught cold at the Buckingham
Palace garden party.  She could see Peter out of the tail of her
eye, criticising her, there, in that corner.  Why, after all, did
she do these things?  Why seek pinnacles and stand drenched in
fire?  Might it consume her anyhow!  Burn her to cinders!  Better
anything, better brandish one's torch and hurl it to earth than
taper and dwindle away like some Ellie Henderson!  It was
extraordinary how Peter put her into these states just by coming
and standing in a corner.  He made her see herself; exaggerate.  It
was idiotic.  But why did he come, then, merely to criticise?  Why
always take, never give?  Why not risk one's one little point of
view?  There he was wandering off, and she must speak to him.  But
she would not get the chance.  Life was that--humiliation,
renunciation.  What Lord Lexham was saying was that his wife would
not wear her furs at the garden party because "my dear, you ladies
are all alike"--Lady Lexham being seventy-five at least!  It was
delicious, how they petted each other, that old couple.  She did
like old Lord Lexham.  She did think it mattered, her party, and it
made her feel quite sick to know that it was all going wrong, all
falling flat.  Anything, any explosion, any horror was better than
people wandering aimlessly, standing in a bunch at a corner like
Ellie Henderson, not even caring to hold themselves upright.

Gently the yellow curtain with all the birds of Paradise blew out
and it seemed as if there were a flight of wings into the room,
right out, then sucked back.  (For the windows were open.)  Was it
draughty, Ellie Henderson wondered?  She was subject to chills.
But it did not matter that she should come down sneezing to-morrow;
it was the girls with their naked shoulders she thought of, being
trained to think of others by an old father, an invalid, late vicar
of Bourton, but he was dead now; and her chills never went to her
chest, never.  It was the girls she thought of, the young girls
with their bare shoulders, she herself having always been a wisp of
a creature, with her thin hair and meagre profile; though now, past
fifty, there was beginning to shine through some mild beam,
something purified into distinction by years of self-abnegation but
obscured again, perpetually, by her distressing gentility, her
panic fear, which arose from three hundred pounds' income, and her
weaponless state (she could not earn a penny) and it made her
timid, and more and more disqualified year by year to meet well-
dressed people who did this sort of thing every night of the
season, merely telling their maids "I'll wear so and so," whereas
Ellie Henderson ran out nervously and bought cheap pink flowers,
half a dozen, and then threw a shawl over her old black dress.  For
her invitation to Clarissa's party had come at the last moment.
She was not quite happy about it.  She had a sort of feeling that
Clarissa had not meant to ask her this year.

Why should she?  There was no reason really, except that they had
always known each other.  Indeed, they were cousins.  But naturally
they had rather drifted apart, Clarissa being so sought after.  It
was an event to her, going to a party.  It was quite a treat just
to see the lovely clothes.  Wasn't that Elizabeth, grown up, with
her hair done in the fashionable way, in the pink dress?  Yet she
could not be more than seventeen.  She was very, very handsome.
But girls when they first came out didn't seem to wear white as
they used.  (She must remember everything to tell Edith.)  Girls
wore straight frocks, perfectly tight, with skirts well above the
ankles.  It was not becoming, she thought.

So, with her weak eyesight, Ellie Henderson craned rather forward,
and it wasn't so much she who minded not having any one to talk to
(she hardly knew anybody there), for she felt that they were all
such interesting people to watch; politicians presumably; Richard
Dalloway's friends; but it was Richard himself who felt that he
could not let the poor creature go on standing there all the
evening by herself.

"Well, Ellie, and how's the world treating YOU?" he said in his
genial way, and Ellie Henderson, getting nervous and flushing and
feeling that it was extraordinarily nice of him to come and talk to
her, said that many people really felt the heat more than the cold.

"Yes, they do," said Richard Dalloway.  "Yes."

But what more did one say?

"Hullo, Richard," said somebody, taking him by the elbow, and, good
Lord, there was old Peter, old Peter Walsh.  He was delighted to
see him--ever so pleased to see him!  He hadn't changed a bit.  And
off they went together walking right across the room, giving each
other little pats, as if they hadn't met for a long time, Ellie
Henderson thought, watching them go, certain she knew that man's
face.  A tall man, middle aged, rather fine eyes, dark, wearing
spectacles, with a look of John Burrows.  Edith would be sure to
know.

The curtain with its flight of birds of Paradise blew out again.
And Clarissa saw--she saw Ralph Lyon beat it back, and go on
talking.  So it wasn't a failure after all! it was going to be all
right now--her party.  It had begun.  It had started.  But it was
still touch and go.  She must stand there for the present.  People
seemed to come in a rush.

Colonel and Mrs. Garrod . . . Mr. Hugh Whitbread . . . Mr. Bowley
. . . Mrs. Hilbery . . . Lady Mary Maddox . . . Mr. Quin . . .
intoned Wilkin.  She had six or seven words with each, and they
went on, they went into the rooms; into something now, not nothing,
since Ralph Lyon had beat back the curtain.

And yet for her own part, it was too much of an effort.  She was
not enjoying it.  It was too much like being--just anybody,
standing there; anybody could do it; yet this anybody she did a
little admire, couldn't help feeling that she had, anyhow, made
this happen, that it marked a stage, this post that she felt
herself to have become, for oddly enough she had quite forgotten
what she looked like, but felt herself a stake driven in at the top
of her stairs.  Every time she gave a party she had this feeling of
being something not herself, and that every one was unreal in one
way; much more real in another.  It was, she thought, partly their
clothes, partly being taken out of their ordinary ways, partly the
background, it was possible to say things you couldn't say anyhow
else, things that needed an effort; possible to go much deeper.
But not for her; not yet anyhow.

"How delightful to see you!" she said.  Dear old Sir Harry!  He
would know every one.

And what was so odd about it was the sense one had as they came up
the stairs one after another, Mrs. Mount and Celia, Herbert Ainsty,
Mrs. Dakers--oh and Lady Bruton!

"How awfully good of you to come!" she said, and she meant it--it
was odd how standing there one felt them going on, going on, some
quite old, some . . .

WHAT name?  Lady Rosseter?  But who on earth was Lady Rosseter?

"Clarissa!"  That voice!  It was Sally Seton!  Sally Seton! after
all these years!  She loomed through a mist.  For she hadn't looked
like THAT, Sally Seton, when Clarissa grasped the hot water can, to
think of her under this roof, under this roof!  Not like that!

All on top of each other, embarrassed, laughing, words tumbled out--
passing through London; heard from Clara Haydon; what a chance of
seeing you!  So I thrust myself in--without an invitation. . . .

One might put down the hot water can quite composedly.  The lustre
had gone out of her.  Yet it was extraordinary to see her again,
older, happier, less lovely.  They kissed each other, first this
cheek then that, by the drawing-room door, and Clarissa turned,
with Sally's hand in hers, and saw her rooms full, heard the roar
of voices, saw the candlesticks, the blowing curtains, and the
roses which Richard had given her.

"I have five enormous boys," said Sally.

She had the simplest egotism, the most open desire to be thought
first always, and Clarissa loved her for being still like that.  "I
can't believe it!" she cried, kindling all over with pleasure at
the thought of the past.

But alas, Wilkins; Wilkins wanted her; Wilkins was emitting in a
voice of commanding authority as if the whole company must be
admonished and the hostess reclaimed from frivolity, one name:

"The Prime Minister," said Peter Walsh.

The Prime Minister?  Was it really?  Ellie Henderson marvelled.
What a thing to tell Edith!

One couldn't laugh at him.  He looked so ordinary.  You might have
stood him behind a counter and bought biscuits--poor chap, all
rigged up in gold lace.  And to be fair, as he went his rounds,
first with Clarissa then with Richard escorting him, he did it very
well.  He tried to look somebody.  It was amusing to watch.  Nobody
looked at him.  They just went on talking, yet it was perfectly
plain that they all knew, felt to the marrow of their bones, this
majesty passing; this symbol of what they all stood for, English
society.  Old Lady Bruton, and she looked very fine too, very
stalwart in her lace, swam up, and they withdrew into a little room
which at once became spied upon, guarded, and a sort of stir and
rustle rippled through every one, openly: the Prime Minister!

Lord, lord, the snobbery of the English! thought Peter Walsh,
standing in the corner.  How they loved dressing up in gold lace
and doing homage!  There!  That must be, by Jove it was, Hugh
Whitbread, snuffing round the precincts of the great, grown rather
fatter, rather whiter, the admirable Hugh!

He looked always as if he were on duty, thought Peter, a
privileged, but secretive being, hoarding secrets which he would
die to defend, though it was only some little piece of tittle-
tattle dropped by a court footman, which would be in all the papers
tomorrow.  Such were his rattles, his baubles, in playing with
which he had grown white, come to the verge of old age, enjoying
the respect and affection of all who had the privilege of knowing
this type of the English public school man.  Inevitably one made up
things like that about Hugh; that was his style; the style of those
admirable letters which Peter had read thousands of miles across
the sea in the Times, and had thanked God he was out of that
pernicious hubble-bubble if it were only to hear baboons chatter
and coolies beat their wives.  An olive-skinned youth from one of
the Universities stood obsequiously by.  Him he would patronise,
initiate, teach how to get on.  For he liked nothing better than
doing kindnesses, making the hearts of old ladies palpitate with
the joy of being thought of in their age, their affliction,
thinking themselves quite forgotten, yet here was dear Hugh driving
up and spending an hour talking of the past, remembering trifles,
praising the home-made cake, though Hugh might eat cake with a
Duchess any day of his life, and, to look at him, probably did
spend a good deal of time in that agreeable occupation.  The All-
judging, the All-merciful, might excuse.  Peter Walsh had no mercy.
Villains there must be, and God knows the rascals who get hanged
for battering the brains of a girl out in a train do less harm on
the whole than Hugh Whitbread and his kindness.  Look at him now,
on tiptoe, dancing forward, bowing and scraping, as the Prime
Minister and Lady Bruton emerged, intimating for all the world to
see that he was privileged to say something, something private, to
Lady Bruton as she passed.  She stopped.  She wagged her fine old
head.  She was thanking him presumably for some piece of servility.
She had her toadies, minor officials in Government offices who ran
about putting through little jobs on her behalf, in return for
which she gave them luncheon.  But she derived from the eighteenth
century.  She was all right.

And now Clarissa escorted her Prime Minister down the room,
prancing, sparkling, with the stateliness of her grey hair.  She
wore ear-rings, and a silver-green mermaid's dress.  Lolloping on
the waves and braiding her tresses she seemed, having that gift
still; to be; to exist; to sum it all up in the moment as she
passed; turned, caught her scarf in some other woman's dress,
unhitched it, laughed, all with the most perfect ease and air of a
creature floating in its element.  But age had brushed her; even as
a mermaid might behold in her glass the setting sun on some very
clear evening over the waves.  There was a breath of tenderness;
her severity, her prudery, her woodenness were all warmed through
now, and she had about her as she said good-bye to the thick gold-
laced man who was doing his best, and good luck to him, to look
important, an inexpressible dignity; an exquisite cordiality; as if
she wished the whole world well, and must now, being on the very
verge and rim of things, take her leave.  So she made him think.
(But he was not in love.)

Indeed, Clarissa felt, the Prime Minister had been good to come.
And, walking down the room with him, with Sally there and Peter
there and Richard very pleased, with all those people rather
inclined, perhaps, to envy, she had felt that intoxication of the
moment, that dilatation of the nerves of the heart itself till it
seemed to quiver, steeped, upright;--yes, but after all it was what
other people felt, that; for, though she loved it and felt it
tingle and sting, still these semblances, these triumphs (dear old
Peter, for example, thinking her so brilliant), had a hollowness;
at arm's length they were, not in the heart; and it might be that
she was growing old but they satisfied her no longer as they used;
and suddenly, as she saw the Prime Minister go down the stairs, the
gilt rim of the Sir Joshua picture of the little girl with a muff
brought back Kilman with a rush; Kilman her enemy.  That was
satisfying; that was real.  Ah, how she hated her--hot,
hypocritical, corrupt; with all that power; Elizabeth's seducer;
the woman who had crept in to steal and defile (Richard would say,
What nonsense!).  She hated her: she loved her.  It was enemies one
wanted, not friends--not Mrs. Durrant and Clara, Sir William and
Lady Bradshaw, Miss Truelock and Eleanor Gibson (whom she saw
coming upstairs).  They must find her if they wanted her.  She was
for the party!

There was her old friend Sir Harry.

"Dear Sir Harry!" she said, going up to the fine old fellow who had
produced more bad pictures than any other two Academicians in the
whole of St. John's Wood (they were always of cattle, standing in
sunset pools absorbing moisture, or signifying, for he had a
certain range of gesture, by the raising of one foreleg and the
toss of the antlers, "the Approach of the Stranger"--all his
activities, dining out, racing, were founded on cattle standing
absorbing moisture in sunset pools).

"What are you laughing at?" she asked him.  For Willie Titcomb and
Sir Harry and Herbert Ainsty were all laughing.  But no.  Sir Harry
could not tell Clarissa Dalloway (much though he liked her; of her
type he thought her perfect, and threatened to paint her) his
stories of the music hall stage.  He chaffed her about her party.
He missed his brandy.  These circles, he said, were above him.  But
he liked her; respected her, in spite of her damnable, difficult
upper-class refinement, which made it impossible to ask Clarissa
Dalloway to sit on his knee.  And up came that wandering will-o'-
the-wisp, that vagulous phosphorescence, old Mrs. Hilbery,
stretching her hands to the blaze of his laughter (about the Duke
and the Lady), which, as she heard it across the room, seemed to
reassure her on a point which sometimes bothered her if she woke
early in the morning and did not like to call her maid for a cup of
tea; how it is certain we must die.

"They won't tell us their stories," said Clarissa.

"Dear Clarissa!" exclaimed Mrs. Hilbery.  She looked to-night, she
said, so like her mother as she first saw her walking in a garden
in a grey hat.

And really Clarissa's eyes filled with tears.  Her mother, walking
in a garden!  But alas, she must go.

For there was Professor Brierly, who lectured on Milton, talking to
little Jim Hutton (who was unable even for a party like this to
compass both tie and waistcoat or make his hair lie flat), and even
at this distance they were quarrelling, she could see.  For
Professor Brierly was a very queer fish.  With all those degrees,
honours, lectureships between him and the scribblers he suspected
instantly an atmosphere not favourable to his queer compound; his
prodigious learning and timidity; his wintry charm without
cordiality; his innocence blent with snobbery; he quivered if made
conscious by a lady's unkempt hair, a youth's boots, of an
underworld, very creditable doubtless, of rebels, of ardent young
people; of would-be geniuses, and intimated with a little toss of
the head, with a sniff--Humph!--the value of moderation; of some
slight training in the classics in order to appreciate Milton.
Professor Brierly (Clarissa could see) wasn't hitting it off with
little Jim Hutton (who wore red socks, his black being at the
laundry) about Milton.  She interrupted.

She said she loved Bach.  So did Hutton.  That was the bond between
them, and Hutton (a very bad poet) always felt that Mrs. Dalloway
was far the best of the great ladies who took an interest in art.
It was odd how strict she was.  About music she was purely
impersonal.  She was rather a prig.  But how charming to look at!
She made her house so nice if it weren't for her Professors.
Clarissa had half a mind to snatch him off and set him down at the
piano in the back room.  For he played divinely.

"But the noise!" she said.  "The noise!"

"The sign of a successful party."  Nodding urbanely, the Professor
stepped delicately off.

"He knows everything in the whole world about Milton," said
Clarissa.

"Does he indeed?" said Hutton, who would imitate the Professor
throughout Hampstead; the Professor on Milton; the Professor on
moderation; the Professor stepping delicately off.

But she must speak to that couple, said Clarissa, Lord Gayton and
Nancy Blow.

Not that THEY added perceptibly to the noise of the party.  They
were not talking (perceptibly) as they stood side by side by the
yellow curtains.  They would soon be off elsewhere, together; and
never had very much to say in any circumstances.  They looked; that
was all.  That was enough.  They looked so clean, so sound, she
with an apricot bloom of powder and paint, but he scrubbed, rinsed,
with the eyes of a bird, so that no ball could pass him or stroke
surprise him.  He struck, he leapt, accurately, on the spot.
Ponies' mouths quivered at the end of his reins.  He had his
honours, ancestral monuments, banners hanging in the church at
home.  He had his duties; his tenants; a mother and sisters; had
been all day at Lords, and that was what they were talking about--
cricket, cousins, the movies--when Mrs. Dalloway came up.  Lord
Gayton liked her most awfully.  So did Miss Blow.  She had such
charming manners.

"It is angelic--it is delicious of you to have come!" she said.
She loved Lords; she loved youth, and Nancy, dressed at enormous
expense by the greatest artists in Paris, stood there looking as if
her body had merely put forth, of its own accord, a green frill.

"I had meant to have dancing," said Clarissa.

For the young people could not talk.  And why should they?  Shout,
embrace, swing, be up at dawn; carry sugar to ponies; kiss and
caress the snouts of adorable chows; and then all tingling and
streaming, plunge and swim.  But the enormous resources of the
English language, the power it bestows, after all, of communicating
feelings (at their age, she and Peter would have been arguing all
the evening), was not for them.  They would solidify young.  They
would be good beyond measure to the people on the estate, but
alone, perhaps, rather dull.

"What a pity!" she said.  "I had hoped to have dancing."

It was so extraordinarily nice of them to have come!  But talk of
dancing!  The rooms were packed.

There was old Aunt Helena in her shawl.  Alas, she must leave them--
Lord Gayton and Nancy Blow.  There was old Miss Parry, her aunt.

For Miss Helena Parry was not dead: Miss Parry was alive.  She was
past eighty.  She ascended staircases slowly with a stick.  She was
placed in a chair (Richard had seen to it).  People who had known
Burma in the 'seventies were always led up to her.  Where had Peter
got to?  They used to be such friends.  For at the mention of
India, or even Ceylon, her eyes (only one was glass) slowly
deepened, became blue, beheld, not human beings--she had no tender
memories, no proud illusions about Viceroys, Generals, Mutinies--it
was orchids she saw, and mountain passes and herself carried on the
backs of coolies in the 'sixties over solitary peaks; or descending
to uproot orchids (startling blossoms, never beheld before) which
she painted in water-colour; an indomitable Englishwoman, fretful
if disturbed by the War, say, which dropped a bomb at her very
door, from her deep meditation over orchids and her own figure
journeying in the 'sixties in India--but here was Peter.

"Come and talk to Aunt Helena about Burma," said Clarissa.

And yet he had not had a word with her all the evening!

"We will talk later," said Clarissa, leading him up to Aunt Helena,
in her white shawl, with her stick.

"Peter Walsh," said Clarissa.

That meant nothing.

Clarissa had asked her.  It was tiring; it was noisy; but Clarissa
had asked her.  So she had come.  It was a pity that they lived in
London--Richard and Clarissa.  If only for Clarissa's health it
would have been better to live in the country.  But Clarissa had
always been fond of society.

"He has been in Burma," said Clarissa.

Ah.  She could not resist recalling what Charles Darwin had said
about her little book on the orchids of Burma.

(Clarissa must speak to Lady Bruton.)

No doubt it was forgotten now, her book on the orchids of Burma,
but it went into three editions before 1870, she told Peter.  She
remembered him now.  He had been at Bourton (and he had left her,
Peter Walsh remembered, without a word in the drawing-room that
night when Clarissa had asked him to come boating).

"Richard so much enjoyed his lunch party," said Clarissa to Lady
Bruton.

"Richard was the greatest possible help," Lady Bruton replied.  "He
helped me to write a letter.  And how are you?"

"Oh, perfectly well!" said Clarissa.  (Lady Bruton detested illness
in the wives of politicians.)

"And there's Peter Walsh!" said Lady Bruton (for she could never
think of anything to say to Clarissa; though she liked her.  She
had lots of fine qualities; but they had nothing in common--she and
Clarissa.  It might have been better if Richard had married a woman
with less charm, who would have helped him more in his work.  He
had lost his chance of the Cabinet).  "There's Peter Walsh!" she
said, shaking hands with that agreeable sinner, that very able
fellow who should have made a name for himself but hadn't (always
in difficulties with women), and, of course, old Miss Parry.
Wonderful old lady!

Lady Bruton stood by Miss Parry's chair, a spectral grenadier,
draped in black, inviting Peter Walsh to lunch; cordial; but
without small talk, remembering nothing whatever about the flora or
fauna of India.  She had been there, of course; had stayed with
three Viceroys; thought some of the Indian civilians uncommonly
fine fellows; but what a tragedy it was--the state of India!  The
Prime Minister had just been telling her (old Miss Parry huddled up
in her shawl, did not care what the Prime Minister had just been
telling her), and Lady Bruton would like to have Peter Walsh's
opinion, he being fresh from the centre, and she would get Sir
Sampson to meet him, for really it prevented her from sleeping at
night, the folly of it, the wickedness she might say, being a
soldier's daughter.  She was an old woman now, not good for much.
But her house, her servants, her good friend Milly Brush--did he
remember her?--were all there only asking to be used if--if they
could be of help, in short.  For she never spoke of England, but
this isle of men, this dear, dear land, was in her blood (without
reading Shakespeare), and if ever a woman could have worn the
helmet and shot the arrow, could have led troops to attack, ruled
with indomitable justice barbarian hordes and lain under a shield
noseless in a church, or made a green grass mound on some primeval
hillside, that woman was Millicent Bruton.  Debarred by her sex and
some truancy, too, of the logical faculty (she found it impossible
to write a letter to the Times), she had the thought of Empire
always at hand, and had acquired from her association with that
armoured goddess her ramrod bearing, her robustness of demeanour,
so that one could not figure her even in death parted from the
earth or roaming territories over which, in some spiritual shape,
the Union Jack had ceased to fly.  To be not English even among the
dead--no, no!  Impossible!

But was it Lady Bruton (whom she used to know)?  Was it Peter Walsh
grown grey?  Lady Rosseter asked herself (who had been Sally
Seton).  It was old Miss Parry certainly--the old aunt who used to
be so cross when she stayed at Bourton.  Never should she forget
running along the passage naked, and being sent for by Miss Parry!
And Clarissa! oh Clarissa!  Sally caught her by the arm.

Clarissa stopped beside them.

"But I can't stay," she said.  "I shall come later.  Wait," she
said, looking at Peter and Sally.  They must wait, she meant, until
all these people had gone.

"I shall come back," she said, looking at her old friends, Sally
and Peter, who were shaking hands, and Sally, remembering the past
no doubt, was laughing.

But her voice was wrung of its old ravishing richness; her eyes not
aglow as they used to be, when she smoked cigars, when she ran down
the passage to fetch her sponge bag, without a stitch of clothing
on her, and Ellen Atkins asked, What if the gentlemen had met her?
But everybody forgave her.  She stole a chicken from the larder
because she was hungry in the night; she smoked cigars in her
bedroom; she left a priceless book in the punt.  But everybody
adored her (except perhaps Papa).  It was her warmth; her vitality--
she would paint, she would write.  Old women in the village never
to this day forgot to ask after "your friend in the red cloak who
seemed so bright."  She accused Hugh Whitbread, of all people (and
there he was, her old friend Hugh, talking to the Portuguese
Ambassador), of kissing her in the smoking-room to punish her for
saying that women should have votes.  Vulgar men did, she said.
And Clarissa remembered having to persuade her not to denounce him
at family prayers--which she was capable of doing with her daring,
her recklessness, her melodramatic love of being the centre of
everything and creating scenes, and it was bound, Clarissa used to
think, to end in some awful tragedy; her death; her martyrdom;
instead of which she had married, quite unexpectedly, a bald man
with a large buttonhole who owned, it was said, cotton mills at
Manchester.  And she had five boys!

She and Peter had settled down together.  They were talking: it
seemed so familiar--that they should be talking.  They would
discuss the past.  With the two of them (more even than with
Richard) she shared her past; the garden; the trees; old Joseph
Breitkopf singing Brahms without any voice; the drawing-room
wallpaper; the smell of the mats.  A part of this Sally must always
be; Peter must always be.  But she must leave them.  There were the
Bradshaws, whom she disliked.  She must go up to Lady Bradshaw (in
grey and silver, balancing like a sea-lion at the edge of its tank,
barking for invitations, Duchesses, the typical successful man's
wife), she must go up to Lady Bradshaw and say . . .

But Lady Bradshaw anticipated her.

"We are shockingly late, dear Mrs. Dalloway, we hardly dared to
come in," she said.

And Sir William, who looked very distinguished, with his grey hair
and blue eyes, said yes; they had not been able to resist the
temptation.  He was talking to Richard about that Bill probably,
which they wanted to get through the Commons.  Why did the sight of
him, talking to Richard, curl her up?  He looked what he was, a
great doctor.  A man absolutely at the head of his profession, very
powerful, rather worn.  For think what cases came before him--
people in the uttermost depths of misery; people on the verge of
insanity; husbands and wives.  He had to decide questions of
appalling difficulty.  Yet--what she felt was, one wouldn't like
Sir William to see one unhappy.  No; not that man.

"How is your son at Eton?" she asked Lady Bradshaw.

He had just missed his eleven, said Lady Bradshaw, because of the
mumps.  His father minded even more than he did, she thought
"being," she said, "nothing but a great boy himself."

Clarissa looked at Sir William, talking to Richard.  He did not
look like a boy--not in the least like a boy.  She had once gone
with some one to ask his advice.  He had been perfectly right;
extremely sensible.  But Heavens--what a relief to get out to the
street again!  There was some poor wretch sobbing, she remembered,
in the waiting-room.  But she did not know what it was--about Sir
William; what exactly she disliked.  Only Richard agreed with her,
"didn't like his taste, didn't like his smell."  But he was
extraordinarily able.  They were talking about this Bill.  Some
case, Sir William was mentioning, lowering his voice.  It had its
bearing upon what he was saying about the deferred effects of shell
shock.  There must be some provision in the Bill.

Sinking her voice, drawing Mrs. Dalloway into the shelter of a
common femininity, a common pride in the illustrious qualities of
husbands and their sad tendency to overwork, Lady Bradshaw (poor
goose--one didn't dislike her) murmured how, "just as we were
starting, my husband was called up on the telephone, a very sad
case.  A young man (that is what Sir William is telling Mr.
Dalloway) had killed himself.  He had been in the army."  Oh!
thought Clarissa, in the middle of my party, here's death, she
thought.

She went on, into the little room where the Prime Minister had gone
with Lady Bruton.  Perhaps there was somebody there.  But there was
nobody.  The chairs still kept the impress of the Prime Minister
and Lady Bruton, she turned deferentially, he sitting four-square,
authoritatively.  They had been talking about India.  There was
nobody.  The party's splendour fell to the floor, so strange it was
to come in alone in her finery.

What business had the Bradshaws to talk of death at her party?  A
young man had killed himself.  And they talked of it at her party--
the Bradshaws, talked of death.  He had killed himself--but how?
Always her body went through it first, when she was told, suddenly,
of an accident; her dress flamed, her body burnt.  He had thrown
himself from a window.  Up had flashed the ground; through him,
blundering, bruising, went the rusty spikes.  There he lay with a
thud, thud, thud in his brain, and then a suffocation of blackness.
So she saw it.  But why had he done it?  And the Bradshaws talked
of it at her party!

She had once thrown a shilling into the Serpentine, never anything
more.  But he had flung it away.  They went on living (she would
have to go back; the rooms were still crowded; people kept on
coming).  They (all day she had been thinking of Bourton, of Peter,
of Sally), they would grow old.  A thing there was that mattered; a
thing, wreathed about with chatter, defaced, obscured in her own
life, let drop every day in corruption, lies, chatter.  This he had
preserved.  Death was defiance.  Death was an attempt to
communicate; people feeling the impossibility of reaching the
centre which, mystically, evaded them; closeness drew apart;
rapture faded, one was alone.  There was an embrace in death.

But this young man who had killed himself--had he plunged holding
his treasure?  "If it were now to die, 'twere now to be most
happy," she had said to herself once, coming down in white.

Or there were the poets and thinkers.  Suppose he had had that
passion, and had gone to Sir William Bradshaw, a great doctor yet
to her obscurely evil, without sex or lust, extremely polite to
women, but capable of some indescribable outrage--forcing your
soul, that was it--if this young man had gone to him, and Sir
William had impressed him, like that, with his power, might he not
then have said (indeed she felt it now), Life is made intolerable;
they make life intolerable, men like that?

Then (she had felt it only this morning) there was the terror; the
overwhelming incapacity, one's parents giving it into one's hands,
this life, to be lived to the end, to be walked with serenely;
there was in the depths of her heart an awful fear.  Even now,
quite often if Richard had not been there reading the Times, so
that she could crouch like a bird and gradually revive, send
roaring up that immeasurable delight, rubbing stick to stick, one
thing with another, she must have perished.  But that young man had
killed himself.

Somehow it was her disaster--her disgrace.  It was her punishment
to see sink and disappear here a man, there a woman, in this
profound darkness, and she forced to stand here in her evening
dress.  She had schemed; she had pilfered.  She was never wholly
admirable.  She had wanted success.  Lady Bexborough and the rest
of it.  And once she had walked on the terrace at Bourton.

It was due to Richard; she had never been so happy.  Nothing could
be slow enough; nothing last too long.  No pleasure could equal,
she thought, straightening the chairs, pushing in one book on the
shelf, this having done with the triumphs of youth, lost herself in
the process of living, to find it, with a shock of delight, as the
sun rose, as the day sank.  Many a time had she gone, at Bourton
when they were all talking, to look at the sky; or seen it between
people's shoulders at dinner; seen it in London when she could not
sleep.  She walked to the window.

It held, foolish as the idea was, something of her own in it, this
country sky, this sky above Westminster.  She parted the curtains;
she looked.  Oh, but how surprising!--in the room opposite the old
lady stared straight at her!  She was going to bed.  And the sky.
It will be a solemn sky, she had thought, it will be a dusky sky,
turning away its cheek in beauty.  But there it was--ashen pale,
raced over quickly by tapering vast clouds.  It was new to her.
The wind must have risen.  She was going to bed, in the room
opposite.  It was fascinating to watch her, moving about, that old
lady, crossing the room, coming to the window.  Could she see her?
It was fascinating, with people still laughing and shouting in the
drawing-room, to watch that old woman, quite quietly, going to bed.
She pulled the blind now.  The clock began striking.  The young man
had killed himself; but she did not pity him; with the clock
striking the hour, one, two, three, she did not pity him, with all
this going on.  There! the old lady had put out her light! the
whole house was dark now with this going on, she repeated, and the
words came to her, Fear no more the heat of the sun.  She must go
back to them.  But what an extraordinary night!  She felt somehow
very like him--the young man who had killed himself.  She felt glad
that he had done it; thrown it away.  The clock was striking.  The
leaden circles dissolved in the air.  He made her feel the beauty;
made her feel the fun.  But she must go back.  She must assemble.
She must find Sally and Peter.  And she came in from the little
room.

"But where is Clarissa?" said Peter.  He was sitting on the sofa
with Sally.  (After all these years he really could not call her
"Lady Rosseter.")  "Where's the woman gone to?" he asked.  "Where's
Clarissa?"

Sally supposed, and so did Peter for the matter of that, that there
were people of importance, politicians, whom neither of them knew
unless by sight in the picture papers, whom Clarissa had to be nice
to, had to talk to.  She was with them.  Yet there was Richard
Dalloway not in the Cabinet.  He hadn't been a success, Sally
supposed?  For herself, she scarcely ever read the papers.  She
sometimes saw his name mentioned.  But then--well, she lived a very
solitary life, in the wilds, Clarissa would say, among great
merchants, great manufacturers, men, after all, who did things.
She had done things too!

"I have five sons!" she told him.

Lord, Lord, what a change had come over her! the softness of
motherhood; its egotism too.  Last time they met, Peter remembered,
had been among the cauliflowers in the moonlight, the leaves "like
rough bronze" she had said, with her literary turn; and she had
picked a rose.  She had marched him up and down that awful night,
after the scene by the fountain; he was to catch the midnight
train.  Heavens, he had wept!

That was his old trick, opening a pocket-knife, thought Sally,
always opening and shutting a knife when he got excited.  They had
been very, very intimate, she and Peter Walsh, when he was in love
with Clarissa, and there was that dreadful, ridiculous scene over
Richard Dalloway at lunch.  She had called Richard "Wickham."  Why
not call Richard "Wickham"?  Clarissa had flared up! and indeed
they had never seen each other since, she and Clarissa, not more
than half a dozen times perhaps in the last ten years.  And Peter
Walsh had gone off to India, and she had heard vaguely that he had
made an unhappy marriage, and she didn't know whether he had any
children, and she couldn't ask him, for he had changed.  He was
rather shrivelled-looking, but kinder, she felt, and she had a real
affection for him, for he was connected with her youth, and she
still had a little Emily Brontë he had given her, and he was to
write, surely?  In those days he was to write.

"Have you written?" she asked him, spreading her hand, her firm and
shapely hand, on her knee in a way he recalled.

"Not a word!" said Peter Walsh, and she laughed.

She was still attractive, still a personage, Sally Seton.  But who
was this Rosseter?  He wore two camellias on his wedding day--that
was all Peter knew of him.  "They have myriads of servants, miles
of conservatories," Clarissa wrote; something like that.  Sally
owned it with a shout of laughter.

"Yes, I have ten thousand a year"--whether before the tax was paid
or after, she couldn't remember, for her husband, "whom you must
meet," she said, "whom you would like," she said, did all that for
her.

And Sally used to be in rags and tatters.  She had pawned her
grandmother's ring which Marie Antoinette had given her great-
grandfather to come to Bourton.

Oh yes, Sally remembered; she had it still, a ruby ring which Marie
Antoinette had given her great-grandfather.  She never had a penny
to her name in those days, and going to Bourton always meant some
frightful pinch.  But going to Bourton had meant so much to her--
had kept her sane, she believed, so unhappy had she been at home.
But that was all a thing of the past--all over now, she said.  And
Mr. Parry was dead; and Miss Parry was still alive.  Never had he
had such a shock in his life! said Peter.  He had been quite
certain she was dead.  And the marriage had been, Sally supposed, a
success?  And that very handsome, very self-possessed young woman
was Elizabeth, over there, by the curtains, in red.

(She was like a poplar, she was like a river, she was like a
hyacinth, Willie Titcomb was thinking.  Oh how much nicer to be in
the country and do what she liked!  She could hear her poor dog
howling, Elizabeth was certain.)  She was not a bit like Clarissa,
Peter Walsh said.

"Oh, Clarissa!" said Sally.

What Sally felt was simply this.  She had owed Clarissa an enormous
amount.  They had been friends, not acquaintances, friends, and she
still saw Clarissa all in white going about the house with her
hands full of flowers--to this day tobacco plants made her think of
Bourton.  But--did Peter understand?--she lacked something.  Lacked
what was it?  She had charm; she had extraordinary charm.  But to
be frank (and she felt that Peter was an old friend, a real friend--
did absence matter? did distance matter?  She had often wanted to
write to him, but torn it up, yet felt he understood, for people
understand without things being said, as one realises growing old,
and old she was, had been that afternoon to see her sons at Eton,
where they had the mumps), to be quite frank then, how could
Clarissa have done it?--married Richard Dalloway? a sportsman, a
man who cared only for dogs.  Literally, when he came into the room
he smelt of the stables.  And then all this?  She waved her hand.

Hugh Whitbread it was, strolling past in his white waistcoat, dim,
fat, blind, past everything he looked, except self-esteem and
comfort.

"He's not going to recognise US," said Sally, and really she hadn't
the courage--so that was Hugh! the admirable Hugh!

"And what does he do?" she asked Peter.

He blacked the King's boots or counted bottles at Windsor, Peter
told her.  Peter kept his sharp tongue still!  But Sally must be
frank, Peter said.  That kiss now, Hugh's.

On the lips, she assured him, in the smoking-room one evening.  She
went straight to Clarissa in a rage.  Hugh didn't do such things!
Clarissa said, the admirable Hugh!  Hugh's socks were without
exception the most beautiful she had ever seen--and now his evening
dress.  Perfect!  And had he children?

"Everybody in the room has six sons at Eton," Peter told her,
except himself.  He, thank God, had none.  No sons, no daughters,
no wife.  Well, he didn't seem to mind, said Sally.  He looked
younger, she thought, than any of them.

But it had been a silly thing to do, in many ways, Peter said, to
marry like that; "a perfect goose she was," he said, but, he said,
"we had a splendid time of it," but how could that be?  Sally
wondered; what did he mean? and how odd it was to know him and yet
not know a single thing that had happened to him.  And did he say
it out of pride?  Very likely, for after all it must be galling for
him (though he was an oddity, a sort of sprite, not at all an
ordinary man), it must be lonely at his age to have no home,
nowhere to go to.  But he must stay with them for weeks and weeks.
Of course he would; he would love to stay with them, and that was
how it came out.  All these years the Dalloways had never been
once.  Time after time they had asked them.  Clarissa (for it was
Clarissa of course) would not come.  For, said Sally, Clarissa was
at heart a snob--one had to admit it, a snob.  And it was that that
was between them, she was convinced.  Clarissa thought she had
married beneath her, her husband being--she was proud of it--a
miner's son.  Every penny they had he had earned.  As a little boy
(her voice trembled) he had carried great sacks.

(And so she would go on, Peter felt, hour after hour; the miner's
son; people thought she had married beneath her; her five sons; and
what was the other thing--plants, hydrangeas, syringas, very, very
rare hibiscus lilies that never grow north of the Suez Canal, but
she, with one gardener in a suburb near Manchester, had beds of
them, positively beds!  Now all that Clarissa had escaped,
unmaternal as she was.)

A snob was she?  Yes, in many ways.  Where was she, all this time?
It was getting late.

"Yet," said Sally, "when I heard Clarissa was giving a party, I
felt I couldn't NOT come--must see her again (and I'm staying in
Victoria Street, practically next door).  So I just came without an
invitation.  But," she whispered, "tell me, do.  Who is this?"

It was Mrs. Hilbery, looking for the door.  For how late it was
getting!  And, she murmured, as the night grew later, as people
went, one found old friends; quiet nooks and corners; and the
loveliest views.  Did they know, she asked, that they were
surrounded by an enchanted garden?  Lights and trees and wonderful
gleaming lakes and the sky.  Just a few fairy lamps, Clarissa
Dalloway had said, in the back garden!  But she was a magician!  It
was a park. . . .  And she didn't know their names, but friends she
knew they were, friends without names, songs without words, always
the best.  But there were so many doors, such unexpected places,
she could not find her way.

"Old Mrs. Hilbery," said Peter; but who was that? that lady
standing by the curtain all the evening, without speaking?  He knew
her face; connected her with Bourton.  Surely she used to cut up
underclothes at the large table in the window?  Davidson, was that
her name?

"Oh, that is Ellie Henderson," said Sally.  Clarissa was really
very hard on her.  She was a cousin, very poor.  Clarissa WAS hard
on people.

She was rather, said Peter.  Yet, said Sally, in her emotional way,
with a rush of that enthusiasm which Peter used to love her for,
yet dreaded a little now, so effusive she might become--how
generous to her friends Clarissa was! and what a rare quality one
found it, and how sometimes at night or on Christmas Day, when she
counted up her blessings, she put that friendship first.  They were
young; that was it.  Clarissa was pure-hearted; that was it.  Peter
would think her sentimental.  So she was.  For she had come to feel
that it was the only thing worth saying--what one felt.  Cleverness
was silly.  One must say simply what one felt.

"But I do not know," said Peter Walsh, "what I feel."

Poor Peter, thought Sally.  Why did not Clarissa come and talk to
them?  That was what he was longing for.  She knew it.  All the
time he was thinking only of Clarissa, and was fidgeting with his
knife.

He had not found life simple, Peter said.  His relations with
Clarissa had not been simple.  It had spoilt his life, he said.
(They had been so intimate--he and Sally Seton, it was absurd not
to say it.)  One could not be in love twice, he said.  And what
could she say?  Still, it is better to have loved (but he would
think her sentimental--he used to be so sharp).  He must come and
stay with them in Manchester.  That is all very true, he said.  All
very true.  He would love to come and stay with them, directly he
had done what he had to do in London.

And Clarissa had cared for him more than she had ever cared for
Richard.  Sally was positive of that.

"No, no, no!" said Peter (Sally should not have said that--she went
too far).  That good fellow--there he was at the end of the room,
holding forth, the same as ever, dear old Richard.  Who was he
talking to? Sally asked, that very distinguished-looking man?
Living in the wilds as she did, she had an insatiable curiosity to
know who people were.  But Peter did not know.  He did not like his
looks, he said, probably a Cabinet Minister.  Of them all, Richard
seemed to him the best, he said--the most disinterested.

"But what has he done?" Sally asked.  Public work, she supposed.
And were they happy together?  Sally asked (she herself was
extremely happy); for, she admitted, she knew nothing about them,
only jumped to conclusions, as one does, for what can one know even
of the people one lives with every day? she asked.  Are we not all
prisoners?  She had read a wonderful play about a man who scratched
on the wall of his cell, and she had felt that was true of life--
one scratched on the wall.  Despairing of human relationships
(people were so difficult), she often went into her garden and got
from her flowers a peace which men and women never gave her.  But
no; he did not like cabbages; he preferred human beings, Peter
said.  Indeed, the young are beautiful, Sally said, watching
Elizabeth cross the room.  How unlike Clarissa at her age!  Could
he make anything of her?  She would not open her lips.  Not much,
not yet, Peter admitted.  She was like a lily, Sally said, a lily
by the side of a pool.  But Peter did not agree that we know
nothing.  We know everything, he said; at least he did.

But these two, Sally whispered, these two coming now (and really
she must go, if Clarissa did not come soon), this distinguished-
looking man and his rather common-looking wife who had been talking
to Richard--what could one know about people like that?

"That they're damnable humbugs," said Peter, looking at them
casually.  He made Sally laugh.

But Sir William Bradshaw stopped at the door to look at a picture.
He looked in the corner for the engraver's name.  His wife looked
too.  Sir William Bradshaw was so interested in art.

When one was young, said Peter, one was too much excited to know
people.  Now that one was old, fifty-two to be precise (Sally was
fifty-five, in body, she said, but her heart was like a girl's of
twenty); now that one was mature then, said Peter, one could watch,
one could understand, and one did not lose the power of feeling, he
said.  No, that is true, said Sally.  She felt more deeply, more
passionately, every year.  It increased, he said, alas, perhaps,
but one should be glad of it--it went on increasing in his
experience.  There was some one in India.  He would like to tell
Sally about her.  He would like Sally to know her.  She was
married, he said.  She had two small children.  They must all come
to Manchester, said Sally--he must promise before they left.

There's Elizabeth, he said, she feels not half what we feel, not
yet.  But, said Sally, watching Elizabeth go to her father, one can
see they are devoted to each other.  She could feel it by the way
Elizabeth went to her father.

For her father had been looking at her, as he stood talking to the
Bradshaws, and he had thought to himself, Who is that lovely girl?
And suddenly he realised that it was his Elizabeth, and he had not
recognised her, she looked so lovely in her pink frock!  Elizabeth
had felt him looking at her as she talked to Willie Titcomb.  So
she went to him and they stood together, now that the party was
almost over, looking at the people going, and the rooms getting
emptier and emptier, with things scattered on the floor.  Even
Ellie Henderson was going, nearly last of all, though no one had
spoken to her, but she had wanted to see everything, to tell Edith.
And Richard and Elizabeth were rather glad it was over, but Richard
was proud of his daughter.  And he had not meant to tell her, but
he could not help telling her.  He had looked at her, he said, and
he had wondered, Who is that lovely girl? and it was his daughter!
That did make her happy.  But her poor dog was howling.

"Richard has improved.  You are right," said Sally.  "I shall go
and talk to him.  I shall say goodnight.  What does the brain
matter," said Lady Rosseter, getting up, "compared with the heart?"

"I will come," said Peter, but he sat on for a moment.  What is
this terror? what is this ecstasy? he thought to himself.  What is
it that fills me with extraordinary excitement?

It is Clarissa, he said.

For there she was.

THE END

\end{document}
